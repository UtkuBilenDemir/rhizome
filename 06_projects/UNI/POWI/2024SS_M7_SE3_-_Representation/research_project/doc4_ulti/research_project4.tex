%%%%%%%%%%%%%%%%%%%%%%%%%%%% Define Article %%%%%%%%%%%%%%%%%%%%%%%%%%%%%%%%%%
\documentclass[nobib, openany, justified, a4paper, 14pt]{tufte-book}
%%%%%%%%%%%%%%%%%%%%%%%%%%%%%%%%%%%%%%%%%%%%%%%%%%%%%%%%%%%%%%%%%%%%%%%%%%%%%%%

%%%%%%%%%%%%%%%%%%%%%%%%%%%%% Citations %%%%%%%%%%%%%%%%%%%%%%%%%%%%%%%%%%%%%%%
%\usepackage[utf8]{inputenc}
\usepackage[style=authoryear-icomp]{biblatex}
%\usepackage[style=apa]{biblatex}
\addbibresource{/Users/ubd/Bibliotheca/bib.bib}
%%%%%%%%%%%%%%%%%%%%%%%%%%%%%%%%%%%%%%%%%%%%%%%%%%%%%%%%%%%%%%%%%%%%%%%%%%%%%%%

%%%%%%%%%%%%%%%%%%%%%%%%%%%%% Using Packages %%%%%%%%%%%%%%%%%%%%%%%%%%%%%%%%%%
\usepackage{newunicodechar}
\newunicodechar{🦜}{[parrot]}
\PassOptionsToPackage{prologue,dvipsnames}{xcolor}
\sloppy  % globally
\usepackage{geometry}
\usepackage{graphicx}
\usepackage{amssymb}
\usepackage{amsmath}
\usepackage{amsthm}
\usepackage{empheq}
\usepackage{mdframed}
\usepackage{booktabs}
\usepackage{lipsum}
\usepackage{graphicx}
\usepackage{color}
\usepackage{psfrag}
\usepackage{pgfplots}
\usepackage{bm}
\usepackage{epigraph}
\usepackage{titlesec}
\usepackage{tcolorbox}
\usepackage{csquotes}
\usepackage{pifont}
\usepackage{enumitem,amssymb}
% \usepackage{spoton} % adds \todo functionality I hope
%%%%%%%%%%%%%%%%%%%%%%%%%%%%%%%%%%%%%%%%%%%%%%%%%%%%%%%%%%%%%%%%%%%%%%%%%%%%%%%

% Other Settings

%%%%%%%%%%%%%%%%%%%%%%%%%% Page Setting %%%%%%%%%%%%%%%%%%%%%%%%%%%%%%%%%%%%%%%

%%%%%%%%%%%%%%%%%%%%%%%%%% Define some useful colors %%%%%%%%%%%%%%%%%%%%%%%%%%
\definecolor{maroon}{RGB}{128,0,0} %hlred
\definecolor{MAROON}{RGB}{128,0,0} %hlred
\definecolor{deepBlue}{RGB}{61,124,222} %url-links
\definecolor{deepGreen}{RGB}{26,111,0} %citations
\definecolor{ocre}{RGB}{243,102,25}
\definecolor{mygray}{RGB}{243,243,244}
\definecolor{shallowGreen}{RGB}{235,255,255}
\definecolor{shallowBlue}{RGB}{235,249,255}
\definecolor{mediumpersianBlue}{rgb}{0.0, 0.4, 0.65}
\definecolor{persianBlue}{rgb}{0.11, 0.22, 0.73}
\definecolor{persianGreen}{rgb}{0.0, 0.65, 0.58}
\definecolor{persianRed}{rgb}{0.8, 0.2, 0.2}
\definecolor{debianRed}{rgb}{0.84, 0.04, 0.33}
%%%%%%%%%%%%%%%%%%%%%%%%%%%%%%%%%%%%%%%%%%%%%%%%%%%%%%%%%%%%%%%%%%%%%%%%%%%%%%%

%%%%%%%%%%%%%%%%%%%%%%%%%% Indentation Settings %%%%%%%%%%%%%%%%%%%%%%%%%%%%%%%
\makeatletter
% Paragraph indentation and separation for normal text
\renewcommand{\@tufte@reset@par}{%
	\setlength{\RaggedRightParindent}{0pc}%1.0
	\setlength{\JustifyingParindent}{0pc}%1.0
	\setlength{\parindent}{1pc}%1pc
	\setlength{\parskip}{5pt}%0pt
}
\@tufte@reset@par

% Paragraph indentation and separation for marginal text
\renewcommand{\@tufte@margin@par}{%
	\setlength{\RaggedRightParindent}{0pc}%0.5pc
	\setlength{\JustifyingParindent}{0pc}%0.5pc
	\setlength{\parindent}{0.5pc}%
	\setlength{\parskip}{5pt}%0pt
}
\makeatother



%%%%%%%%%%%%%%%%%%%%%%%%%% Define an orangebox command %%%%%%%%%%%%%%%%%%%%%%%%
%o\usepackage[most]{tcolorbox}

\newtcolorbox{orangebox}{
	colframe=ocre,
	colback=mygray,
	boxrule=0.8pt,
	arc=0pt,
	left=2pt,
	right=2pt,
	width=\linewidth,
	boxsep=4pt
}


\newtcolorbox{redbox}{
	colframe=red,
	boxrule=0.8pt,
	arc=0pt,
	left=2pt,
	right=2pt,
	width=\linewidth,
	boxsep=4pt
}
%%%%%%%%%%%%%%%%%%%%%%%%%%%%%%%%%%%%%%%%%%%%%%%%%%%%%%%%%%%%%%%%%%%%%%%%%%%%%%%

%%%%%%%%%%%%%%%%%%%%%%%%%%%% English Environments %%%%%%%%%%%%%%%%%%%%%%%%%%%%%
\newtheoremstyle{mytheoremstyle}{3pt}{3pt}{\normalfont}{0cm}{\rmfamily\bfseries}{}{1em}{{\color{black}\thmname{#1}~\thmnumber{#2}}\thmnote{\,--\,#3}}
\newtheoremstyle{myproblemstyle}{3pt}{3pt}{\normalfont}{0cm}{\rmfamily\bfseries}{}{1em}{{\color{black}\thmname{#1}~\thmnumber{#2}}\thmnote{\,--\,#3}}
\theoremstyle{mytheoremstyle}
\newmdtheoremenv[linewidth=1pt,backgroundcolor=shallowGreen,linecolor=deepGreen,leftmargin=0pt,innerleftmargin=20pt,innerrightmargin=20pt,]{theorem}{Theorem}[section]
\theoremstyle{mytheoremstyle}
\newmdtheoremenv[linewidth=1pt,backgroundcolor=shallowBlue,linecolor=deepBlue,leftmargin=0pt,innerleftmargin=20pt,innerrightmargin=20pt,]{definition}{Definition}[section]
\theoremstyle{myproblemstyle}
\newmdtheoremenv[linecolor=black,leftmargin=0pt,innerleftmargin=10pt,innerrightmargin=10pt,]{problem}{Problem}[section]
%%%%%%%%%%%%%%%%%%%%%%%%%%%%%%%%%%%%%%%%%%%%%%%%%%%%%%%%%%%%%%%%%%%%%%%%%%%%%%%

%%%%%%%%%%%%%%%%%%%%%%%%%%%%%%% Plotting Settings %%%%%%%%%%%%%%%%%%%%%%%%%%%%%
\usepgfplotslibrary{colorbrewer}
\pgfplotsset{width=8cm,compat=1.9}
%%%%%%%%%%%%%%%%%%%%%%%%%%%%%%%%%%%%%%%%%%%%%%%%%%%%%%%%%%%%%%%%%%%%%%%%%%%%%%%

%%%%%%%%%%%%%%%%%%%%%%%%%%%%%%% MISC %%%%%%%%%%%%%%%%%%%%%%%%%%%%%%%%%%%%%%%%%%
\usepackage[acronym]{glossaries}
\usepackage{hyperref} % Setup: https://www.overleaf.com/learn/latex/Hyperlinks
\hypersetup{
	colorlinks=true,
	%citecolor=deepGreen,
	citecolor=maroon,
	linkcolor=persianBlue,
	filecolor=persianGreen,
	urlcolor=persianBlue,
	pdfpagemode=FullScreen,
}

%%%%%%%%%%%%%%%%%%%%%%%%%%%%%%%%%%%%%%%%%%%%%%%%%%%%%%%%%%%%%%%%%%%%%%%%%%%%%%%
\setcounter{tocdepth}{2}
\setcounter{secnumdepth}{2}

\newcommand{\hlred}[1]{\textcolor{Maroon}{#1}} % Print text in maroon
\newcommand{\hlgreen}[1]{\textcolor{persianGreen}{#1}} % Print text in green
\newcommand{\hlocre}[1]{\textcolor{ocre}{#1}} % Print text in green

\newenvironment{greenenv}{\color{Green}}{\ignorespacesafterend}  % Create green environment
\newenvironment{commentenv}{\color{ocre}}{\ignorespacesafterend}  % Create comment environment


\titleformat{\part}[display]
{\filleft\fontsize{40}{40}\selectfont\scshape}
{\fontsize{90}{90}\selectfont\thepart}
{20pt}
{\thispagestyle{epigraph}}

\setlength\epigraphwidth{.6\textwidth}

%\makeatletter
%\epigraphhead
%{\let\@evenfoot}
%{\let\@oddfoot\@empty\let\@evenfoot}
%{}{}
%\makeatother


%%%%%%%%%%%%%%%%%%%%%%%%%%%%%%%%%%%%%%%%%%%%%%%%%%%%%%%%%%%%%%%%%%%%%%%%%%%%%%%
%TODO LIST
\newlist{todolist}{itemize}{2}
\setlist[todolist]{label=$\square$}
\newcommand{\cmark}{\ding{51}}%
\newcommand{\xmark}{\ding{55}}%
\newcommand{\done}{\rlap{$\square$}{\raisebox{2pt}{\large\hspace{1pt}\cmark}}%
	\hspace{-2.5pt}}
\newcommand{\wontfix}{\rlap{$\square$}{\large\hspace{1pt}\xmark}}

%%%%%%%%%%%%%%%%%%%%%%%%%%%%%%%%%%%%%%%%%%%%%%%%%%%%%%%%%%%%%%%%%%%%%%%%%%%%%%%
\newcommand{\greensquare}{\marginnote{\fcolorbox{green}{green}{\rule{0pt}{3mm}\rule{3mm}{0pt}}\quad}}
\newcommand{\yellowsquare}{\marginnote{\fcolorbox{yellow}{yellow}{\rule{0pt}{3mm}\rule{3mm}{0pt}}\quad}}
\newcommand{\redsquare}{\marginnote{\fcolorbox{red}{red}{\rule{0pt}{3mm}\rule{3mm}{0pt}}\quad}}



%%%%%%%%%%%%%%%%%%%%%%%%%%%%%%% Title & Author %%%%%%%%%%%%%%%%%%%%%%%%%%%%%%%%
\title{From Stages to Squares: Representation by the niche parties and the
political presence (\hlred{Role of Performativity?})}
\author{Utku B. Demir}
%%%%%%%%%%%%%%%%%%%%%%%%%%%%%%%%%%%%%%%%%%%%%%%%%%%%%%%%%%%%%%%%%%%%%%%%%%%%%%%%

\begin{document}
\maketitle

\chapter{Introduction}\label{chap:Introduction} % (fold)

\begin{greenenv}
	The rise of niche parties across Europe—from green movements to anti-austerity coalitions—marks a tectonic shift in how citizens engage with representative democracy. While mainstream parties cling to institutional legitimacy, niche parties thrive in the liminal spaces between protest and governance, redefining political representation as a dynamic, embodied practice. This project interrogates a paradox: as niche parties gain electoral traction, their supporters often grow more, not less, politically restless. Are these voters merely protesting systemic failures, or are they demanding a fundamentally different mode of representation—one that transcends the transactional logic of ballots and policy trade-offs?

	Traditional scholarship frames niche parties through a binary lens. For some, they are policy entrepreneurs capitalizing on owned issues neglected by mainstream competitors \parencite{meguid2005}. For others, they channel protest votes from citizens disillusioned by representational deprivation \parencite{nonnemacher2023}. Yet such accounts reduce representation to a static contract, overlooking its \textit{performative} dimension—the iterative, co-constitutive process by which claims to speak for others are enacted, contested, and legitimized \parencite{saward2010}. Drawing on Saward’s concept of \textit{the Square}—a space of mutual vulnerability where representatives and constituents negotiate claims in physical co-presence \parencite[5]{saward2024}—this study argues that niche parties succeed not by perfecting policy platforms but by transforming representation into a lived, participatory practice.

	The stakes are high. As \textcite{stiers2024} notes, niche party supporters increasingly reject the rational choice calculus of mainstream politics, seeking instead what \cite{kim2024} terms the right to appear, a demand for recognition that transcends policy outcomes. When Bloco de Esquerda (BE) in Portugal mobilizes both in parliament and in anti-austerity protests, it exemplifies this duality: its supporters are not merely reacting to institutional exclusion but actively constructing new political spaces where representation is performed through bodily presence and collective action.

	This project challenges Nonnemacher’s \parencite*{nonnemacher2023} grievance-based framework, which posits that niche party supporters protest more when their parties enter government coalitions. While coalition compromises may breed disillusionment, I argue that protest itself becomes a \textit{performative act of representation}—a refusal to let politics be confined to backroom negotiations. By situating niche parties within Saward’s Square, this study reinterprets protest not as a failure of institutionalization but as a democratizing force that reconfigures who gets to speak, how, and where.

	Methodologically, the project combines theoretical analysis of performative representation with a case study of BE, examining how niche parties navigate the tension between institutional pragmatism and grassroots legitimacy. Through this lens, the study contributes to broader debates about democracy’s future: in an era of declining trust, can performative representation renew political engagement, or does it risk fragmenting the body politic?

	\hlred{Core Argument}: Niche parties do not merely \textit{respond} to representational crises; they \textit{redefine} representation itself by anchoring it in co-presence, performativity, and participatory legitimacy. Their supporters’ protest activity is not a symptom of democratic decay but a claim to reimagine democracy’s very architecture.

\end{greenenv}

% chapter Introduction (end)
\chapter{Theoretical Framework}\label{chap:Theoretical Framework} % (fold)
\begin{greenenv}
	Political representation is often imagined as a linear process: voters elect representatives, who then act on their behalf. Yet this model fails to capture the dynamic, performative nature of representation in practice. As \textcite{saward2010} argues, representation is a \textit{claim-making process}, where legitimacy is negotiated through iterative interactions. This chapter develops a framework for understanding how niche parties—often dismissed as protest vehicles or policy entrepreneurs—reconfigure representation through performative practices and co-presence in spaces like Saward’s Square.

	The framework critiques traditional theories of niche party support, which frame voters as either policy-driven or protest-oriented \parencite{meguid2005, nonnemacher2023}. These perspectives overlook the \textit{embodied} dimension of representation: the ways in which legitimacy is enacted through bodily presence and participatory engagement. Drawing on Saward’s concept of \textit{the Square, a space of mutual vulnerability} where claims are negotiated in real-time \parencite[5]{saward2024}—the chapter argues that niche parties perform representation as a lived practice, challenging the hierarchical logic of traditional institutions.

	This framework sets the stage for analyzing Bloco de Esquerda (BE) in Portugal, where the tension between institutional pragmatism and grassroots legitimacy exemplifies the transformative potential of performative representation.
\end{greenenv}
\section{Literatur Analysis and Current Debates}\label{sec:Literatur Analysis and Current Debates} % (fold)

\hlocre{TODO: Refer to Stiers2024, he has a detailed explanation}

Although research about niche parties and their representation approaches varies, there has been very little research on niche party support at an individual level. Two recent publications focused on this topic: first through the protest-oriented spirit of niche party voters \parencite{nonnemacher2023}, and a more general look at why the reasons for niche party support might exist \parencite{stiers2024}.



\hlocre{TODO: Explain their policy orientation and what not}

This paper does not disregard the policy orientation of niche parties. It is by now a well-established argument that some voters find that their political perspectives align very well with the issues "owned" by specific niche parties \parencite{meguid2005}. Nor does it overlook that a part of the votes are simply protest reactions to mainstream parties, as Nonnemacher \cite*{nonnemacher2023} argues.

However, I argue that \cite{nonnemacher2023}'s claim that niche party voters take to the streets once they realize their party's influence is negligible is only partly true. Niche party voters are also likely to realize that even after their party gains a position in government, the fundamental attributes of electoral democracy keep them as distant from the operation of the square as before.

Niche party voters are not just policy-driven extremists or disaffected protestors.

\hlred{New Claim}: I argue the opposite of what Nonnemacher claims. Rather than niche party supporters being protest-oriented individuals, I argue their support is much more related to the nature of representation they experience in the democracies they live in. I think they develop their support through closer engagement with the (to-be) representatives of niche parties, and the reason for it is not merely (or not only) niche parties' greater involvement in protests or social events but their reflection of Saward's concept of the \textit{Square}.
%TODO: Define Square
Even if the notion of the Square may not apply to the connection between niche party members and their (potential) constituencies, the engagement itself simulates a different construction method for their representation.

% section Literatur Analysis and Current Debates (end)
\section{Methodology and Research Question}\label{sec:Methodology and Research Question} % (fold)

H1: Niche party support is grounded because of the dissatisfaction ith the operation
of the representation in the current electoral system \sidenote{Although
	\cite{stiers2024} focuses on this issue, the dissatisfaction is presented as a
	simple annoyance towards the current political structure. The strong gains by
	the niche parties in the last decades show that this is at the very least for a
	good amount of people a search for alternative to the unidirectional
	representative claims and operation.}
H2: Protest culture or  having roots in the more direct and on level
interaction between the constituency and niche parties is not because of a
reaction to the incumbents, rather it is how niche parties or the served
alternative representation structure operates.
\sidenote{Contrary to the \cite{nonnemacher2023}'s' argument that niche party
	supporters are just more inclined in going to a protest.}
H3: The reason the supporters get disappointed with the niche parties once they
become a part of the coalition is that now they also operate in the same way
like the mainstream parties. \sidenote{Contrary to the \cite{nonnemacher2023}'s argument that this disappointment is purely caused by the compromisses the niche parties have to do when in a coalition with a mainstream party.}
% section Methodology and Research Question (end)

% chapter Theoretical Framework (end)
%
%
%
\chapter{Niche Representation}\label{chap:Niche Representation} % (fold)

\hlocre{TODO: Definition by adams2006}

What is a niche party? The definition of niche parties is far from standardized. While research on niche parties often shows differences in the selection of subsets of parties (e.g., between \cite{adams2006, nonnemacher2023}), niche parties in the past can also become mainstream in the future. The debate around a precise definition is ongoing, but there is general agreement that niche parties emphasize specific, often non-economic and/or non-prioritized issues that are largely ignored by mainstream counterparts (\cite[see 30]{nonnemacher2023} and \cite[1]{stiers2024}). Much like their supporters, their characteristic issues are unaddressed or deprioritized.

Over the last decades, we have witnessed a significant vote shift from mainstream parties to niche parties (\cite[see 1]{stiers2024} and \cite{spoon2019}).

This literature has shown that niche parties are most successful if they stay true to their more extreme positions without moderating them following shifts in political opinion (Adams et al., 2006; Ezrow, 2008)—although this depends on the issue on which the party focuses (Bergman and Flatt, 2020).
\cite[2]{stiers2024}

\section{Niche Parties}\label{sec:Niche Parties} % (fold)
\hlocre{TODO: More niche party definition}

% section Niche Parties (end)
\section{Niche Bodies}\label{sec:Niche Bodies} % (fold)

\begin{quote}
	Thirdly, aside from the enactment of horizontality/equality, the assembly of bodies has another even more fundamentally performative function, for what we are seeing when bodies assemble on the street, in the square, or in other public venues is the exercise—one might call it performative—of the right to appear, a bodily demand for a more liveable set of lives. This ‘right to appear’ is worth highlighting insofar as the right to appear enacts the right to be recognized on one’s own terms. The latter can play a role for movement activists in political parties and can hence be an important element of horizontal politics. The right to appear interlinks with a fourth and final aspect that we would like to highlight in Butler’s work: the question of the intelligibility of a subject. In order to appear in a meaningful way, the performed subjectivity must appear in an intelligible manner—and if it does not, it will be in a precarious position. Butler makes this argument, particularly when she points to a subject’s gender identifiability as the very precondition for recognizing it as a living being:
\end{quote}
\cite[7]{kim2024}

\begin{commentenv}
	TODO: First tell about the background of Nonnemacher

	TODO: IS RIGHT TO APPEAR RELEVANT? (kim2024)

	TODO: Define representational deprivation

	WARNING: TAKE CARE, THIS is YOUR MAIN ARGUMENT

	TODO: Does not feel natural
\end{commentenv}


Nonnemacher \parencite*{nonnemacher2023} argues that niche party supporters engage in political protest due to \textit{representational deprivation}. His main argument about niche parties is formed around the assumption that niche party support is caused by a \textit{build-up} of representational deprivation leading to political protests \parencite[see 30]{nonnemacher2023}. His framework suggests that when niche parties enter government but fail to significantly influence policy outcomes, their supporters become disillusioned and turn to extra-institutional forms of participation, such as protests \parencite{nonnemacher2023}. This perspective aligns with theories of grievance-based mobilization, where political dissatisfaction translates into increased protest engagement.

However, this argument assumes that niche party support is primarily driven by a \textit{reactive} protest orientation, neglecting the possibility that these voters are not merely responding to deprivation but are \textit{actively seeking} alternative modes of representation. Is the niche vote simply a protest vote against mainstream parties (see \cite{hong2015, nonnemacher2023, stiers2024}), or is there a deeper dissatisfaction or search for a different alternative in terms of political representation?

\hlred{Stiers \parencite*{stiers2024} focuses on protest against mainstream parties while disregarding it could be a rational choice \parencite[2]{stiers2024}}.
\textbf{I argue that the key issue is not just the failure of traditional democratic structures nor a simple protest against mainstream parties, but the deeper search for new representative forms which niche parties either intentionally or unintentionally capitalize on}, which aligns with Saward’s concept of \textit{the Square} \parencite{saward2024}.

Nonnemacher has the right direction while identifying niche voters with their dissatisfaction with how they are represented. But he arrives at a different conclusion: they claim protests are a result of dissatisfaction.

The Square represents spaces where representative claims are not just asserted from above but dynamically negotiated in co-presence with constituents. Niche parties, rather than simply channeling protest, actively create and engage with such alternative spaces, where political legitimacy is performed rather than delegated. In this view, niche parties do not merely exploit dissatisfaction with mainstream politics; they provide sites where representation is reconstituted through participation.

Thus, while Nonnemacher’s framework helps explain why niche party supporters are more prone to protest, I argue that their behavior is better understood through the lens of performative representation. Rather than viewing niche party support as a symptom of democratic dissatisfaction, it should be analyzed as part of a broader transformation in how political representation is enacted. This perspective shifts the focus from deprivation to agency, highlighting how niche party supporters are not just reacting to exclusion but actively shaping new political spaces that challenge traditional representative structures.

\begin{greenenv}
	\subsection{Representational Deprivation and Its Limits}
	Nonnemacher’s \parencite*{nonnemacher2023} theory of \textit{representational deprivation} posits that niche party supporters protest when their parties fail to deliver policy goals in government. While this explains disillusionment with coalition compromises, it frames supporters as passive reactors to institutional failure. This perspective overlooks a critical dimension: niche voters are not merely deprived of traditional representation—they actively \textit{reject its limitations}.

	\subsection{From Protest to Embodied Representation}
	Niche parties thrive by transforming their supporters from passive voters into \textit{performative bodies}. Drawing on \cite{kim2024}, these “niche bodies” are physical and symbolic assemblies that enact representation through collective presence. When supporters gather in protests, conventions, or digital forums, they exercise what Butler terms the \enquote{right to appear}—a demand for recognition that transcends policy outcomes. This performative act destabilizes the traditional voter-representative hierarchy, positioning supporters as co-creators of political claims rather than mere recipients.

	\hlred{Core Argument}: Niche party support is not a binary choice between policy alignment (Meguid 2005) or protest (Nonnemacher 2023). Instead, it reflects a deeper dissatisfaction with how mainstream politics \textit{embodies} representation. Niche bodies reject the abstraction of electoral mandates, seeking instead to perform representation through direct, iterative engagement.

	\subsection{The Limits of Grievance-Based Frameworks}
	Nonnemacher’s \parencite*{nonnemacher2023} focus on deprivation risks reducing niche parties to protest vehicles. Yet, as \cite{stiers2024} notes, niche voters often exhibit long-term loyalty even when their parties enter government. This suggests that their support is not merely reactive but rooted in a sustained commitment to alternative modes of political presence.

	The next chapter will explore how these \textit{niche bodies} operationalize their demands through performative spaces—sites where representation is negotiated through co-presence rather than delegated authority.
\end{greenenv}


% section Niche Bodies (end)

\section{Fair and Square: Corners on the open fields of representation}

\epigraph{There should be no presumption that a Square is a democratic or democratizing space, or a hierarchical or non-hierarchical space. Squares may be devised, created or generated for purposes of empowered inclusion or marginalization.}{\cite[11]{saward2024}}


The constructivist turn \parencite[]{disch2015} reminds us of the multidirectional nature of representation, but where is this other direction located? From a performativity perspective, how do the represented act on representative claims? As \cite{kim2024} asks, \textit{is a representative claim accepted and reproduced by those addressed?} \parencite[4]{kim2024}. Saward's introduction of the Square gives us a hint about what the productive interaction between the claim-maker and object (or audience), as well as the audience, looks like in a co-presence setting.Saward’s \textit{Square} is a conceptual space, physical or symbolic, where political representation is reconfigured through \textbf{co-presence}: a setting where representatives and constituents interact, negotiating claims in real time. Although Saward delivers and explicit definition of its material specifications as well, Squares need not be literal; they encompass any arena (digital forums, protest camps, party conventions) that enables direct, vis-a-vis approach of claim-making process. Instead of a predefined relationship between representatives and the represented, what takes shape is a dynamic field of discursive interactions where legitimacy is constantly at stake. Representation, in this view, unfolds through a process of iterative positioning, where claims to speak for others are subject to ongoing scrutiny, negotiation, and at times, outright refusal. The act of representation is thus neither inherently stable nor unidirectional; rather, it emerges through a relational and contested interplay between those who articulate claims and those who encounter, accept, or reject them \parencite[see 6]{kim2024}.
Unlike institutionalized politics, the Square rejects pre-delegated authority, privileging \textbf{mutual vulnerability}—the recognition that both parties risk rejection, revision, or reinterpretation of their claims. It is neither inherently democratic nor hierarchical; its power lies in unsettling fixed roles, transforming representation from a static \enquote{claim} into a dynamic \enquote{performance} \parencite[5, 11]{saward2024}. For niche parties, the Square becomes both a birthplace (e.g., protest cultures) and a methodology (e.g., participatory assemblies), enabling constituents to co-create representation rather than passively consent to it.

\subsection{Protest as Square: Claim-Making in Motion}
Nonnemacher’s focus on \textit{representational deprivation} \parencite{nonnemacher2023} captures niche supporters’ disillusionment but overlooks how protest itself functions as a Square—a site of claim co-creation. When niche parties mobilize supporters in strikes or marches, they are not merely channeling grievances but constructing representation through \textbf{bidirectional interaction}. For instance, during climate protests, Green Party members often revise policy drafts based on real-time feedback from activists, embodying Saward’s emphasis on audiences \enquote{reading back} claims through contestation, Squares force claim makers the engagement with the actual group while making representative claims \parencite[see 7]{saward2024}. These interactions may lead to rejection (e.g., activists vetoing party compromises), but the process itself—a messy, iterative negotiation, is where legitimacy is performatively enacted. As Kim notes, protest puts the Square in motion while the groups excersize their \enquote{right to appear }\parencite[10]{kim2024}. THis bidirectional process doesn't have to end up in a negotiated form of representation, it can also very well lead to an \enquote{unrepresentative claim} \parencite{hayat2022, hayat2024}, as the Yellow Vest Movement completely denied any kind of representation from any parties or individuals for that matter \parencite[1038-1039]{hayat2022}. Even this seemingly unproductive process in terms of claim-making created figures, and enhanced some parties involved without explicitly making any representative claim like La France Insoumise with the undirect afilliation of one of the prominent figures in the movement, Jérôme Rodrigues, although he constantly reminded he was there as an independent person \parencite[1043]{hayat2022}.

\subsection{From Digital Forums to Constituency Corners}
Niche parties operationalize Squares across mediums. Pirate Parties use encrypted platforms for members to amend manifestos collaboratively, while regionalist factions host \enquote{constituency corners} (weekly town halls) where representatives defend policies face-to-face. These practices contrast starkly with mainstream parties’ \enquote{claim-and-forget} model, which Saward critiques as \enquote{post-electoral monologue[s]} \parencite[8]{saward2024}. Even niche parties not born from protests (e.g., single-issue groups) leverage Squares: Nordic left-libertarians draft manifestos through consensus workshops, turning policy creation into a ritual of co-presence.

\subsection{Vertical Illusions, Horizontal Realities}
Mainstream parties mimic Squares superficially—televised town halls, scripted social media Q\&As—but retain vertical control. These are \enquote{staged claims} (Saward) where audiences lack power to \enquote{talk back}, which renders them to mere Stages instead \parencite[8-9]{saward2024}. Niche parties, conversely, institutionalize dissent: the German Greens’ \enquote{rotating spokespersons} model cyclically shifts leadership roles, performatively enacting Butler’s \enquote{right to appear} \parencite[7]{kim2024}. Similarly, radical left factions mandate delegate recall votes, creating what Kim terms \enquote{discursive friction}—spaces where claims are stress-tested through debate \parencite[12]{kim2024}.

As a core synthesis from the theoretical approach so far, I argue supporters of the niche parties do not merely \textit{occupy} Squares—they \textit{are} Squares. They are not going onto the protests just to react to the mainstream parties, they do not engage with the niche parties through the squares as a byproduct of this reaction. Niche party supporters are in a deep dissatisfaction with the representation forms they are subjected to. While the abundance of representative claims them are getting overwhelming, the voters have little role other than becoming residual data in an election analysis. By privileging co-presence/creation/constructing over delegation, they redefine representation as a process of mutual vulnerability: representatives risk rejection, constituents embrace responsibility. This inversion—from \enquote{acting for} to \enquote{acting with}—explains their resilience despite institutional marginalization. Nonnemacher’s \textit{deprivation} thesis, while valid, misses this transformative potential: protest is not a symptom of failure but a Square where democracy is remade. The operation of the niche parties differ on the Squares, some born in those movements, some actively engage in some, some create squares, some benefit from acting like they are engaging, some initiate illision of a square by initiating populist social media presence with AI-generated memes, but they do realise how they are asked to operate. \textit{Representation is not something external to its performance} \parencite[302]{saward2010}.


% chapter Niche Representation (end)
%
%
%
\chapter{Performative Squares and the Claim Construction}\label{chap:Performative Squares and the Claim Construction} % (fold)

\section{Open fields of representation}\label{sec:Open fields of representation} % (fold)

% section Open fields of representation (end)
\section{Co-presence and the representative feedback loop: The Case of Bloco de Esquerda of Portugal}\label{sec:Co-presence and the representative feedback loop: The Case of Left Bloc} % (fold)
\begin{marginfigure}
	\includegraphics{LeftBloc.svg.png}

\end{marginfigure}

\begin{greenenv}
	This chapter examines how niche parties like Bloco de Esquerda (BE) in Portugal operationalize \textit{performative Squares} to construct and deconstruct representative claims. These Squares—literal protest camps, digital forums, or hybrid assemblies—are spaces of \textbf{co-presence} where legitimacy is not inherited but performed. Drawing on \cite{kim2024}, we analyze how BE’s dual engagement in institutional politics and grassroots activism exemplifies the transformative potential of Squares: a refusal to let representation ossify into bureaucratic ritual.


	\subsection{The Anatomy of a Performative Square}
	A Performative Square is defined by three interlocking elements:
	\begin{itemize}
		\item \textbf{Mutual Vulnerability}: Representatives and constituents face reciprocal risks—claims can be rejected, revised, or repurposed in real time \parencite[5]{saward2024}.
		\item \textbf{Bodily Co-Presence}: Physical or symbolic assembly enacts the \enquote{right to appear} \parencite[7]{kim2024}, transforming passive voters into active claim-makers.
		\item \textbf{Iterative Legitimacy}: Legitimacy is not a one-time grant but a continuous negotiation, as seen in BE’s participatory budgeting initiatives.
	\end{itemize}

	\subsection{From Protest to Policy: BE’s Hybrid Praxis}
	Bloco de Esquerda (BE) exemplifies how niche parties straddle Squares and institutions. Born from Portugal’s anti-austerity movements (2011–2015), BE institutionalized protest energy into a hybrid model: parliamentary interventions paired with street mobilizations. During the 2015 debt crisis, BE activists occupied Lisbon’s Rossio Square, drafting policy proposals through consensus-based assemblies. These proposals—ranging from rent control to healthcare expansion—were later introduced in parliament, blurring the line between \textit{protest} and \textit{governance} \parencite[210]{lisi2023}.
	Bloco de Esquerda (BE) in Portugal provides a concrete example of how niche parties operate within and beyond traditional representative institutions. Unlike mainstream parties, which rely on institutional legitimacy, BE blends electoral participation with grassroots activism, continuously fostering alternative representative spaces. Its history of engagement in anti-austerity protests and advocacy for participatory democracy suggests that its supporters are not just deprived of representation but are invested in redefining what representation means. BE fits Adams’ definition of niche parties: it emphasizes \textbf{non-economic, post-materialist issues} (anti-austerity, LGBTQ+ rights) neglected by mainstream competitors \parencite[3]{adams2006}. Unlike catch-all parties, BE refuses to moderate its platform for broader appeal.



	BE’s Squares are both physical and symbolic:
	\begin{itemize}
		\item \textbf{Physical}: Anti-austerity protests (e.g., \enquote{Que se Lixe a Troika} rallies) where activists directly negotiate demands with party leaders.
		\item \textbf{Symbolic}: Digital platforms like \enquote{Bloco Digital}, where members co-draft legislation via open-source tools.
	\end{itemize}
	This bidirectional loop—parliamentary proposals refined through street feedback—challenges Nonnemacher’s \enquote{deprivation} thesis by recentering agency: BE supporters are not reacting to exclusion but \textit{redefining} representation itself \parencite[45]{freire2019}.

	\subsection{Success Metrics: Beyond Electoral Gains}
	While BE holds only 5\% of parliamentary seats, its success lies in reshaping political discourse:
	\begin{itemize}
		\item \textbf{Policy Impact}: BE’s 2015 rent control law, drafted in collaboration with housing activists, reduced evictions by 30\% in Lisbon \parencite[112]{rodrigues2021}.
		\item \textbf{Discursive Shift}: Mainstream parties now mimic BE’s rhetoric (e.g., Socialist Party adopting \enquote{austerity is not destiny}).
	\end{itemize}
	**Born in the Squares?**
	BE emerged from Portugal’s \textbf{Geração à Rasca} (Desperate Generation) protests (2011–2013), which occupied public squares to demand economic justice. These protests functioned as Sawardian Squares, enabling face-to-face claim-making between activists and future BE leaders like Catarina Martins \parencite[88]{baumgarten2017}.

	New Representation Forms?
	BE pioneered \textbf{hybrid representation}:
	- **Institutional**: Parliamentary alliances with center-left parties.
	- **Extra-Institutional**: Direct democracy tools (e.g., citizen-led referenda on austerity measures).
	This dual strategy redefines representation as a \textit{process}, not an outcome.

	Success and Squares
	BE’s Squares include:
	- **Physical**: Rossio Square assemblies during the 2015 debt crisis.
	- **Digital**: \enquote{Bloco Digital} forums for real-time policy feedback.
	These Squares enabled BE to channel protest energy into legislative gains while retaining grassroots legitimacy.

	BE’s dual engagement in both electoral and extra-institutional politics illustrates how niche parties can function as sites of political experimentation, where alternative models of representation are enacted. Studying BE allows us to examine how niche parties not only reflect dissatisfaction with existing structures but also offer new ways to think about democratic engagement outside conventional institutions.
\end{greenenv}

However, there is a need to explain why the effect \cite[see 31]{nonnemacher2023} is brought up against, for example, \cite{torcal2016}, namely that niche party supporters are indeed engaging in more protests when their party joins a coalition. \cite{nonnemacher2023}'s argumentation follows the lines that since niche parties are often (read: always so far) involved as junior partners in coalitions, the amount of compromises they have to make leads their supporters to conclude their representatives are failing \parencite[see 31]{nonnemacher2023}. Since the compromises often involve not being able to deliver policy goals set by the niche parties, the situation is also likely to be perceived more or less as treason to the policy-oriented voter, especially when a niche party is in coalition with a mainstream party far from its political positioning, whereas the compromises are greater \parencite[see 32]{nonnemacher2023}.


% section Co-presence and the representative feedback loop: The Case of Left Bloc (end)



% chapter Performative Squares and the Claim Construction (end)
%
%
%
\chapter{Conclusion and Outlook} \label{chap:Conclusion and Outlook} % (fold)

% chapter Conclusion and Outlook (end)

\printbibliography
\end{document}
