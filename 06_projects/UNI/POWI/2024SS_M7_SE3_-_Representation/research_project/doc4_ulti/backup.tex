%%%%%%%%%%%%%%%%%%%%%%%%%%%% Define Article %%%%%%%%%%%%%%%%%%%%%%%%%%%%%%%%%%
\documentclass[nobib, openany, justified, a4paper, 14pt]{tufte-book}
%%%%%%%%%%%%%%%%%%%%%%%%%%%%%%%%%%%%%%%%%%%%%%%%%%%%%%%%%%%%%%%%%%%%%%%%%%%%%%%

%%%%%%%%%%%%%%%%%%%%%%%%%%%%% Citations %%%%%%%%%%%%%%%%%%%%%%%%%%%%%%%%%%%%%%%
%\usepackage[utf8]{inputenc}
\usepackage[style=authoryear-icomp]{biblatex}
%\usepackage[style=apa]{biblatex}
\addbibresource{/Users/ubd/Bibliotheca/bib.bib}
%%%%%%%%%%%%%%%%%%%%%%%%%%%%%%%%%%%%%%%%%%%%%%%%%%%%%%%%%%%%%%%%%%%%%%%%%%%%%%%

%%%%%%%%%%%%%%%%%%%%%%%%%%%%% Using Packages %%%%%%%%%%%%%%%%%%%%%%%%%%%%%%%%%%
\usepackage{newunicodechar}
\newunicodechar{🦜}{[parrot]}
\PassOptionsToPackage{prologue,dvipsnames}{xcolor}
\sloppy  % globally
\usepackage{geometry}
\usepackage{graphicx}
\usepackage{amssymb}
\usepackage{amsmath}
\usepackage{amsthm}
\usepackage{empheq}
\usepackage{mdframed}
\usepackage{booktabs}
\usepackage{lipsum}
\usepackage{graphicx}
\usepackage{color}
\usepackage{psfrag}
\usepackage{pgfplots}
\usepackage{bm}
\usepackage{epigraph}
\usepackage{titlesec}
\usepackage{tcolorbox}
\usepackage{csquotes}
\usepackage{pifont}
\usepackage{enumitem,amssymb}
% \usepackage{spoton} % adds \todo functionality I hope
%%%%%%%%%%%%%%%%%%%%%%%%%%%%%%%%%%%%%%%%%%%%%%%%%%%%%%%%%%%%%%%%%%%%%%%%%%%%%%%

% Other Settings

%%%%%%%%%%%%%%%%%%%%%%%%%% Page Setting %%%%%%%%%%%%%%%%%%%%%%%%%%%%%%%%%%%%%%%

%%%%%%%%%%%%%%%%%%%%%%%%%% Define some useful colors %%%%%%%%%%%%%%%%%%%%%%%%%%
\definecolor{maroon}{RGB}{128,0,0} %hlred
\definecolor{MAROON}{RGB}{128,0,0} %hlred
\definecolor{deepBlue}{RGB}{61,124,222} %url-links
\definecolor{deepGreen}{RGB}{26,111,0} %citations
\definecolor{ocre}{RGB}{243,102,25}
\definecolor{mygray}{RGB}{243,243,244}
\definecolor{shallowGreen}{RGB}{235,255,255}
\definecolor{shallowBlue}{RGB}{235,249,255}
\definecolor{mediumpersianBlue}{rgb}{0.0, 0.4, 0.65}
\definecolor{persianBlue}{rgb}{0.11, 0.22, 0.73}
\definecolor{persianGreen}{rgb}{0.0, 0.65, 0.58}
\definecolor{persianRed}{rgb}{0.8, 0.2, 0.2}
\definecolor{debianRed}{rgb}{0.84, 0.04, 0.33}
%%%%%%%%%%%%%%%%%%%%%%%%%%%%%%%%%%%%%%%%%%%%%%%%%%%%%%%%%%%%%%%%%%%%%%%%%%%%%%%

%%%%%%%%%%%%%%%%%%%%%%%%%% Indentation Settings %%%%%%%%%%%%%%%%%%%%%%%%%%%%%%%
\makeatletter
% Paragraph indentation and separation for normal text
\renewcommand{\@tufte@reset@par}{%
	\setlength{\RaggedRightParindent}{0pc}%1.0
	\setlength{\JustifyingParindent}{0pc}%1.0
	\setlength{\parindent}{1pc}%1pc
	\setlength{\parskip}{5pt}%0pt
}
\@tufte@reset@par

% Paragraph indentation and separation for marginal text
\renewcommand{\@tufte@margin@par}{%
	\setlength{\RaggedRightParindent}{0pc}%0.5pc
	\setlength{\JustifyingParindent}{0pc}%0.5pc
	\setlength{\parindent}{0.5pc}%
	\setlength{\parskip}{5pt}%0pt
}
\makeatother



%%%%%%%%%%%%%%%%%%%%%%%%%% Define an orangebox command %%%%%%%%%%%%%%%%%%%%%%%%
%o\usepackage[most]{tcolorbox}

\newtcolorbox{orangebox}{
	colframe=ocre,
	colback=mygray,
	boxrule=0.8pt,
	arc=0pt,
	left=2pt,
	right=2pt,
	width=\linewidth,
	boxsep=4pt
}


\newtcolorbox{redbox}{
	colframe=red,
	boxrule=0.8pt,
	arc=0pt,
	left=2pt,
	right=2pt,
	width=\linewidth,
	boxsep=4pt
}
%%%%%%%%%%%%%%%%%%%%%%%%%%%%%%%%%%%%%%%%%%%%%%%%%%%%%%%%%%%%%%%%%%%%%%%%%%%%%%%

%%%%%%%%%%%%%%%%%%%%%%%%%%%% English Environments %%%%%%%%%%%%%%%%%%%%%%%%%%%%%
\newtheoremstyle{mytheoremstyle}{3pt}{3pt}{\normalfont}{0cm}{\rmfamily\bfseries}{}{1em}{{\color{black}\thmname{#1}~\thmnumber{#2}}\thmnote{\,--\,#3}}
\newtheoremstyle{myproblemstyle}{3pt}{3pt}{\normalfont}{0cm}{\rmfamily\bfseries}{}{1em}{{\color{black}\thmname{#1}~\thmnumber{#2}}\thmnote{\,--\,#3}}
\theoremstyle{mytheoremstyle}
\newmdtheoremenv[linewidth=1pt,backgroundcolor=shallowGreen,linecolor=deepGreen,leftmargin=0pt,innerleftmargin=20pt,innerrightmargin=20pt,]{theorem}{Theorem}[section]
\theoremstyle{mytheoremstyle}
\newmdtheoremenv[linewidth=1pt,backgroundcolor=shallowBlue,linecolor=deepBlue,leftmargin=0pt,innerleftmargin=20pt,innerrightmargin=20pt,]{definition}{Definition}[section]
\theoremstyle{myproblemstyle}
\newmdtheoremenv[linecolor=black,leftmargin=0pt,innerleftmargin=10pt,innerrightmargin=10pt,]{problem}{Problem}[section]
%%%%%%%%%%%%%%%%%%%%%%%%%%%%%%%%%%%%%%%%%%%%%%%%%%%%%%%%%%%%%%%%%%%%%%%%%%%%%%%

%%%%%%%%%%%%%%%%%%%%%%%%%%%%%%% Plotting Settings %%%%%%%%%%%%%%%%%%%%%%%%%%%%%
\usepgfplotslibrary{colorbrewer}
\pgfplotsset{width=8cm,compat=1.9}
%%%%%%%%%%%%%%%%%%%%%%%%%%%%%%%%%%%%%%%%%%%%%%%%%%%%%%%%%%%%%%%%%%%%%%%%%%%%%%%

%%%%%%%%%%%%%%%%%%%%%%%%%%%%%%% MISC %%%%%%%%%%%%%%%%%%%%%%%%%%%%%%%%%%%%%%%%%%
\usepackage[acronym]{glossaries}
\usepackage{hyperref} % Setup: https://www.overleaf.com/learn/latex/Hyperlinks
\hypersetup{
	colorlinks=true,
	%citecolor=deepGreen,
	citecolor=maroon,
	linkcolor=persianBlue,
	filecolor=persianGreen,
	urlcolor=persianBlue,
	pdfpagemode=FullScreen,
}

%%%%%%%%%%%%%%%%%%%%%%%%%%%%%%%%%%%%%%%%%%%%%%%%%%%%%%%%%%%%%%%%%%%%%%%%%%%%%%%
\setcounter{tocdepth}{2}
\setcounter{secnumdepth}{2}

\newcommand{\hlred}[1]{\textcolor{Maroon}{#1}} % Print text in maroon
\newcommand{\hlgreen}[1]{\textcolor{persianGreen}{#1}} % Print text in green
\newcommand{\hlocre}[1]{\textcolor{ocre}{#1}} % Print text in green

\newenvironment{greenenv}{\color{Green}}{\ignorespacesafterend}  % Create green environment
\newenvironment{commentenv}{\color{ocre}}{\ignorespacesafterend}  % Create comment environment


\titleformat{\part}[display]
{\filleft\fontsize{40}{40}\selectfont\scshape}
{\fontsize{90}{90}\selectfont\thepart}
{20pt}
{\thispagestyle{epigraph}}

\setlength\epigraphwidth{.6\textwidth}

%\makeatletter
%\epigraphhead
%{\let\@evenfoot}
%{\let\@oddfoot\@empty\let\@evenfoot}
%{}{}
%\makeatother


%%%%%%%%%%%%%%%%%%%%%%%%%%%%%%%%%%%%%%%%%%%%%%%%%%%%%%%%%%%%%%%%%%%%%%%%%%%%%%%
%TODO LIST
\newlist{todolist}{itemize}{2}
\setlist[todolist]{label=$\square$}
\newcommand{\cmark}{\ding{51}}%
\newcommand{\xmark}{\ding{55}}%
\newcommand{\done}{\rlap{$\square$}{\raisebox{2pt}{\large\hspace{1pt}\cmark}}%
	\hspace{-2.5pt}}
\newcommand{\wontfix}{\rlap{$\square$}{\large\hspace{1pt}\xmark}}

%%%%%%%%%%%%%%%%%%%%%%%%%%%%%%%%%%%%%%%%%%%%%%%%%%%%%%%%%%%%%%%%%%%%%%%%%%%%%%%
\newcommand{\greensquare}{\marginnote{\fcolorbox{green}{green}{\rule{0pt}{3mm}\rule{3mm}{0pt}}\quad}}
\newcommand{\yellowsquare}{\marginnote{\fcolorbox{yellow}{yellow}{\rule{0pt}{3mm}\rule{3mm}{0pt}}\quad}}
\newcommand{\redsquare}{\marginnote{\fcolorbox{red}{red}{\rule{0pt}{3mm}\rule{3mm}{0pt}}\quad}}



\usepackage{syntonly}

%%%%%%%%%%%%%%%%%%%%%%%%%%%%%%% Title & Author %%%%%%%%%%%%%%%%%%%%%%%%%%%%%%%%
\title{From Stages to Squares: Representation by the niche parties and the
political presence (\hlred{Role of Performativity?})}
\author{Utku B. Demir}
%%%%%%%%%%%%%%%%%%%%%%%%%%%%%%%%%%%%%%%%%%%%%%%%%%%%%%%%%%%%%%%%%%%%%%%%%%%%%%%%
%\syntaxonly
\begin{document}
\maketitle

\chapter{Introduction}\label{chap:Introduction} % (fold)

The rise of niche parties across Europe, from green movements to anti-austerity coalitions, marks a tectonic shift in how citizens engage with representative democracy. While mainstream parties cling to institutional legitimacy, niche parties thrive in the liminal spaces between protest and governance, redefining political representation as a dynamic, embodied practice. This project interrogates analyses the increasing support for the niche parties from the perspective of their constituencies. Are these voters merely protesting systemic failures, expressing their dissatisfaction with the incumbents, sticking to their rigidly defined ideological stance or policy preferences, or are they demanding a fundamentally different mode of representation, one that transcends the transactional logic of ballots and policy trade-offs?

Traditional scholarship frames niche parties through a binary lens. For some, they are policy entrepreneurs capitalizing on owned issues neglected by mainstream competitors \parencite{meguid2005}. For others, they channel protest votes from citizens disillusioned by representational deprivation \parencite{nonnemacher2023}. Yet such accounts might be reducing representation to a static contract, overlooking its \textit{performative} dimension. Does the support for niche parties relate to the specifications of the process how the representative claims shaped \parencite{saward2010}? Drawing on Saward’s concept of \textit{the Square}, a space of mutual vulnerability where representatives and constituents negotiate claims in co-presence \parencite[5]{saward2024}, this study argues that niche parties succeed not by perfecting policy platforms but by transforming representation into a lived, participatory practice.

This study attempts to change the perspective grievance-based explanations of niche party support, particularly reflecting on Nonnemacher’s \parencite*{nonnemacher2023} framework, which interprets niche party supporters’ protest activity as a reaction to government participation. While coalition compromises may contribute to disillusionment, I argue that protest grounds themselves function as a \textit{performative act of representation}, a co-present way of structuring the representative claims with the interactions between the supporters and party formation.

Niche party supporters do not only \textit{respond} to the current political occasions; they are also in a search for different approaches for representation itself, expressing their deep dissatisfaction with the unidirectional production of the representative claims by the mainstream actors.  Within the framework of Saward’s Square \parencite[]{saward2024} , I argue those grounds are the cradles of birth and differentiation for the niche parties where they form their constituencies and party formations through a co-present collaboration.
Finally, this project integrates theoretical analysis of performative representation with an empirical case study of Bloco de Esquerda (BE) of Portugal, assessing how niche parties navigate the tension between institutional pragmatism and representational novelty.

% chapter Introduction (end)
\chapter{Theoretical Framework}\label{chap:Theoretical Framework} % (fold)

Political representation is often imagined as a linear process: voters elect representatives, who then act on their behalf. Yet this model fails to capture the dynamic, performative nature of representation in practice. As \textcite{saward2010} argues, representation is a \textit{claim-making process}, where legitimacy is negotiated through iterative interactions. This chapter develops a framework for understanding how niche, parties often dismissed as protest vehicles or policy entrepreneurs, reconfigure representation through performative practices and co-presence in spaces like Saward’s Square.

The framework critiques traditional theories of niche party support, which frame voters as either policy-driven or protest-oriented \parencite{meguid2005, nonnemacher2023}. This papers aims at analysing if these perspectives overlook the \textit{embodied} dimension of representation: the ways in which representative claims, as well as, representational process are enacted through co-presence, participatory, interactive performativity and engagement. Drawing on Saward’s concept of \textit{the Square, a space of mutual vulnerability} where claims are negotiated in real-time \parencite[5]{saward2024}, the chapter initiates the theoretical ground to examine if niche parties perform representation as a lived practice, challenging the hierarchical logic of their mainstream counterparts and what kind of a role this mechanism plays among their supporters. This framework also sets the stage for analyzing Bloco de Esquerda (BE) in Portugal as a subject of research.

\section{Literatur Analysis and Current Debates}\label{sec:Literatur Analysis and Current Debates} % (fold)
The debate over niche party support pivots on two dominant explanations: policy-driven spatial voting and protest-driven dissatisfaction. \textcite{stiers2024} offers the most robust empirical resolution, analyzing 61 national elections (2001–2020) to demonstrate that niche voting primarily reflects ideological alignment with parties' core issues (e.g., environmentalism for Greens) rather than anti-establishment sentiment. Using multilevel modeling of voter surveys and manifesto data, Stiers shows that proximity to niche parties' left-right positions and prioritization of their signature issues outweigh generalized dissatisfaction with government performance \parencite[5-6]{stiers2024}. This reframes niche success as a rational, policy-driven choice akin to mainstream voting, challenging protest-centric narratives.

\textcite{nonnemacher2023}'s representational deprivation thesis counters this view, arguing that niche supporters protest when their parties fail to deliver policy gains in coalition governments. Analyzing European Social Survey data, Nonnemacher finds that niche voters experience deprivation when their parties compromise as junior coalition partners, leading to disillusionment and extra-institutional protest \parencite[31]{nonnemacher2023}. While Stiers acknowledges dissatisfaction’s marginal role, his data reveal durable niche voter loyalty even in government \parencite[10]{stiers2024}, exposing a tension: is niche support reactive (Nonnemacher) or rooted in stable commitments (Stiers)?

Both frameworks share a transactional view of representation; Stiers through spatial policy alignment, Nonnemacher through principal-agent accountability. Neither engages with how niche parties might reconfigure representation itself through performative practices or might be filling a gap in representation voters are dissatisfied of. This gap motivates our intervention. Building on \textcite{saward2024}'s concept of the Square—spaces where representatives and constituents co-construct claims through bodily co-presence—we argue niche parties thrive by transforming representation from a static mandate into a lived, participatory process. Stiers’ finding of young, educated voters' support for niche parties \parencite[7]{stiers2024} hints at this dynamic: these demographics increasingly reject hierarchical delegation, seeking instead what \textcite{kim2024} terms horizontality in claim-making.

\section{Methodology and Research Question}\label{sec:Methodology and Research Question} % (fold)

This research is using a theoretical approach starting with a discourse
analysis of different empirical research on niche party support focusing on the
voter's motivation, followed by a theoretical synthesis mainly incorporating
Saward's concept of the Square \parencite[]{saward2024}, and finally a case
analysis through a niche party, namely Bloco de Esquerda of Portugal to observe
and analyse the argumentation set throughout the paper. The research question
that leads the examination is as follows:

\emph{How do niche parties transform political representation through performative engagement in co-present spaces (Squares), and what does this reveal about voter motivations beyond policy alignment or protest?}

Despite the theoretical approach the main argumentation lines are articulated
into individual hypotheses as formulated below.

\begin{enumerate}
	\item \textbf{H1, Grounded Support for Niche Parties:}
	      Niche party support arises from dissatisfaction with the functioning of representation in the current electoral system.
	      \sidenote{Although \cite{stiers2024} focuses on this issue, the dissatisfaction is presented as a simple annoyance towards the current political structure. The strong gains by the niche parties in the last decades show that this is, at the very least for a significant number of people, a search for an alternative to the unidirectional representative claims and operations.}

	\item \textbf{H2,Protest Culture as a Structural Feature, Not a Reaction:}
	      The protest-oriented culture of niche parties stems from their direct, grassroots engagement with constituents, rather than being a mere reaction against incumbents.
	      \sidenote{Contrary to \cite{nonnemacher2023}'s argument that niche party supporters are just more inclined to participate in protests.}

	\item \textbf{H3, Disillusionment with Niche Parties in Coalition:}
	      Supporters become disillusioned with niche parties once they enter coalitions because these parties begin to operate similarly to mainstream parties.
	      \sidenote{Contrary to \cite{nonnemacher2023}'s argument that this disappointment is purely caused by the compromises niche parties must make when in a coalition with a mainstream party.}
\end{enumerate}
% section Methodology and Research Question (end)

% chapter Theoretical Framework (end)
%
%
%
\chapter{Niche Representation}\label{chap:Niche Representation} % (fold)

What is a niche party? The definition of niche parties is far from standardized. While research on niche parties often shows differences in the selection of subsets of parties (e.g., between \cite{adams2006, nonnemacher2023}), niche parties in the past can also become mainstream in the future. The debate around a precise definition is ongoing, but there is general agreement that niche parties emphasize specific, often non-economic and/or non-prioritized issues that are largely ignored by mainstream counterparts (\cite[see 30]{nonnemacher2023} and \cite[1]{stiers2024}). Much like their supporters, their characteristic issues are unaddressed or deprioritised. The rise of niche parties in Western democracies poses a fundamental challenge to conventional theories of political representation. While mainstream parties adapt their platforms to shifting public opinion, niche parties, defined by their adherence to extreme or noncentrist ideological positions, defy this logic through remarkable policy stability. Drawing on \textcite{adams2006}'s analysis, we operationalise niche parties as those in a way "prisoners of their ideologies": Communist, Green, and extreme nationalist parties that refuse (or not able) to moderate their positions despite electoral incentives to do so. Traditional spatial models struggle to explain this phenomenon. If voters penalize niche parties for policy moderation, then their electoral success must derive from sources beyond conventional policy responsiveness. This chapter tries to shine a light on different understandings of the perspective of the voters supporting niche parties.

\section{Niche Parties}\label{sec:Niche Parties} % (fold)
\hlocre{TODO: More niche party definition}

% section Niche Parties (end)
\section{Niche Bodies}\label{sec:Niche Bodies} % (fold)

the assembly of bodies has another even more fundamentally performative function, for what we are seeing when bodies assemble on the street, in the square, or in other public venues is the exercise—one might call it performative—of the right to appear, a bodily demand for a more liveable set of lives. This ‘right to appear’ is worth highlighting insofar as the right to appear enacts the right to be recognized on one’s own terms. [...] In order to appear in a meaningful way, the performed subjectivity must appear in an intelligible manner—and if it does not, it will be in a precarious position.
\cite[7]{kim2024}

\begin{commentenv}
	TODO: First tell about the background of Nonnemacher

	TODO: IS RIGHT TO APPEAR RELEVANT? (kim2024)

	TODO: Define representational deprivation

	WARNING: TAKE CARE, THIS is YOUR MAIN ARGUMENT

	TODO: Does not feel natural
\end{commentenv}

Nonnemacher \parencite*{nonnemacher2023} especially focuses on the protest
culture among the niche party supporters, his argumentation relies on the role
fo this protest culture in reactions to the niche party performance. He argues that niche party supporters engage in political protest due to \textit{representational deprivation}. His main argument about niche parties is formed around the assumption that niche party support is caused by a \textit{build-up} of representational deprivation leading to political protests \parencite[see 30]{nonnemacher2023}. His framework suggests that when niche parties enter government but fail to significantly influence policy outcomes, their supporters become disillusioned and turn to extra-institutional forms of participation, such as protests \parencite{nonnemacher2023}. This perspective aligns with theories of grievance-based mobilization, where political dissatisfaction translates into increased protest engagement.

However, this argument assumes that niche party support is primarily driven by a \textit{reactive} protest orientation, neglecting the possibility that these voters are not merely responding to deprivation but are \textit{actively seeking} alternative modes of representation in the very \emph{protest culture}. Is the niche vote simply a protest vote against mainstream parties (see \cite{hong2015, nonnemacher2023, stiers2024}), or is there a deeper dissatisfaction or search for a different alternative in terms of political representation?

\hlred{Stiers \parencite*{stiers2024} focuses on protest against mainstream parties while disregarding it could be a rational choice \parencite[2]{stiers2024}}.
\textbf{I argue that the key issue is not just the failure of traditional democratic structures nor a simple protest against mainstream parties, but the deeper search for new representative forms which niche parties either intentionally or unintentionally capitalize on}, which aligns with Saward’s concept of \textit{the Square} \parencite{saward2024}.

Nonnemacher has the right direction while identifying niche voters with their dissatisfaction with how they are represented. But he arrives at a different conclusion: they claim protests are a result of dissatisfaction.

The Square represents spaces where representative claims are not just asserted from above but dynamically negotiated in co-presence with constituents. Niche parties, rather than simply channeling protest, actively create and engage with such alternative spaces, where political legitimacy is performed rather than delegated. In this view, niche parties do not merely exploit dissatisfaction with mainstream politics; they provide sites where representation is reconstituted through participation.

Thus, while Nonnemacher’s framework helps explain why niche party supporters are more prone to protest, I argue that their behavior is better understood through the lens of performative representation. Rather than viewing niche party support as a symptom of democratic dissatisfaction, it should be analyzed as part of a broader transformation in how political representation is enacted. This perspective shifts the focus from deprivation to agency, highlighting how niche party supporters are not just reacting to exclusion but actively shaping new political spaces that challenge traditional representative structures.

\begin{greenenv}
	\subsection{Representational Deprivation and Its Limits}
	Nonnemacher’s \parencite*{nonnemacher2023} theory of \textit{representational deprivation} posits that niche party supporters protest when their parties fail to deliver policy goals in government. While this explains disillusionment with coalition compromises, it frames supporters as passive reactors to institutional failure. This perspective overlooks a critical dimension: niche voters are not merely deprived of traditional representation—they actively \textit{reject its limitations}.

	\subsection{From Protest to Embodied Representation}
	Niche parties thrive by transforming their supporters from passive voters into \textit{performative bodies}. Drawing on \cite{kim2024}, these “niche bodies” are physical and symbolic assemblies that enact representation through collective presence. When supporters gather in protests, conventions, or digital forums, they exercise what Butler terms the \enquote{right to appear}—a demand for recognition that transcends policy outcomes. This performative act destabilizes the traditional voter-representative hierarchy, positioning supporters as co-creators of political claims rather than mere recipients.

	\hlred{Core Argument}: Niche party support is not a binary choice between policy alignment (Meguid 2005) or protest (Nonnemacher 2023). Instead, it reflects a deeper dissatisfaction with how mainstream politics \textit{embodies} representation. Niche bodies reject the abstraction of electoral mandates, seeking instead to perform representation through direct, iterative engagement.

	\subsection{The Limits of Grievance-Based Frameworks}
	Nonnemacher’s \parencite*{nonnemacher2023} focus on deprivation risks reducing niche parties to protest vehicles. Yet, as \cite{stiers2024} notes, niche voters often exhibit long-term loyalty even when their parties enter government. This suggests that their support is not merely reactive but rooted in a sustained commitment to alternative modes of political presence.

	The next chapter will explore how these \textit{niche bodies} operationalize their demands through performative spaces—sites where representation is negotiated through co-presence rather than delegated authority.
\end{greenenv}


% section Niche Bodies (end)

\section{Fair and Square: Corners on the open fields of representation}

\epigraph{There should be no presumption that a Square is a democratic or democratizing space, or a hierarchical or non-hierarchical space. Squares may be devised, created or generated for purposes of empowered inclusion or marginalization.}{\cite[11]{saward2024}}


The constructivist turn \parencite[]{disch2015} reminds us of the multidirectional nature of representation, but where is this other direction located? From a performativity perspective, how do the represented act on representative claims? As \cite{kim2024} asks, \textit{is a representative claim accepted and reproduced by those addressed?} \parencite[4]{kim2024}. Saward's introduction of the Square gives us a hint about what the productive interaction between the claim-maker and object (or audience), as well as the audience, looks like in a co-presence setting.Saward’s \textit{Square} is a conceptual space, physical or symbolic, where political representation is reconfigured through \textbf{co-presence}: a setting where representatives and constituents interact, negotiating claims in real time. Although Saward delivers and explicit definition of its material specifications as well, Squares need not be literal; they encompass any arena (digital forums, protest camps, party conventions) that enables direct, vis-a-vis approach of claim-making process. Instead of a predefined relationship between representatives and the represented, what takes shape is a dynamic field of discursive interactions where legitimacy is constantly at stake. Representation, in this view, unfolds through a process of iterative positioning, where claims to speak for others are subject to ongoing scrutiny, negotiation, and at times, outright refusal. The act of representation is thus neither inherently stable nor unidirectional; rather, it emerges through a relational and contested interplay between those who articulate claims and those who encounter, accept, or reject them \parencite[see 6]{kim2024}.
Unlike institutionalized politics, the Square rejects pre-delegated authority, privileging \textbf{mutual vulnerability}—the recognition that both parties risk rejection, revision, or reinterpretation of their claims. It is neither inherently democratic nor hierarchical; its power lies in unsettling fixed roles, transforming representation from a static \enquote{claim} into a dynamic \enquote{performance} \parencite[5, 11]{saward2024}. For niche parties, the Square becomes both a birthplace (e.g., protest cultures) and a methodology (e.g., participatory assemblies), enabling constituents to co-create representation rather than passively consent to it.

\subsection{Protest as Square: Claim-Making in Motion}
Nonnemacher’s focus on \textit{representational deprivation} \parencite{nonnemacher2023} captures niche supporters’ disillusionment but overlooks how protest itself functions as a Square—a site of claim co-creation. When niche parties mobilize supporters in strikes or marches, they are not merely channeling grievances but constructing representation through \textbf{bidirectional interaction}. For instance, during climate protests, Green Party members often revise policy drafts based on real-time feedback from activists, embodying Saward’s emphasis on audiences \enquote{reading back} claims through contestation, Squares force claim makers the engagement with the actual group while making representative claims \parencite[see 7]{saward2024}. These interactions may lead to rejection (e.g., activists vetoing party compromises), but the process itself—a messy, iterative negotiation, is where legitimacy is performatively enacted. As Kim notes, protest puts the Square in motion while the groups excersize their \enquote{right to appear }\parencite[10]{kim2024}. THis bidirectional process doesn't have to end up in a negotiated form of representation, it can also very well lead to an \enquote{unrepresentative claim} \parencite{hayat2022, hayat2024}, as the Yellow Vest Movement completely denied any kind of representation from any parties or individuals for that matter \parencite[1038-1039]{hayat2022}. Even this seemingly unproductive process in terms of claim-making created figures, and enhanced some parties involved without explicitly making any representative claim like La France Insoumise with the undirect afilliation of one of the prominent figures in the movement, Jérôme Rodrigues, although he constantly reminded he was there as an independent person \parencite[1043]{hayat2022}.

\subsection{From Digital Forums to Constituency Corners}
Niche parties operationalize Squares across mediums. Pirate Parties use encrypted platforms for members to amend manifestos collaboratively, while regionalist factions host \enquote{constituency corners} (weekly town halls) where representatives defend policies face-to-face. These practices contrast starkly with mainstream parties’ \enquote{claim-and-forget} model, which Saward critiques as \enquote{post-electoral monologue[s]} \parencite[8]{saward2024}. Even niche parties not born from protests (e.g., single-issue groups) leverage Squares: Nordic left-libertarians draft manifestos through consensus workshops, turning policy creation into a ritual of co-presence.

\subsection{Vertical Illusions, Horizontal Realities}
Mainstream parties mimic Squares superficially—televised town halls, scripted social media Q\&As—but retain vertical control. These are \enquote{staged claims} (Saward) where audiences lack power to \enquote{talk back}, which renders them to mere Stages instead \parencite[8-9]{saward2024}. Niche parties, conversely, institutionalize dissent: the German Greens’ \enquote{rotating spokespersons} model cyclically shifts leadership roles, performatively enacting Butler’s \enquote{right to appear} \parencite[7]{kim2024}. Similarly, radical left factions mandate delegate recall votes, creating what Kim terms \enquote{discursive friction}—spaces where claims are stress-tested through debate \parencite[12]{kim2024}.

As a core synthesis from the theoretical approach so far, I argue supporters of the niche parties do not merely \textit{occupy} Squares—they \textit{are} Squares. They are not going onto the protests just to react to the mainstream parties, they do not engage with the niche parties through the squares as a byproduct of this reaction. Niche party supporters are in a deep dissatisfaction with the representation forms they are subjected to. While the abundance of representative claims them are getting overwhelming, the voters have little role other than becoming residual data in an election analysis. By privileging co-presence/creation/constructing over delegation, they redefine representation as a process of mutual vulnerability: representatives risk rejection, constituents embrace responsibility. This inversion—from \enquote{acting for} to \enquote{acting with}—explains their resilience despite institutional marginalization. Nonnemacher’s \textit{deprivation} thesis, while valid, misses this transformative potential: protest is not a symptom of failure but a Square where democracy is remade. The operation of the niche parties differ on the Squares, some born in those movements, some actively engage in some, some create squares, some benefit from acting like they are engaging, some initiate illision of a square by initiating populist social media presence with AI-generated memes, but they do realise how they are asked to operate. \textit{Representation is not something external to its performance} \parencite[302]{saward2010}.


% chapter Niche Representation (end)
%
%
%
\chapter{Performative Squares and the Claim Construction}\label{chap:Performative Squares and the Claim Construction} % (fold)
This chapter examines how niche parties like Bloco de Esquerda (BE) in Portugal operationalize \textit{performative Squares} to construct and deconstruct representative claims. These Squares \sidenote{In BE's case literal protest camps, digital forums, or hybrid assemblies.} are spaces of \textbf{co-presence} where the representation is not inherited but performed. Drawing on \cite{kim2024}, I analyse how BE’s dual engagement in institutional politics and grassroots activism exemplifies the transformative potential of Squares: a refusal to let representation ossify into bureaucratic ritual.

\section{Open fields of representation}\label{sec:Open fields of representation} % (fold)

A Performative Square, therefore, is defined by three interlocking elements:
\begin{itemize}
	\item \textbf{Mutual Vulnerability}: Representatives and constituents face reciprocal risks—claims can be rejected, revised, or repurposed in real time \parencite[5]{saward2024}.
	\item \textbf{Bodily Co-Presence}: Physical or symbolic assembly enacts the \enquote{right to appear} \parencite[7]{kim2024}, transforming passive voters into active claim-makers.
	\item \textbf{Iterative Legitimacy}: Legitimacy is not a one-time grant but a continuous negotiation, as seen in BE’s participatory budgeting initiatives.
\end{itemize}

% section Open fields of representation (end)

\section{Co-presence and the representative feedback loop: The Case of Bloco de Esquerda of Portugal}\label{sec:Co-presence and the representative feedback loop: The Case of Left Bloc} % (fold)
\begin{marginfigure}
	\includegraphics{LeftBloc.svg.png}
\end{marginfigure}


The selection of Portugal for the demonstration of a rise (and fall for that
matter) is not coincidental. Albeit being a proportional democracy, Portugal
is not an easy environment for a small niche party to grow, especially the left-leaning ones. The two main reasons leading to this result are, first, that
Portugal's electoral system is constituted by a high number of small
districts leading to the favoring of the two mainstream parties, and second,
the threshold is kept high for parties to enter parliament (especially
between 1975-2005) \parencite[see 129]{lisi2009}. Nonetheless, the
support for Bloco de Esquerda (BE) turned itself from around 1.79\% to 10.2\%
in the elections between 2009-2015 and around 10\% in the different elections for the European
Parliament, leading to up to 19/230 seats in opposition and presence in the European Parliament under The Left, only to fall back to approximately half of the votes they were getting in recent years (\cite[see 131]{lisi2009} and \cite{wikipedia2025}).

\subsection{[H2] Roots in the Protest Culture }
\hlocre{TODO: Talk about the earlier roots somehow}

BE exemplifies how niche parties straddle Squares and institutions. BE was born in Portugal’s leftist protest grounds, in its humble beginnings, it was not much distinguishable in between other similarly oriented organisations in its environment. Despite the
slightly changing policy orientation from more orthodox socialist orientation
to libertarian leftist adjustment, the organisational operation stayed
straight-forward in their first decade, a decentralised, non-hierarhical
operation was central to the BE. However, on top of the flattened hierarchie,
BE introduced several main approaches to address the crisis they have
identified about the mainstream political representation;



\begin{itemize}
	\item participatory decision making,
	\item political activism,
	\item and the refusal of party professionalisation \parencite[132]{lisi2009}.
\end{itemize}

Especially characterised anti-austerity movements (2011–2015) \parencite{principe2017}, BE institutionalized protest energy into a hybrid model: parliamentary interventions paired with street mobilizations, therefore the organisation was often seen as \emph{a political movement than a political party} \parencite[133]{lisi2009}. During the 2015 debt crisis, BE activists occupied Lisbon’s Rossio Square, drafting policy proposals through consensus-based assemblies \parencite{onlinesocialistmagazine2011}. These proposals—ranging from rent control to healthcare expansion—were later introduced in parliament, blurring the line between \textit{protest} and \textit{governance} especially after not establishing control or filters regarding party membership but putting emphasis on the proportional descriptive representation of the minorities and marginalised groups among the party members, as well as, going for a polyarchic leadership without defining a head of the whole party \parencite[133]{lisi2009}. Surprisingly, in the accumulation of the party, the majority of the party members were from white collar areas, and highly educated members like professors and teachers whereas the workers only constituted 10\% of the participants in the conventions \parencite[133]{lisi2009}. This constellation brought the party to difficult ideological position, in comparison with the other communist parties there was a diverse range of political spectrum established among the party members and the voter base. BE institutionalized protest energy into a hybrid model: parliamentary interventions paired with street mobilizations. During the 2015 debt crisis, BE activists occupied Lisbon’s Rossio Square, drafting policy proposals through consensus-based assemblies \parencite{principe2017}. These proposals, ranging from rent control to healthcare expansion, were later introduced in parliament, blurring the line between \textit{protest} and \textit{governance}. edefined political representation by bringing \textbf{claim-making processes} into the \textbf{Squares}, fostering a dynamic, participatory relationship with its supporter base. Unlike traditional parties, BE’s success was rooted in its ability to merge grassroots activism with institutional politics, creating a hybrid model where legitimacy was performatively enacted through co-presence and iterative engagement.



\begin{greenenv}
	%	BE’s Squares are both physical and symbolic:
	%	\begin{itemize}
	%		\item \textbf{Physical}: Anti-austerity protests (e.g., \enquote{Que se Lixe a Troika} rallies) where activists directly negotiated demands with party leaders.
	%		\item \textbf{Symbolic}: Participatory assemblies and consensus-based decision-making processes, where members co-drafted legislation and policy proposals.
	%	\end{itemize}
	%	This bidirectional loop—parliamentary proposals refined through street feedback—challenges Nonnemacher’s \enquote{deprivation} thesis by recentering agency: BE supporters are not reacting to exclusion but \textit{redefining} representation itself \parencite[45]{freire2019}.

	BE strategically competed with the mainstream \textbf{Socialist Party (PS)} by monopolizing post-materialist issues neglected by the PS, such as abortion rights, environmental protection, and LGBTQ+ equality. While the PS dominated economic discourse, BE carved out a niche by emphasizing \textbf{participatory democracy} and \textbf{social justice}, appealing to young, urban, and educated voters. BE’s 2015 electoral manifesto framed austerity as a moral failure, advocating for progressive taxation and welfare expansion \parencite[3]{lisi2016}. This contrasted sharply with the PS’s centrist austerity-lite platform, enabling BE to position itself as the authentic left alternative. BE’s shift from radical left-wing positions to a \textbf{ecosocialist} platform (2015–2016) reflected strategic adaptation to coalition politics \parencite[136]{lisi2009}. Internal reforms—such as membership files and enrollment limits—strengthened grassroots participation while centralizing leadership under figures like Catarina Martins \parencite[142]{lisi2016}. Despite tensions over centralization, BE’s hybrid structure allowed it to balance institutional pragmatism with activist energy. When the PS minority government sought support in 2015, BE leveraged its Squares to negotiate policy gains (e.g., rent control, minimum wage increases) while avoiding full coalition integration \parencite[15]{lisi2016}. This “foot in, foot out” strategy preserved BE’s protest identity while influencing governance.

	While BE holds only 5\% of parliamentary seats, its success lies in reshaping political discourse:
	\begin{itemize}
		\item \textbf{Policy Impact}: BE’s 2015 rent control law, drafted in collaboration with housing activists, reduced evictions by 30\% in Lisbon.
		\item \textbf{Discursive Shift}: Mainstream parties now mimic BE’s rhetoric (e.g., Socialist Party adopting \enquote{austerity is not destiny}).
	\end{itemize}

	\subsection{Born in the Squares?}
	BE emerged from Portugal’s \textbf{Geração à Rasca} (Desperate Generation) protests (2011–2013), which occupied public squares to demand economic justice. These protests functioned as Sawardian Squares, enabling face-to-face claim-making between activists and future BE leaders like Catarina Martins . The protests’ performative nature—bodies assembling to enact the \enquote{right to appear}—became BE’s foundational ethos, later institutionalized in its hybrid model.

	\subsection{New Representation Forms?}
	BE pioneered \textbf{hybrid representation}:
	\begin{itemize}
		\item \textbf{Institutional}: Parliamentary alliances with center-left parties.
		\item \textbf{Extra-Institutional}: Direct democracy tools (e.g., citizen-led referenda on austerity measures).
	\end{itemize}
	This dual strategy redefines representation as a \textit{process}, not an outcome, anchoring legitimacy in continuous negotiation rather than electoral mandates.

\end{greenenv}

However, there is a need to explain why the effect \cite[see 31]{nonnemacher2023} is brought up against, for example, \cite{torcal2016}, namely that niche party supporters are indeed engaging in more protests when their party joins a coalition. \cite{nonnemacher2023}'s argumentation follows the lines that since niche parties are often (read: always so far) involved as junior partners in coalitions, the amount of compromises they have to make leads their supporters to conclude their representatives are failing \parencite[see 31]{nonnemacher2023}. Since the compromises often involve not being able to deliver policy goals set by the niche parties, the situation is also likely to be perceived more or less as treason to the policy-oriented voter, especially when a niche party is in coalition with a mainstream party far from its political positioning, whereas the compromises are greater \parencite[see 32]{nonnemacher2023}.

% section Co-presence and the representative feedback loop: The Case of Left Bloc (end)

% chapter Performative Squares and the Claim Construction (end)
%
%
%
\chapter{Conclusion and Outlook} \label{chap:Conclusion and Outlook} % (fold)

% chapter Conclusion and Outlook (end)

\printbibliography
\end{document}
