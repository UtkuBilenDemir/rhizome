%%%%%%%%%%%%%%%%%%%%%%%%%%%% Define Article %%%%%%%%%%%%%%%%%%%%%%%%%%%%%%%%%%
\documentclass[nobib, openany, justified, a4paper, 14pt]{tufte-book}
%%%%%%%%%%%%%%%%%%%%%%%%%%%%%%%%%%%%%%%%%%%%%%%%%%%%%%%%%%%%%%%%%%%%%%%%%%%%%%%

%%%%%%%%%%%%%%%%%%%%%%%%%%%%% Citations %%%%%%%%%%%%%%%%%%%%%%%%%%%%%%%%%%%%%%%
%\usepackage[utf8]{inputenc}
\usepackage[style=authoryear-icomp]{biblatex}
%\usepackage[style=apa]{biblatex}
\addbibresource{/Users/ubd/Bibliotheca/bib.bib}
%%%%%%%%%%%%%%%%%%%%%%%%%%%%%%%%%%%%%%%%%%%%%%%%%%%%%%%%%%%%%%%%%%%%%%%%%%%%%%%

%%%%%%%%%%%%%%%%%%%%%%%%%%%%% Using Packages %%%%%%%%%%%%%%%%%%%%%%%%%%%%%%%%%%
\usepackage{newunicodechar}
\newunicodechar{🦜}{[parrot]}
\PassOptionsToPackage{prologue,dvipsnames}{xcolor}
\sloppy  % globally
\usepackage{geometry}
\usepackage{graphicx}
\usepackage{amssymb}
\usepackage{amsmath}
\usepackage{amsthm}
\usepackage{empheq}
\usepackage{mdframed}
\usepackage{booktabs}
\usepackage{lipsum}
\usepackage{graphicx}
\usepackage{color}
\usepackage{psfrag}
\usepackage{pgfplots}
\usepackage{bm}
\usepackage{epigraph}
\usepackage{titlesec}
\usepackage{tcolorbox}
\usepackage{csquotes}
\usepackage{pifont}
\usepackage{enumitem,amssymb}
% \usepackage{spoton} % adds \todo functionality I hope
%%%%%%%%%%%%%%%%%%%%%%%%%%%%%%%%%%%%%%%%%%%%%%%%%%%%%%%%%%%%%%%%%%%%%%%%%%%%%%%

% Other Settings

%%%%%%%%%%%%%%%%%%%%%%%%%% Page Setting %%%%%%%%%%%%%%%%%%%%%%%%%%%%%%%%%%%%%%%

%%%%%%%%%%%%%%%%%%%%%%%%%% Define some useful colors %%%%%%%%%%%%%%%%%%%%%%%%%%
\definecolor{maroon}{RGB}{128,0,0} %hlred
\definecolor{MAROON}{RGB}{128,0,0} %hlred
\definecolor{deepBlue}{RGB}{61,124,222} %url-links
\definecolor{deepGreen}{RGB}{26,111,0} %citations
\definecolor{ocre}{RGB}{243,102,25}
\definecolor{mygray}{RGB}{243,243,244}
\definecolor{shallowGreen}{RGB}{235,255,255}
\definecolor{shallowBlue}{RGB}{235,249,255}
\definecolor{mediumpersianBlue}{rgb}{0.0, 0.4, 0.65}
\definecolor{persianBlue}{rgb}{0.11, 0.22, 0.73}
\definecolor{persianGreen}{rgb}{0.0, 0.65, 0.58}
\definecolor{persianRed}{rgb}{0.8, 0.2, 0.2}
\definecolor{debianRed}{rgb}{0.84, 0.04, 0.33}
%%%%%%%%%%%%%%%%%%%%%%%%%%%%%%%%%%%%%%%%%%%%%%%%%%%%%%%%%%%%%%%%%%%%%%%%%%%%%%%

%%%%%%%%%%%%%%%%%%%%%%%%%% Indentation Settings %%%%%%%%%%%%%%%%%%%%%%%%%%%%%%%
\makeatletter
% Paragraph indentation and separation for normal text
\renewcommand{\@tufte@reset@par}{%
	\setlength{\RaggedRightParindent}{0pc}%1.0
	\setlength{\JustifyingParindent}{0pc}%1.0
	\setlength{\parindent}{1pc}%1pc
	\setlength{\parskip}{5pt}%0pt
}
\@tufte@reset@par

% Paragraph indentation and separation for marginal text
\renewcommand{\@tufte@margin@par}{%
	\setlength{\RaggedRightParindent}{0pc}%0.5pc
	\setlength{\JustifyingParindent}{0pc}%0.5pc
	\setlength{\parindent}{0.5pc}%
	\setlength{\parskip}{5pt}%0pt
}
\makeatother



%%%%%%%%%%%%%%%%%%%%%%%%%% Define an orangebox command %%%%%%%%%%%%%%%%%%%%%%%%
%o\usepackage[most]{tcolorbox}

\newtcolorbox{orangebox}{
	colframe=ocre,
	colback=mygray,
	boxrule=0.8pt,
	arc=0pt,
	left=2pt,
	right=2pt,
	width=\linewidth,
	boxsep=4pt
}


\newtcolorbox{redbox}{
	colframe=red,
	boxrule=0.8pt,
	arc=0pt,
	left=2pt,
	right=2pt,
	width=\linewidth,
	boxsep=4pt
}
%%%%%%%%%%%%%%%%%%%%%%%%%%%%%%%%%%%%%%%%%%%%%%%%%%%%%%%%%%%%%%%%%%%%%%%%%%%%%%%

%%%%%%%%%%%%%%%%%%%%%%%%%%%% English Environments %%%%%%%%%%%%%%%%%%%%%%%%%%%%%
\newtheoremstyle{mytheoremstyle}{3pt}{3pt}{\normalfont}{0cm}{\rmfamily\bfseries}{}{1em}{{\color{black}\thmname{#1}~\thmnumber{#2}}\thmnote{\,--\,#3}}
\newtheoremstyle{myproblemstyle}{3pt}{3pt}{\normalfont}{0cm}{\rmfamily\bfseries}{}{1em}{{\color{black}\thmname{#1}~\thmnumber{#2}}\thmnote{\,--\,#3}}
\theoremstyle{mytheoremstyle}
\newmdtheoremenv[linewidth=1pt,backgroundcolor=shallowGreen,linecolor=deepGreen,leftmargin=0pt,innerleftmargin=20pt,innerrightmargin=20pt,]{theorem}{Theorem}[section]
\theoremstyle{mytheoremstyle}
\newmdtheoremenv[linewidth=1pt,backgroundcolor=shallowBlue,linecolor=deepBlue,leftmargin=0pt,innerleftmargin=20pt,innerrightmargin=20pt,]{definition}{Definition}[section]
\theoremstyle{myproblemstyle}
\newmdtheoremenv[linecolor=black,leftmargin=0pt,innerleftmargin=10pt,innerrightmargin=10pt,]{problem}{Problem}[section]
%%%%%%%%%%%%%%%%%%%%%%%%%%%%%%%%%%%%%%%%%%%%%%%%%%%%%%%%%%%%%%%%%%%%%%%%%%%%%%%

%%%%%%%%%%%%%%%%%%%%%%%%%%%%%%% Plotting Settings %%%%%%%%%%%%%%%%%%%%%%%%%%%%%
\usepgfplotslibrary{colorbrewer}
\pgfplotsset{width=8cm,compat=1.9}
%%%%%%%%%%%%%%%%%%%%%%%%%%%%%%%%%%%%%%%%%%%%%%%%%%%%%%%%%%%%%%%%%%%%%%%%%%%%%%%

%%%%%%%%%%%%%%%%%%%%%%%%%%%%%%% MISC %%%%%%%%%%%%%%%%%%%%%%%%%%%%%%%%%%%%%%%%%%
\usepackage[acronym]{glossaries}
\usepackage{hyperref} % Setup: https://www.overleaf.com/learn/latex/Hyperlinks
\hypersetup{
	colorlinks=true,
	%citecolor=deepGreen,
	citecolor=maroon,
	linkcolor=persianBlue,
	filecolor=persianGreen,
	urlcolor=persianBlue,
	pdfpagemode=FullScreen,
}

%%%%%%%%%%%%%%%%%%%%%%%%%%%%%%%%%%%%%%%%%%%%%%%%%%%%%%%%%%%%%%%%%%%%%%%%%%%%%%%
\setcounter{tocdepth}{2}
\setcounter{secnumdepth}{2}

\newcommand{\hlred}[1]{\textcolor{Maroon}{#1}} % Print text in maroon
\newcommand{\hlgreen}[1]{\textcolor{persianGreen}{#1}} % Print text in green
\newcommand{\hlocre}[1]{\textcolor{ocre}{#1}} % Print text in green

\newenvironment{greenenv}{\color{Green}}{\ignorespacesafterend}  % Create green environment
\newenvironment{commentenv}{\color{ocre}}{\ignorespacesafterend}  % Create comment environment


\titleformat{\part}[display]
{\filleft\fontsize{40}{40}\selectfont\scshape}
{\fontsize{90}{90}\selectfont\thepart}
{20pt}
{\thispagestyle{epigraph}}

\setlength\epigraphwidth{.6\textwidth}

%\makeatletter
%\epigraphhead
%{\let\@evenfoot}
%{\let\@oddfoot\@empty\let\@evenfoot}
%{}{}
%\makeatother


%%%%%%%%%%%%%%%%%%%%%%%%%%%%%%%%%%%%%%%%%%%%%%%%%%%%%%%%%%%%%%%%%%%%%%%%%%%%%%%
%TODO LIST
\newlist{todolist}{itemize}{2}
\setlist[todolist]{label=$\square$}
\newcommand{\cmark}{\ding{51}}%
\newcommand{\xmark}{\ding{55}}%
\newcommand{\done}{\rlap{$\square$}{\raisebox{2pt}{\large\hspace{1pt}\cmark}}%
	\hspace{-2.5pt}}
\newcommand{\wontfix}{\rlap{$\square$}{\large\hspace{1pt}\xmark}}

%%%%%%%%%%%%%%%%%%%%%%%%%%%%%%%%%%%%%%%%%%%%%%%%%%%%%%%%%%%%%%%%%%%%%%%%%%%%%%%
\newcommand{\greensquare}{\marginnote{\fcolorbox{green}{green}{\rule{0pt}{3mm}\rule{3mm}{0pt}}\quad}}
\newcommand{\yellowsquare}{\marginnote{\fcolorbox{yellow}{yellow}{\rule{0pt}{3mm}\rule{3mm}{0pt}}\quad}}
\newcommand{\redsquare}{\marginnote{\fcolorbox{red}{red}{\rule{0pt}{3mm}\rule{3mm}{0pt}}\quad}}



\usepackage{syntonly}
\usepackage{titlesec}
\setcounter{tocdepth}{2}
\setcounter{secnumdepth}{2}
%%%%%%%%%%%%%%%%%%%%%%%%%%%%%%% Title & Author %%%%%%%%%%%%%%%%%%%%%%%%%%%%%%%%
\title{From Stages\\to Squares:\\Representation by\\the niche parties\\and the
political\\presence (\hlred{Role of Performativity?})}
\author{Utku B. Demir \newline \small{Word count: 4948} }
%%%%%%%%%%%%%%%%%%%%%%%%%%%%%%%%%%%%%%%%%%%%%%%%%%%%%%%%%%%%%%%%%%%%%%%%%%%%%%%%
%\syntaxonly
\begin{document}
\maketitle
\tableofcontents


\chapter{Introduction}\label{chap:Introduction} % (fold)

The rise of niche parties across Europe, from green movements to anti-austerity coalitions, marks a tectonic shift in how citizens engage with representative democracy. While mainstream parties cling to institutional legitimacy, niche parties thrive in the liminal spaces between protest and governance, redefining political representation as a dynamic, embodied practice. This project analyses the increasing support for niche parties from the perspective of their constituencies. Are these voters merely protesting systemic failures, expressing their dissatisfaction with the incumbents, adhering to rigidly defined ideological stances or policy preferences, or are they demanding a fundamentally different mode of representation—one that transcends the transactional logic of ballots and policy trade-offs?

Traditional scholarship frames niche parties through a binary lens. For some, they are policy entrepreneurs capitalising on owned issues neglected by mainstream competitors \parencite{meguid2005}. For others, they channel protest votes from citizens disillusioned by representational deprivation \parencite{nonnemacher2023}. Yet such accounts may reduce representation to a static contract, overlooking its \textit{performative} dimension. Does support for niche parties relate to the specific processes through which representative claims are shaped \parencite{saward2010}? Drawing on Saward’s concept of \textit{the Square}—a space of mutual vulnerability where representatives and constituents negotiate claims in co-presence \parencite[5]{saward2024}—this study argues that niche parties succeed not by perfecting policy platforms but by transforming representation into a lived, participatory practice.

This study attempts to shift the perspective of grievance-based explanations of niche party support, particularly reflecting on Nonnemacher’s \parencite*{nonnemacher2023} framework, which interprets niche party supporters’ protest activity as a reaction to government participation. While coalition compromises may contribute to disillusionment, I argue that protest grounds themselves function as a \textit{performative act of representation}—a co-present way of structuring representative claims through interactions between supporters and party formation.

Niche party supporters do not simply \textit{respond} to current political conditions; they actively search for alternative approaches to representation, expressing their deep dissatisfaction with the unidirectional production of representative claims by mainstream actors. Within the framework of Saward’s Square \parencite[]{saward2024}, I argue that these protest grounds serve as cradles of birth and differentiation for niche parties, where they form their constituencies and party structures through co-present collaboration. Finally, this project integrates a theoretical analysis of performative representation with an empirical case study of Bloco de Esquerda (BE) in Portugal, assessing how niche parties navigate the tension between institutional pragmatism and representational novelty.

% chapter Introduction (end)
\chapter{Theoretical Framework}\label{chap:Theoretical Framework} % (fold)

Political representation is often imagined as a linear process: voters elect representatives, who then act on their behalf. Yet this model fails to capture the dynamic, performative nature of representation in practice. As \textcite{saward2010} argues, representation is a \textit{claim-making process}, where legitimacy is negotiated through iterative interactions. This chapter develops a framework for understanding how niche parties—often dismissed as protest vehicles or policy entrepreneurs—reconfigure representation through performative practices and co-presence in spaces like Saward’s Square.

The framework critiques traditional theories of niche party support, which frame voters as either policy-driven or protest-oriented \parencite{meguid2005, nonnemacher2023}. This paper aims to analyse whether these perspectives overlook the \textit{embodied} dimension of representation: the ways in which representative claims, as well as the representational process, are enacted through co-presence, participatory engagement, and interactive performativity. Drawing on Saward’s concept of \textit{the Square, a space of mutual vulnerability} where claims are negotiated in real time \parencite[5]{saward2024}, this chapter establishes the theoretical foundation for examining whether niche parties perform representation as a lived practice, challenging the hierarchical logic of their mainstream counterparts and investigating the role this mechanism plays among their supporters. This framework also sets the stage for analysing Bloco de Esquerda (BE) in Portugal as a case study.

\section{Literature Analysis and Current Debates}\label{sec:Literature Analysis and Current Debates} % (fold)
The debate over niche party support pivots on two dominant explanations: policy-driven spatial voting and protest-driven dissatisfaction. \textcite{stiers2024} offers the most robust empirical resolution, analysing 61 national elections (2001–2020) to demonstrate that niche voting primarily reflects ideological alignment with parties' core issues (e.g., environmentalism for Greens) rather than anti-establishment sentiment. Using multilevel modelling of voter surveys and manifesto data, Stiers shows that proximity to niche parties' left-right positions and prioritisation of their signature issues outweighs generalised dissatisfaction with government performance \parencite[5-6]{stiers2024}. This reframes niche party success as a rational, policy-driven choice akin to mainstream voting, challenging protest-centric narratives.

\textcite{nonnemacher2023}'s representational deprivation thesis counters this view, arguing that niche supporters protest when their parties fail to deliver policy gains in coalition governments. Analysing European Social Survey data, Nonnemacher finds that niche voters experience deprivation when their parties compromise as junior coalition partners, leading to disillusionment and extra-institutional protest \parencite[31]{nonnemacher2023}. While Stiers acknowledges dissatisfaction’s marginal role, his data reveal durable niche voter loyalty even in government \parencite[10]{stiers2024}, exposing a tension: is niche support reactive (Nonnemacher) or rooted in stable commitments (Stiers)?

Both frameworks share a transactional view of representation—Stiers through spatial policy alignment, Nonnemacher through principal-agent accountability. Neither engages with how niche parties might reconfigure representation itself through performative practices or how they might fill a representational gap that voters perceive. This gap motivates our intervention. Building on \textcite{saward2024}'s concept of the Square—spaces where representatives and constituents co-construct claims through bodily co-presence—we argue that niche parties thrive by transforming representation from a static mandate into a lived, participatory process. Stiers’ findings on young, educated voters’ support for niche parties \parencite[7]{stiers2024} hint at this dynamic: these demographics increasingly reject hierarchical delegation, seeking instead what \textcite{kim2024} terms horizontality in claim-making.

\section{Methodology and Research Question}\label{sec:Methodology and Research Question} % (fold)

This research follows a theoretical approach, beginning with a discourse
analysis of empirical studies on niche party support, focusing on voter motivation. It then synthesises these insights with theoretical perspectives, primarily incorporating Saward's concept of the Square \parencite[]{saward2024}, before applying them to a case study of Bloco de Esquerda in Portugal. The research question guiding this investigation is as follows:

\emph{How do niche parties transform political representation through performative engagement in co-present spaces (Squares), and what does this reveal about voter motivations beyond policy alignment or protest?}

Despite its theoretical approach, the main lines of argumentation are articulated
into individual hypotheses, as formulated below.

\begin{enumerate}
	\item \textbf{H1, Grounded Support for Niche Parties:}
	      Niche party support arises from dissatisfaction with the functioning of representation in the current electoral system.
	      \sidenote{Although \cite{stiers2024} focuses on this issue, dissatisfaction is presented as a simple annoyance towards the current political structure. The strong gains by niche parties in the last decades show that this is, at the very least, for a significant number of people, a search for an alternative to the unidirectional representative claims and operations.}

	\item \textbf{H2, Protest Culture as a Structural Feature, Not a Reaction:}
	      The protest-oriented culture of niche parties stems from their direct, grassroots engagement with constituents, rather than being a mere reaction against incumbents.
	      \sidenote{Contrary to \cite{nonnemacher2023}'s argument that niche party supporters are just more inclined to participate in protests.}

	\item \textbf{H3, Disillusionment with Niche Parties in Coalition:}
	      Supporters become disillusioned with niche parties once they enter coalitions because these parties begin to operate similarly to mainstream parties.
	      \sidenote{Contrary to \cite{nonnemacher2023}'s argument that this disappointment is purely caused by the compromises niche parties must make when in a coalition with a mainstream party.}
\end{enumerate}
% chapter Theoretical Framework (end)
%
%
%
\chapter{Niche Representation}\label{chap:Niche Representation} % (fold)

What is a niche party? The definition of niche parties is far from standardized. While research on niche parties often shows differences in the selection of subsets of parties (e.g., between \cite{adams2006, nonnemacher2023}), niche parties in the past can also become mainstream in the future. The debate around a precise definition is ongoing, but there is general agreement that niche parties emphasize specific, often non-economic and/or non-prioritized issues that are largely ignored by mainstream counterparts (\cite[see 30]{nonnemacher2023} and \cite[1]{stiers2024}). Much like their supporters, their characteristic issues are unaddressed or deprioritised. The rise of niche parties in Western democracies poses a fundamental challenge to conventional theories of political representation. While mainstream parties adapt their platforms to shifting public opinion, niche parties, defined by their adherence to extreme or noncentrist ideological positions, defy this logic through remarkable policy stability. Drawing on \textcite{adams2006}'s analysis, we operationalise niche parties as those in a way "prisoners of their ideologies": Communist, Green, and extreme nationalist parties that refuse (or not able) to moderate their positions despite electoral incentives to do so. Traditional spatial models struggle to explain this phenomenon. If voters penalize niche parties for policy moderation, then their electoral success must derive from sources beyond conventional policy responsiveness. This chapter tries to shine a light on different understandings of the perspective of the voters supporting niche parties.


\begin{quote}
	[...] the assembly of bodies has another even more fundamentally performative function, for what we are seeing when bodies assemble on the street, in the square, or in other public venues is the exercise—one might call it performative—of the right to appear, a bodily demand for a more liveable set of lives. This ‘right to appear’ is worth highlighting insofar as the right to appear enacts the right to be recognized on one’s own terms. [...] In order to appear in a meaningful way, the performed subjectivity must appear in an intelligible manner—and if it does not, it will be in a precarious position.

	- \cite[7]{kim2024}
\end{quote}

The research on the niche parties mainly focuses on their ideological and
organisational structures and their drastic differences to the mainstream
parties (see e.g. \cite{adams2006} and \cite{meguid2005}). The specifications
that often associated these parties rely on their rigid policy positions and
often radical ideological positions. Their connection to their constituencies
are more often then not explained through those questions. However, the
research on the voters of the niche parties are on a much more thinner side
\parencite[2]{stiers2024}, only a few publications reflect on the issue from
this perspective in the past \parencite[26-27]{nonnemacher2023}. 2 leading publications in the recent years namely
the ones from \textcite{nonnemacher2023} and \textcite{stiers2024} are trying
to shed light on how the \emph{niche bodies} are formed and what explanation
there might be for their support for the niche parties. While these research
are much more complex in the nature, this chapter will be dealing only the
relevant parts relating to the hypotheses of this study.


\section{Prologue to [H1]: Niche party support as a protest vote}
Stiers \parencite*{stiers2024} focuses on the explanation of the niche party
vote as a protest form against the mainstream/incumbent parties \parencite[2]{stiers2024}. This argumentation is already relatively well established in the literature. As much as the claims that niche party vote might be also explained from a purely ideological perspective. Stiers’ quantitative analysis examines both perspectives and ultimately challenges the notion that niche party support is primarily an expression of frustration with mainstream politics. His findings indicate that voters choosing niche parties are not simply casting protest votes but are actively prioritizing specific issues and aligning with parties that reflect their ideological preferences. While dissatisfaction with the political system or the current government may play a role, it appears to be a secondary factor rather than the primary driver of niche voting behavior \parencite[see 6-9]{stiers2024}.

While Stiers' study relates to the both of the issues in depth , especially
since the dissatisfaction is only measured with a narrow set of variables,
\sidenote{Namely with the variables `Dissatisfaction Democracy` and
	`Dissatisfaction Performance` \parencite[see 5-6]{stiers2024}.}
the question remains that what kind of dissatisfaction is ecactly examined and
is it possible to rely on the magnitude estimated. Although a general
dissatisfaction with democracy can be an indication, a dissatisfaction with the
how the voters are represented or in a more abstract sense how the
representational mechanism works might not be covered under both general
dissatisfaction with democracy or with the incumbent. I argue that the key issue is not just the failure of traditional democratic structures nor a simple protest against mainstream parties, but the deeper search for new representative forms which niche parties either intentionally or unintentionally capitalize on, which aligns with Saward’s concept of \textit{the Square} (\textcite{saward2024}, see next chapter).


\section{Prologue to [H2]: Representational Deprivation and Protest Culture}
Nonnemacher \parencite*{nonnemacher2023} emphasises the tendency among niche
protest voters towards a protest culture. He especially focuses on the protest
culture among the niche party supporters, his argumentation relies on the role
fo this protest culture in reactions to the niche party performance
\parencite[see 25-26]{nonnemacher2023}. He argues that niche party supporters engage in political protest due to \textit{representational deprivation}. This argument is mainly formed around the assumption that niche party support is caused by a \textit{build-up} of representational deprivation leading to political protests \parencite[see 30]{nonnemacher2023}. The claim is that the protest culture is a reactive way to the dissatisfaction with the incumbent or mainstream parties.

However, this argument assumes that niche party support is primarily driven by a \textit{reactive} protest orientation, neglecting the possibility that these voters are not merely responding to deprivation but they might be \textit{actively seeking} alternative modes of representation in the very \emph{protest culture}. Is the niche vote simply a protest vote against mainstream parties (see \cite{hong2015, nonnemacher2023, stiers2024}), or is there a deeper dissatisfaction or search for a different alternative in terms of political representation? Saying in a different way, do protest culture might have a productive nature, or is it a mere reaction?

Thus, while Nonnemacher’s framework helps explain why niche party supporters are more prone to protest, I argue that their behavior is better understood through the lens of performative representation. Rather than viewing niche party support as a symptom of democratic dissatisfaction, it should be analyzed as part of a broader transformation in how political representation is enacted. This perspective shifts the focus from deprivation to agency, highlighting how niche party supporters are not just reacting to exclusion but actively shaping new political spaces that challenge traditional representative structures.


\section{Prologue to [H3]: Protests continue despite the success?}
In his framework, Nonnemacher also attempts to tackle the question why the
protest culture ofen continues even if the niche party finds a place in the
coalition \parencite[34-38]{nonnemacher2023} . Nonnemacher's framework yields that when niche parties in the coalition often diverge away from the position of the prime minister's party. Be it because of the rigid ideological structures niche parties might often hold or their constant adjustment to find a strategic position in the opposition, the results show as the distance gets greater there is a higher likelihood that niche party supporters continue protesting \parencite[36-37]{nonnemacher2023}.

Nonnemacher’s \parencite*{nonnemacher2023} focus on deprivation risks reducing niche parties to protest vehicles. Yet, as \cite{stiers2024} notes, niche voters often exhibit long-term loyalty even when their parties enter government. This suggests that their support is not merely reactive but rooted in a sustained commitment to alternative modes of political presence. However, I launch the question what other compromisses niche parties are likely making if we assume that the main driver of the niche party voters' motivations is their dissatisfaction with how they are being represented. Wouldn't also be likely for them to sway away form the niche party as well once the approaches at representation starts to change when the party's involvement in the parliament gets deeper? Or can the compromisses niche parties have to accept also lead them to a less interactive form of representation in the long run? Are their supporters simply become disillusioned and turn to extra-institutional forms of participation, such as protests once again?


I claim niche party support is not a binary choice between policy alignment (Meguid 2005) or protest (Nonnemacher 2023). Instead, it reflects a deeper dissatisfaction with how mainstream politics \textit{embodies} representation. Niche bodies reject the abstraction of electoral mandates, seeking instead to perform representation through direct, iterative engagement. The next chapter will explore how these \textit{niche bodies} operationalize their demands through performative spaces—sites where representation is negotiated through co-presence rather than delegated authority.


% section Niche Bodies (end)

\chapter{Fair and Square: Corners on the Open Fields of Representation}

\epigraph{There should be no presumption that a Square is a democratic or democratising space, or a hierarchical or non-hierarchical space. Squares may be devised, created or generated for purposes of empowered inclusion or marginalisation.}{\cite[11]{saward2024}}

The constructivist turn \parencite[]{disch2015} reminds us of the multidirectional nature of representation, but where is this other direction located? From a performativity perspective, how do the represented act on representative claims? As \cite{kim2024} asks, \textit{is a representative claim accepted and reproduced by those addressed?} \parencite[4]{kim2024}. I argue that behind the increasing support for niche parties, an unnamed variable plays a pivotal role. Niche parties, whether intentionally or unintentionally, genuinely or for the sake of populism, offer constituencies something they cannot obtain from mainstream parties: an actual interactive representative claim construction process, facilitated through platforms, co-presence, and proximity within a flatter hierarchical structure. \textcite{saward2024}'s recent conceptualisation of the \emph{Square} provides a framework for understanding this underexplored effect.

Saward's introduction of the Square offers insight into how productive interaction occurs between the claim-maker and object (or audience), as well as among the audience itself, in a co-presence setting. Saward’s \textit{Square} is a conceptual space, physical or symbolic, where political representation is reconfigured through \textbf{co-presence}: a setting where representatives and constituents interact, negotiating claims in real time. Although Saward provides an explicit definition of its material specifications, Squares need not be literal; they encompass any arena (digital forums, protest camps, party conventions) that enables a direct, face-to-face approach to the claim-making process. Instead of a predefined relationship between representatives and the represented, what takes shape is a dynamic field of discursive interactions where legitimacy is constantly at stake. Representation, in this view, unfolds through a process of iterative positioning, where claims to speak for others are subject to ongoing scrutiny, negotiation, and, at times, outright rejection. The act of representation is thus neither inherently stable nor unidirectional; rather, it emerges through a relational and contested interplay between those who articulate claims and those who encounter, accept, or reject them \parencite[see 6]{kim2024}.

Unlike institutionalised politics, the Square rejects pre-delegated authority, privileging \textbf{mutual vulnerability}—the recognition that both parties risk rejection, revision, or reinterpretation of their claims. It is neither inherently democratic nor hierarchical; its power lies in unsettling fixed roles, transforming representation from a static \enquote{claim} into a dynamic \enquote{performance} \parencite[5, 11]{saward2024}. For niche parties, the Square serves both as a birthplace (e.g., protest cultures) and as a methodology (e.g., participatory assemblies), enabling constituents to co-create representation rather than passively consent to it.

\section{Protest as Square: Claim-Making in Motion}
Nonnemacher’s focus on \textit{representational deprivation} \parencite{nonnemacher2023} captures niche supporters’ disillusionment but overlooks how protest itself functions as a Square—a site of claim co-creation. When niche parties mobilise supporters in strikes or marches, they are not merely channelling grievances but constructing representation through \textbf{bidirectional interaction}. For instance, during climate protests, Green Party members often revise policy drafts based on real-time feedback from activists, embodying Saward’s emphasis on audiences \enquote{reading back} claims through contestation. Squares compel claim-makers to engage with the actual group while making representative claims \parencite[see 7]{saward2024}. These interactions may lead to rejection (e.g., activists vetoing party compromises), but the process itself—a messy, iterative negotiation—is where legitimacy is performatively enacted. As Kim notes, protest puts the Square in motion while groups exercise their \enquote{right to appear} \parencite[10]{kim2024}.

This bidirectional process does not necessarily result in a negotiated form of representation; it can also lead to an \enquote{unrepresentative claim} \parencite{hayat2022, hayat2024}, as seen in the Yellow Vest Movement, which entirely rejected any form of representation from parties or individuals \parencite[1038-1039]{hayat2022}. Even this seemingly unproductive process in terms of claim-making produced figures and enhanced some parties’ visibility without explicitly making a representative claim. For instance, La France Insoumise gained indirect affiliation with one of the movement’s prominent figures, Jérôme Rodrigues, despite his repeated assertions that he was participating as an independent individual \parencite[1043]{hayat2022}.

\section{From Digital Forums to Constituency Corners}
Niche parties operationalise Squares across different mediums. Pirate Parties use encrypted platforms for members to amend manifestos collaboratively, while regionalist factions host \enquote{constituency corners} (weekly town halls) where representatives defend policies face-to-face. These practices contrast starkly with mainstream parties’ \enquote{claim-and-forget} model, which Saward critiques as \enquote{post-electoral monologue[s]} \parencite[8]{saward2024}. Even niche parties that did not emerge from protests (e.g., single-issue groups) leverage Squares: Nordic left-libertarians draft manifestos through consensus workshops, transforming policy creation into a ritual of co-presence.

\section{Vertical Illusions, Horizontal Realities}
Mainstream parties mimic Squares superficially—televised town halls, scripted social media Q\&As—but retain vertical control. These are \enquote{staged claims} (Saward) where audiences lack the power to \enquote{talk back}, rendering them mere Stages instead \parencite[8-9]{saward2024}. Niche parties, conversely, institutionalise dissent: the German Greens’ \enquote{rotating spokespersons} model cyclically shifts leadership roles, performatively enacting Butler’s \enquote{right to appear} \parencite[7]{kim2024}. Similarly, radical left factions mandate delegate recall votes, creating what Kim terms \enquote{discursive friction}—spaces where claims are stress-tested through debate \parencite[12]{kim2024}.

As a core synthesis of the theoretical approach developed so far, I argue that supporters of niche parties do not merely \textit{occupy} Squares—they \textit{are} Squares. They do not protest simply as a reaction to mainstream parties, nor do they engage with niche parties through Squares as a by-product of this reaction. Niche party supporters are deeply dissatisfied with the representational structures imposed upon them. While they are inundated with representative claims, voters themselves have little role beyond becoming residual data in electoral analysis. By privileging co-presence, co-creation, and construction over delegation, they redefine representation as a process of mutual vulnerability: representatives risk rejection, while constituents embrace responsibility. This inversion—from \enquote{acting for} to \enquote{acting with}—explains their resilience despite institutional marginalisation. Nonnemacher’s \textit{deprivation} thesis, while valid, overlooks this transformative potential: protest is not merely a symptom of failure but a Square where democracy is remade.

The operation of niche parties differs across Squares: some are born from movements, some actively engage in them, some create their own Squares, while others benefit from appearing as if they do. Some initiate the illusion of a Square through populist social media presence and AI-generated memes, but they all recognise the expectations placed upon them. \textit{Representation is not something external to its performance} \parencite[302]{saward2010}.

The Square represents a space where representative claims are not just asserted from above but dynamically negotiated in co-presence with constituents. Niche parties, rather than simply channelling protest, actively create and engage with such alternative spaces, where political legitimacy is performed rather than delegated. In this view, niche parties do not merely exploit dissatisfaction with mainstream politics; they provide sites where representation is reconstituted through participation.
% chapter Niche Representation (end)
%
%
%
\chapter{Performative Squares and the Claim Construction}\label{chap:Performative Squares and the Claim Construction} % (fold)
This chapter examines how niche parties like Bloco de Esquerda (BE) in Portugal operationalise \textit{performative Squares} to construct and deconstruct representative claims. These Squares \sidenote{In BE’s case, literal protest camps, digital forums, or hybrid assemblies.} are spaces of \textbf{co-presence} where representation is not inherited but performed. Drawing on \cite{kim2024}, I analyse how BE’s dual engagement in institutional politics and grassroots activism exemplifies the transformative potential of Squares: a refusal to let representation ossify into bureaucratic ritual.

\section{Co-presence and the Representative Feedback Loop: The Case of Bloco de Esquerda in Portugal}\label{sec:Co-presence and the Representative Feedback Loop: The Case of Left Bloc} % (fold)
\begin{marginfigure}
	\includegraphics{LeftBloc.svg.png}
\end{marginfigure}

The selection of Portugal to demonstrate both the rise and fall of a niche party is not coincidental. Although Portugal operates under a proportional democracy, it does not provide an easy environment for small niche parties to grow, particularly left-leaning ones. Two main factors contribute to this difficulty: first, Portugal's electoral system consists of numerous small districts that favour the two mainstream parties; second, the threshold for parliamentary entry was historically high (especially between 1975 and 2005) \parencite[see 129]{lisi2009}. Nonetheless, BE transformed itself from securing around 1.79\% of the vote to reaching 10.2\% between 2009 and 2015, maintaining around 10\% in various elections for the European Parliament. This growth led to BE winning up to 19 out of 230 seats in opposition and securing representation in the European Parliament under *The Left*. However, in recent years, BE’s vote share has declined significantly, returning to approximately half of its previous support (\cite[see 131]{lisi2009} and \cite{wikipedia2025}).

BE exemplifies how niche parties straddle Squares and institutions. The party was born within Portugal’s leftist protest movements, and in its early years, it was not easily distinguishable from other similarly oriented organisations in its environment. Despite a gradual shift from a more orthodox socialist stance to a libertarian leftist approach, BE’s organisational operations remained largely unchanged throughout its first decade. The party prioritised a decentralised, non-hierarchical structure while introducing several key strategies aimed at addressing what it identified as a crisis in mainstream political representation. These strategies included:

\begin{itemize}
	\item participatory decision-making,
	\item political activism,
	\item and the refusal of party professionalisation \parencite[132]{lisi2009}.
\end{itemize}

Particularly during the anti-austerity movements (2011–2015) \parencite{principe2017}, BE institutionalised protest energy into a hybrid model that combined parliamentary interventions with street mobilisation. This unique structure meant that BE was often perceived as \emph{a political movement rather than a political party} \parencite[133]{lisi2009}. During the 2015 debt crisis, BE activists occupied Lisbon’s Rossio Square, drafting policy proposals through consensus-based assemblies \parencite{onlinesocialistmagazine2011}. These proposals—ranging from rent control to healthcare expansion—were later introduced in parliament, effectively blurring the line between \textit{protest} and \textit{governance}. Unlike traditional parties, BE did not impose strict control or filtering mechanisms regarding party membership but instead emphasised proportional descriptive representation of minorities and marginalised groups among its members. Additionally, the party adopted a polyarchic leadership model, foregoing the designation of a singular party head \parencite[133]{lisi2009}.

Interestingly, despite BE’s grassroots origins, its membership composition was predominantly white-collar professionals; professors and teachers made up a significant portion of the party, while industrial and manual workers constituted only about 10\% of participants in conventions \parencite[133]{lisi2009}. This internal sociological divide created ideological tensions within the party, differentiating it from other communist organisations, which typically maintained a more homogenous class profile. In contrast, BE’s structure facilitated a diverse range of political ideologies among both its membership and voter base. By embedding its claim-making processes within the Square, BE fostered a dynamic and participatory relationship with its supporters, merging grassroots activism with institutional politics.

BE strategically positioned itself in opposition to both mainstream right-wing parties and the \textbf{Socialist Party (PS)}. In particular, it distinguished itself from PS by monopolising post-materialist issues that the Socialist Party neglected, such as abortion rights, environmental protection, and LGBTQ+ equality. While PS dominated economic discourse, BE differentiated itself by championing \textbf{participatory democracy} and \textbf{social justice}, successfully appealing to younger, urban, and educated voters. BE’s 2015 electoral manifesto framed austerity as a moral failure, advocating for progressive taxation and welfare expansion \parencite[3]{lisi2016}. This stance contrasted sharply with PS’s centrist, austerity-lite platform, allowing BE to position itself as the authentic left-wing alternative.

BE’s shift from radical left-wing positions to an \textbf{eco-socialist} platform (2015–2016) reflected its strategic adaptation to coalition politics \parencite[136]{lisi2009}. Despite internal tensions over centralisation, BE’s hybrid structure allowed it to balance institutional pragmatism with activist energy. When the PS minority government sought external support in 2015, BE leveraged its Squares to negotiate policy gains (e.g., rent control, minimum wage increases) while simultaneously avoiding full coalition integration \parencite[15]{lisi2016}.


\subsection{[H1, H2] From Protest to Embodied Representation}

Niche parties thrive by transforming their supporters from passive voters into \textit{performative bodies}. Drawing on \cite{kim2024}, these “niche bodies” are physical and symbolic assemblies that enact representation through collective presence. When supporters gather in protests, conventions, or digital forums, they exercise what Butler terms the \enquote{right to appear}—a demand for recognition that transcends policy outcomes. This performative act destabilizes the traditional voter-representative hierarchy, positioning supporters as co-creators of political claims rather than mere recipients.

BE’s origins lie in a protest culture. It was not a reaction that led to the party’s formation; rather, the protest culture itself was its foundation. The BE constituency was deeply dissatisfied with various socio-political issues in the country; yet, I argue that the most significant factor (if not the most prominent) was disillusionment with mainstream parties' representative structures. The fact that a radical left party could attract such a diverse support base indicates how BE’s claim-making process was embraced across different sectors of society.

\subsection{[H3] Disillusionment and the Downfall of Niche Parties}

For the longest time, BE successfully maintained the "Square format" of the party; both its representative operations and decision-making system functioned within this framework. However, one might ask: what led to the decline in their electoral support in recent years?

Following the formation of a minority government with PS, BE's operational structure became overloaded, particularly due to the austerity measures the government was forced to reintroduce \parencite[see 17]{lisi2016}. This period brought significant changes to the party's internal structure. I argue that the centralization, increasing hierarchy, and detachment from BE's novel representative mechanisms led to the disillusionment of its supporters—especially those who had been engaged precisely because of the party’s Square-based operational and representative structure. This challenge is one that niche parties frequently encounter.
% section Co-presence and the representative feedback loop: The Case of Left Bloc (end)

% chapter Performative Squares and the Claim Construction (end)
%
%
%

\chapter{Conclusion}

This study aimed to reflect on the increasing support for the niche parties by
focusing on the significance of how their representative claims are
constructred by focusing on their supporter bases. The rise of niche parties is not merely a symptom of political dissatisfaction but the result of a search for the reconfiguration of representative processes themselves. Saward’s Square offers a useful framework for understanding how legitimacy emerges not from policy outcomes alone but from co-presence, where voters and representatives engage in ongoing claim-making.

Ultimately, niche parties are not anomalies within democracy but their constituencies signal where the evolution of the representation should be going. Their challenge lies in maintaining participatory legitimacy without succumbing to the constraints of institutionalization. As voters increasingly demand presence over delegation, the Square emerges not just as a theoretical model but as a democratic imperative one that invites both niche and mainstream parties to rethink how representation is enacted in practice.

\printbibliography
\end{document}
