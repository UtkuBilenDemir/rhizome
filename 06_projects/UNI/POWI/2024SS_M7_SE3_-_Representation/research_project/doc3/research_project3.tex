%%%%%%%%%%%%%%%%%%%%%%%%%%%% Define Article %%%%%%%%%%%%%%%%%%%%%%%%%%%%%%%%%%
\documentclass[nobib, openany, justified, a4paper, 14pt]{tufte-book}
%%%%%%%%%%%%%%%%%%%%%%%%%%%%%%%%%%%%%%%%%%%%%%%%%%%%%%%%%%%%%%%%%%%%%%%%%%%%%%%

%%%%%%%%%%%%%%%%%%%%%%%%%%%%% Citations %%%%%%%%%%%%%%%%%%%%%%%%%%%%%%%%%%%%%%%
%\usepackage[utf8]{inputenc}
\usepackage[style=authoryear-icomp]{biblatex}
%\usepackage[style=apa]{biblatex}
\addbibresource{/Users/ubd/Bibliotheca/bib.bib}
%%%%%%%%%%%%%%%%%%%%%%%%%%%%%%%%%%%%%%%%%%%%%%%%%%%%%%%%%%%%%%%%%%%%%%%%%%%%%%%

%%%%%%%%%%%%%%%%%%%%%%%%%%%%% Using Packages %%%%%%%%%%%%%%%%%%%%%%%%%%%%%%%%%%
\usepackage{newunicodechar}
\newunicodechar{🦜}{[parrot]}
\PassOptionsToPackage{prologue,dvipsnames}{xcolor}
\sloppy  % globally
\usepackage{geometry}
\usepackage{graphicx}
\usepackage{amssymb}
\usepackage{amsmath}
\usepackage{amsthm}
\usepackage{empheq}
\usepackage{mdframed}
\usepackage{booktabs}
\usepackage{lipsum}
\usepackage{graphicx}
\usepackage{color}
\usepackage{psfrag}
\usepackage{pgfplots}
\usepackage{bm}
\usepackage{epigraph}
\usepackage{titlesec}
\usepackage{tcolorbox}
\usepackage{csquotes}
\usepackage{pifont}
\usepackage{enumitem,amssymb}
% \usepackage{spoton} % adds \todo functionality I hope
%%%%%%%%%%%%%%%%%%%%%%%%%%%%%%%%%%%%%%%%%%%%%%%%%%%%%%%%%%%%%%%%%%%%%%%%%%%%%%%

% Other Settings

%%%%%%%%%%%%%%%%%%%%%%%%%% Page Setting %%%%%%%%%%%%%%%%%%%%%%%%%%%%%%%%%%%%%%%

%%%%%%%%%%%%%%%%%%%%%%%%%% Define some useful colors %%%%%%%%%%%%%%%%%%%%%%%%%%
\definecolor{maroon}{RGB}{128,0,0} %hlred
\definecolor{MAROON}{RGB}{128,0,0} %hlred
\definecolor{deepBlue}{RGB}{61,124,222} %url-links
\definecolor{deepGreen}{RGB}{26,111,0} %citations
\definecolor{ocre}{RGB}{243,102,25}
\definecolor{mygray}{RGB}{243,243,244}
\definecolor{shallowGreen}{RGB}{235,255,255}
\definecolor{shallowBlue}{RGB}{235,249,255}
\definecolor{mediumpersianBlue}{rgb}{0.0, 0.4, 0.65}
\definecolor{persianBlue}{rgb}{0.11, 0.22, 0.73}
\definecolor{persianGreen}{rgb}{0.0, 0.65, 0.58}
\definecolor{persianRed}{rgb}{0.8, 0.2, 0.2}
\definecolor{debianRed}{rgb}{0.84, 0.04, 0.33}
%%%%%%%%%%%%%%%%%%%%%%%%%%%%%%%%%%%%%%%%%%%%%%%%%%%%%%%%%%%%%%%%%%%%%%%%%%%%%%%

%%%%%%%%%%%%%%%%%%%%%%%%%% Indentation Settings %%%%%%%%%%%%%%%%%%%%%%%%%%%%%%%
\makeatletter
% Paragraph indentation and separation for normal text
\renewcommand{\@tufte@reset@par}{%
	\setlength{\RaggedRightParindent}{0pc}%1.0
	\setlength{\JustifyingParindent}{0pc}%1.0
	\setlength{\parindent}{1pc}%1pc
	\setlength{\parskip}{5pt}%0pt
}
\@tufte@reset@par

% Paragraph indentation and separation for marginal text
\renewcommand{\@tufte@margin@par}{%
	\setlength{\RaggedRightParindent}{0pc}%0.5pc
	\setlength{\JustifyingParindent}{0pc}%0.5pc
	\setlength{\parindent}{0.5pc}%
	\setlength{\parskip}{5pt}%0pt
}
\makeatother



%%%%%%%%%%%%%%%%%%%%%%%%%% Define an orangebox command %%%%%%%%%%%%%%%%%%%%%%%%
%o\usepackage[most]{tcolorbox}

\newtcolorbox{orangebox}{
	colframe=ocre,
	colback=mygray,
	boxrule=0.8pt,
	arc=0pt,
	left=2pt,
	right=2pt,
	width=\linewidth,
	boxsep=4pt
}


\newtcolorbox{redbox}{
	colframe=red,
	boxrule=0.8pt,
	arc=0pt,
	left=2pt,
	right=2pt,
	width=\linewidth,
	boxsep=4pt
}
%%%%%%%%%%%%%%%%%%%%%%%%%%%%%%%%%%%%%%%%%%%%%%%%%%%%%%%%%%%%%%%%%%%%%%%%%%%%%%%

%%%%%%%%%%%%%%%%%%%%%%%%%%%% English Environments %%%%%%%%%%%%%%%%%%%%%%%%%%%%%
\newtheoremstyle{mytheoremstyle}{3pt}{3pt}{\normalfont}{0cm}{\rmfamily\bfseries}{}{1em}{{\color{black}\thmname{#1}~\thmnumber{#2}}\thmnote{\,--\,#3}}
\newtheoremstyle{myproblemstyle}{3pt}{3pt}{\normalfont}{0cm}{\rmfamily\bfseries}{}{1em}{{\color{black}\thmname{#1}~\thmnumber{#2}}\thmnote{\,--\,#3}}
\theoremstyle{mytheoremstyle}
\newmdtheoremenv[linewidth=1pt,backgroundcolor=shallowGreen,linecolor=deepGreen,leftmargin=0pt,innerleftmargin=20pt,innerrightmargin=20pt,]{theorem}{Theorem}[section]
\theoremstyle{mytheoremstyle}
\newmdtheoremenv[linewidth=1pt,backgroundcolor=shallowBlue,linecolor=deepBlue,leftmargin=0pt,innerleftmargin=20pt,innerrightmargin=20pt,]{definition}{Definition}[section]
\theoremstyle{myproblemstyle}
\newmdtheoremenv[linecolor=black,leftmargin=0pt,innerleftmargin=10pt,innerrightmargin=10pt,]{problem}{Problem}[section]
%%%%%%%%%%%%%%%%%%%%%%%%%%%%%%%%%%%%%%%%%%%%%%%%%%%%%%%%%%%%%%%%%%%%%%%%%%%%%%%

%%%%%%%%%%%%%%%%%%%%%%%%%%%%%%% Plotting Settings %%%%%%%%%%%%%%%%%%%%%%%%%%%%%
\usepgfplotslibrary{colorbrewer}
\pgfplotsset{width=8cm,compat=1.9}
%%%%%%%%%%%%%%%%%%%%%%%%%%%%%%%%%%%%%%%%%%%%%%%%%%%%%%%%%%%%%%%%%%%%%%%%%%%%%%%

%%%%%%%%%%%%%%%%%%%%%%%%%%%%%%% MISC %%%%%%%%%%%%%%%%%%%%%%%%%%%%%%%%%%%%%%%%%%
\usepackage[acronym]{glossaries}
\usepackage{hyperref} % Setup: https://www.overleaf.com/learn/latex/Hyperlinks
\hypersetup{
	colorlinks=true,
	%citecolor=deepGreen,
	citecolor=maroon,
	linkcolor=persianBlue,
	filecolor=persianGreen,
	urlcolor=persianBlue,
	pdfpagemode=FullScreen,
}

%%%%%%%%%%%%%%%%%%%%%%%%%%%%%%%%%%%%%%%%%%%%%%%%%%%%%%%%%%%%%%%%%%%%%%%%%%%%%%%
\setcounter{tocdepth}{2}
\setcounter{secnumdepth}{2}

\newcommand{\hlred}[1]{\textcolor{Maroon}{#1}} % Print text in maroon
\newcommand{\hlgreen}[1]{\textcolor{persianGreen}{#1}} % Print text in green
\newcommand{\hlocre}[1]{\textcolor{ocre}{#1}} % Print text in green

\newenvironment{greenenv}{\color{Green}}{\ignorespacesafterend}  % Create green environment
\newenvironment{commentenv}{\color{ocre}}{\ignorespacesafterend}  % Create comment environment


\titleformat{\part}[display]
{\filleft\fontsize{40}{40}\selectfont\scshape}
{\fontsize{90}{90}\selectfont\thepart}
{20pt}
{\thispagestyle{epigraph}}

\setlength\epigraphwidth{.6\textwidth}

%\makeatletter
%\epigraphhead
%{\let\@evenfoot}
%{\let\@oddfoot\@empty\let\@evenfoot}
%{}{}
%\makeatother


%%%%%%%%%%%%%%%%%%%%%%%%%%%%%%%%%%%%%%%%%%%%%%%%%%%%%%%%%%%%%%%%%%%%%%%%%%%%%%%
%TODO LIST
\newlist{todolist}{itemize}{2}
\setlist[todolist]{label=$\square$}
\newcommand{\cmark}{\ding{51}}%
\newcommand{\xmark}{\ding{55}}%
\newcommand{\done}{\rlap{$\square$}{\raisebox{2pt}{\large\hspace{1pt}\cmark}}%
	\hspace{-2.5pt}}
\newcommand{\wontfix}{\rlap{$\square$}{\large\hspace{1pt}\xmark}}

%%%%%%%%%%%%%%%%%%%%%%%%%%%%%%%%%%%%%%%%%%%%%%%%%%%%%%%%%%%%%%%%%%%%%%%%%%%%%%%
\newcommand{\greensquare}{\marginnote{\fcolorbox{green}{green}{\rule{0pt}{3mm}\rule{3mm}{0pt}}\quad}}
\newcommand{\yellowsquare}{\marginnote{\fcolorbox{yellow}{yellow}{\rule{0pt}{3mm}\rule{3mm}{0pt}}\quad}}
\newcommand{\redsquare}{\marginnote{\fcolorbox{red}{red}{\rule{0pt}{3mm}\rule{3mm}{0pt}}\quad}}



\usepackage{tcolorbox}
%%%%%%%%%%%%%%%%%%%%%%%%%%%%%%% Title & Author %%%%%%%%%%%%%%%%%%%%%%%%%%%%%%%%
\title{From Stages to Squares: Representation by the Niche Parties and the Political Presence (\hlred{Role of Performativity?})}
\author{Utku B. Demir}
%%%%%%%%%%%%%%%%%%%%%%%%%%%%%%%%%%%%%%%%%%%%%%%%%%%%%%%%%%%%%%%%%%%%%%%%%%%%%%%%

\begin{document}
\maketitle

\chapter{Introduction}\label{chap:Introduction}
\epigraph{The opposite of representation is not participation}{\cite[19]{plotke1997}}

\chapter{Theoretical Framework}\label{chap:Theoretical Framework and Current Debates}

\section{Literature Analysis and Current Debates}\label{chap:Literatur Analysis}
%%TODO: Refer to Stiers2024, he has a detailed explanation
%
%\textcite{blumenau2024}
%
%Although research about niche parties and their representation approaches varies, there has been very little research on niche party support at an individual level. Two recent publications focused on this topic: first through the protest-oriented spirit of niche party voters \parencite{nonnemacher2023}, and a more general look at why the reasons for niche party support might exist \parencite{stiers2024}.
%
%This paper does not disregard the policy orientation of niche parties. It is by now a well-established argument that some voters find that their political perspectives align very well with the issues "owned" by specific niche parties \parencite{meguid2005}. Nor does it overlook that a part of the votes are simply protest reactions to mainstream parties, as Nonnemacher \cite*{nonnemacher2023} argues.
%
%However, I argue that \cite{nonnemacher2023}'s claim that niche party voters take to the streets once they realize their party's influence is negligible is only partly true. Niche party voters are also likely to realize that even after their party gains a position in government, the fundamental attributes of electoral democracy keep them as distant from the operation of the square as before.
%
%Niche party voters are not just policy-driven extremists or disaffected protestors.
%
%\hlred{New Claim}: I argue the opposite of what Nonnemacher claims. Rather than niche party supporters being protest-oriented individuals, I argue their support is much more related to the nature of representation they experience in the democracies they live in. I think they develop their support through closer engagement with the (to-be) representatives of niche parties, and the reason for it is not merely (or not only) niche parties' greater involvement in protests or social events but their reflection of Saward's concept of the \textit{Square}.
%%TODO: Define Square
%Even if the notion of the Square may not apply to the connection between niche party members and their (potential) constituencies, the engagement itself simulates a different construction method for their representation.
%
\section{Methodology and Research Question}\label{sec:Methodology and Research Question}

\chapter{Main Chapter}\label{chap:Main Chapter}

%What is a niche party? The definition of niche parties is far from standardized. While research on niche parties often shows differences in the selection of subsets of parties (e.g., between \cite{adams2006, nonnemacher2023}), niche parties in the past can also become mainstream in the future. The debate around a precise definition is ongoing, but there is general agreement that niche parties emphasize specific, often non-economic and/or non-prioritized issues that are largely ignored by mainstream counterparts (\cite[see 30]{nonnemacher2023} and \cite[1]{stiers2024}). Much like their supporters, their characteristic issues are unaddressed or deprioritized.
%
%Over the last decades, we have witnessed a significant vote shift from mainstream parties to niche parties (\cite[see 1]{stiers2024} and \cite{spoon2019}).
%
%This literature has shown that niche parties are most successful if they stay true to their more extreme positions without moderating them following shifts in political opinion (Adams et al., 2006; Ezrow, 2008)—although this depends on the issue on which the party focuses (Bergman and Flatt, 2020).
%\cite[2]{stiers2024}
%
\section{Saward's Square}\label{sec:Saward's Square}
%\epigraph{There should be no presumption that a Square is a democratic or democratizing space, or a hierarchical or non-hierarchical space. Squares may be devised, created or generated for purposes of empowered inclusion or marginalization.}{\cite[11]{saward2024}}
%
%Representative claims matter much more when they are developed on the same level as the constituencies themselves.
%
%\section{Where is exactly the constructivism in the performative claims?}\label{sec:Where is exactly the construction in the construtivist turn?}
%\epigraph{[\ldots] representation is not something external to its performance.}{\cite[302]{saward2010}}
%
%The constructivist turn \parencite[]{disch2015} reminds us of the multidirectional nature of representation, but where is this other direction located? From a performativity perspective, how do the represented act on representative claims? As \cite{kim2024} asks, \textit{is a representative claim accepted and reproduced by those addressed?} \parencite[4]{kim2024}.
%
%Saward's introduction of the Square gives us a hint about what the productive interaction between the claim-maker and object (or audience), as well as the audience, looks like in a co-presence setting. As much as the Square is an abstraction \sidenote{Although Saward tries to define it as explicitly as possible as a concrete space with a set of specifications; he still notes this should help with the abstraction of the concept. %TODO: Citation Needed
%}, in the literal sense, it also includes settings like smaller party gatherings or conventions where bodily interactions are convened in different horizontal and vertical ways \parencite[see 4]{kim2024}. Yet, to consider representation merely as a structured mechanism that delivers political presence would be misleading. Instead of a predefined relationship between representatives and the represented, what takes shape is a dynamic field of discursive interactions where legitimacy is constantly at stake. Representation, in this view, unfolds through a process of iterative positioning, where claims to speak for others are subject to ongoing scrutiny, negotiation, and at times, outright refusal. The act of representation is thus neither inherently stable nor unidirectional; rather, it emerges through a relational and contested interplay between those who articulate claims and those who encounter, accept, or reject them \parencite[see 6]{kim2024}.
%
%Ultimately, Saward's work releases the performative practice of claim-making from institutionalized settings \parencite[4]{kim2024}.
%
%\hlred{We are talking about people's engagement with the referent associated with them, how they accept or reject it and how they shape it.}

\section{Niche Bodies: What Forms a Constituency for the Niche Parties}\label{sec:Niche Bodies}
%\begin{quote}
%	Thirdly, aside from the enactment of horizontality/equality, the assembly of bodies has another even more fundamentally performative function, for what we are seeing when bodies assemble on the street, in the square, or in other public venues is the exercise—one might call it performative—of the right to appear, a bodily demand for a more liveable set of lives. This ‘right to appear’ is worth highlighting insofar as the right to appear enacts the right to be recognized on one’s own terms. The latter can play a role for movement activists in political parties and can hence be an important element of horizontal politics. The right to appear interlinks with a fourth and final aspect that we would like to highlight in Butler’s work: the question of the intelligibility of a subject. In order to appear in a meaningful way, the performed subjectivity must appear in an intelligible manner—and if it does not, it will be in a precarious position. Butler makes this argument, particularly when she points to a subject’s gender identifiability as the very precondition for recognizing it as a living being:
%\end{quote}
%\cite[7]{kim2024}
%
%%TODO: First tell about the background of Nonnemacher
%%TODO: IS RIGHT TO APPEAR RELEVANT? (kim2024)
%%TODO: Define representational deprivation
%%WARNING: TAKE CARE, THIS is YOUR MAIN ARGUMENT
%%TODO: Does not feel natural
%Nonnemacher \parencite*{nonnemacher2023} argues that niche party supporters engage in political protest due to \textit{representational deprivation}. His main argument about niche parties is formed around the assumption that niche party support is caused by a \textit{build-up} of representational deprivation leading to political protests \parencite[see 30]{nonnemacher2023}. His framework suggests that when niche parties enter government but fail to significantly influence policy outcomes, their supporters become disillusioned and turn to extra-institutional forms of participation, such as protests \parencite{nonnemacher2023}. This perspective aligns with theories of grievance-based mobilization, where political dissatisfaction translates into increased protest engagement.
%
%However, this argument assumes that niche party support is primarily driven by a \textit{reactive} protest orientation, neglecting the possibility that these voters are not merely responding to deprivation but are \textit{actively seeking} alternative modes of representation. Is the niche vote simply a protest vote against mainstream parties (see \cite{hong2015, nonnemacher2023, stiers2024}), or is there a deeper dissatisfaction or search for a different alternative in terms of political representation?
%
%\hlred{Stiers \parencite*{stiers2024} focuses on protest against mainstream parties while disregarding it could be a rational choice \parencite[2]{stiers2024}}.
%\textbf{I argue that the key issue is not just the failure of traditional democratic structures nor a simple protest against mainstream parties, but the deeper search for new representative forms which niche parties either intentionally or unintentionally capitalize on}, which aligns with Saward’s concept of \textit{the Square} \parencite{saward2024}.
%
%Nonnemacher has the right direction while identifying niche voters with their dissatisfaction with how they are represented. But he arrives at a different conclusion: they claim protests are a result of dissatisfaction.
%
%The Square represents spaces where representative claims are not just asserted from above but dynamically negotiated in co-presence with constituents. Niche parties, rather than simply channeling protest, actively create and engage with such alternative spaces, where political legitimacy is performed rather than delegated. In this view, niche parties do not merely exploit dissatisfaction with mainstream politics; they provide sites where representation is reconstituted through participation.
%
%Thus, while Nonnemacher’s framework helps explain why niche party supporters are more prone to protest, I argue that their behavior is better understood through the lens of performative representation. Rather than viewing niche party support as a symptom of democratic dissatisfaction, it should be analyzed as part of a broader transformation in how political representation is enacted. This perspective shifts the focus from deprivation to agency, highlighting how niche party supporters are not just reacting to exclusion but actively shaping new political spaces that challenge traditional representative structures.

\chapter{Case Study or Empirical Work}\label{chap:Case Study or Empirical Work}

%\begin{tcolorbox}
%	Bloco de Esquerda (BE) in Portugal provides a concrete example of how niche parties operate within and beyond traditional representative institutions. Unlike mainstream parties, which rely on institutional legitimacy, BE blends electoral participation with grassroots activism, continuously fostering alternative representative spaces. Its history of engagement in anti-austerity protests and advocacy for participatory democracy suggests that its supporters are not just deprived of representation but are invested in redefining what representation means.
%
%	BE’s dual engagement in both electoral and extra-institutional politics illustrates how niche parties can function as sites of political experimentation, where alternative models of representation are enacted. Studying BE allows us to examine how niche parties not only reflect dissatisfaction with existing structures but also offer new ways to think about democratic engagement outside conventional institutions.
%\end{tcolorbox}
%
%However, there is a need to explain why the effect \citeauthoryear[see 31]{nonnemacher2023} is brought up against, for example, \citeauthoryear{torcal2016}, namely that niche party supporters are indeed engaging in more protests when their party joins a coalition. \citeauthoryear{Nonnemacher2023}'s argumentation follows the lines that since niche parties are often (read: always so far) involved as junior partners in coalitions, the amount of compromises they have to make leads their supporters to conclude their representatives are failing \parencite[see 31]{nonnemacher2023}. Since the compromises often involve not being able to deliver policy goals set by the niche parties, the situation is also likely to be perceived more or less as treason to the policy-oriented voter, especially when a niche party is in coalition with a mainstream party far from its political positioning, whereas the compromises are greater \parencite[see 32]{nonnemacher2023}.

%TODO:Argument it whereas the compromises are greater

\chapter{Quotes}\label{chap:quotes}

\section{Saward}\label{sec:saward}

\begin{quote}
	In contexts of physical co-presence of would-be representers and would-be represented, the fact of real-world, real-time presence raises the political stakes of representation—who controls the performance and its outcome, who is empowered or disempowered, placed or displaced?
\end{quote}
— \cite[3]{saward2024}

\begin{quote}
	The bottom line is this: ‘Squares’ for the purposes of this chapter might literally be square, or they might not; it is the physical co-presence of would-be representers and represented that is the focus, and not the exact configuration of the space of that co-presence (which may well change shape in or through a specific process of interaction).
\end{quote}
— \cite[5]{saward2024}

\begin{quote}
	The nature and consequences of performing representation in the Square are distinctive from more familiar contexts where there is not a physical co-presence between representers and represented. It is a space where you cannot speak about or for without also speaking to. Physical co-presence heightens the sense of contingency and uncertainty (how will these people react to each other, on the spot and in real time?). [...] \textbf{the Square is a space of ‘mutual vulnerability’}.
\end{quote}
— \cite[5]{saward2024}

\begin{quote}
	Two generalised possibilities for power dynamics in the site are (A) where structures, terms of entry, etc, advantage or privilege formal representative claim-makers, discourage or disable counterclaims by audience members, and set up a sharp division between who can act and who cannot; and (B) where the opposite is the case—differently positioned actors can move and speak in a more fluid structure, and so on. Case B holds out the promise of being characterised by a greater sense of equality—not as sameness, but as respect and recognition across difference.
\end{quote}
— \cite[13]{saward2024}

\chapter{Main}\label{chap:main}

As Saward notes:

\begin{quote}
	Political representation in contexts where there is not physical co-presence—which is most of the time—involves complex power relations between representers and represented. In general terms, representation in such contexts means that claimants to representation do not have the prospect of combined immediate plus face-to-face feedback.
\end{quote}

What is lacking in political representation is establishing a productive feedback cycle rather than responsiveness. That means establishing the framework to make the constituencies able to embody themselves \sidenote{As much as Saward's Square doesn't have to relate to any of the meanings of the word, the body here in its critical theory-related context is used liberatively, both as a body and as a substance of an individual.} and not remain a huge chunk of liquefied vote mass.

\printbibliography
\end{document}
