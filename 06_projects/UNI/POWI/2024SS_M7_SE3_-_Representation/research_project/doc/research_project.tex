%%%%%%%%%%%%%%%%%%%%%%%%%%%% Define Article %%%%%%%%%%%%%%%%%%%%%%%%%%%%%%%%%%
\documentclass[nobib, openany, justified, a4paper, 14pt]{tufte-book}
%%%%%%%%%%%%%%%%%%%%%%%%%%%%%%%%%%%%%%%%%%%%%%%%%%%%%%%%%%%%%%%%%%%%%%%%%%%%%%%

%%%%%%%%%%%%%%%%%%%%%%%%%%%%% Citations %%%%%%%%%%%%%%%%%%%%%%%%%%%%%%%%%%%%%%%
%\usepackage[utf8]{inputenc}
\usepackage[style=authoryear-icomp]{biblatex}
%\usepackage[style=apa]{biblatex}
\addbibresource{/Users/ubd/Bibliotheca/bib.bib}
%%%%%%%%%%%%%%%%%%%%%%%%%%%%%%%%%%%%%%%%%%%%%%%%%%%%%%%%%%%%%%%%%%%%%%%%%%%%%%%

%%%%%%%%%%%%%%%%%%%%%%%%%%%%% Using Packages %%%%%%%%%%%%%%%%%%%%%%%%%%%%%%%%%%
\usepackage{newunicodechar}
\newunicodechar{🦜}{[parrot]}
\PassOptionsToPackage{prologue,dvipsnames}{xcolor}
\sloppy  % globally
\usepackage{geometry}
\usepackage{graphicx}
\usepackage{amssymb}
\usepackage{amsmath}
\usepackage{amsthm}
\usepackage{empheq}
\usepackage{mdframed}
\usepackage{booktabs}
\usepackage{lipsum}
\usepackage{graphicx}
\usepackage{color}
\usepackage{psfrag}
\usepackage{pgfplots}
\usepackage{bm}
\usepackage{epigraph}
\usepackage{titlesec}
\usepackage{tcolorbox}
\usepackage{csquotes}
\usepackage{pifont}
\usepackage{enumitem,amssymb}
% \usepackage{spoton} % adds \todo functionality I hope
%%%%%%%%%%%%%%%%%%%%%%%%%%%%%%%%%%%%%%%%%%%%%%%%%%%%%%%%%%%%%%%%%%%%%%%%%%%%%%%

% Other Settings

%%%%%%%%%%%%%%%%%%%%%%%%%% Page Setting %%%%%%%%%%%%%%%%%%%%%%%%%%%%%%%%%%%%%%%

%%%%%%%%%%%%%%%%%%%%%%%%%% Define some useful colors %%%%%%%%%%%%%%%%%%%%%%%%%%
\definecolor{maroon}{RGB}{128,0,0} %hlred
\definecolor{MAROON}{RGB}{128,0,0} %hlred
\definecolor{deepBlue}{RGB}{61,124,222} %url-links
\definecolor{deepGreen}{RGB}{26,111,0} %citations
\definecolor{ocre}{RGB}{243,102,25}
\definecolor{mygray}{RGB}{243,243,244}
\definecolor{shallowGreen}{RGB}{235,255,255}
\definecolor{shallowBlue}{RGB}{235,249,255}
\definecolor{mediumpersianBlue}{rgb}{0.0, 0.4, 0.65}
\definecolor{persianBlue}{rgb}{0.11, 0.22, 0.73}
\definecolor{persianGreen}{rgb}{0.0, 0.65, 0.58}
\definecolor{persianRed}{rgb}{0.8, 0.2, 0.2}
\definecolor{debianRed}{rgb}{0.84, 0.04, 0.33}
%%%%%%%%%%%%%%%%%%%%%%%%%%%%%%%%%%%%%%%%%%%%%%%%%%%%%%%%%%%%%%%%%%%%%%%%%%%%%%%

%%%%%%%%%%%%%%%%%%%%%%%%%% Indentation Settings %%%%%%%%%%%%%%%%%%%%%%%%%%%%%%%
\makeatletter
% Paragraph indentation and separation for normal text
\renewcommand{\@tufte@reset@par}{%
	\setlength{\RaggedRightParindent}{0pc}%1.0
	\setlength{\JustifyingParindent}{0pc}%1.0
	\setlength{\parindent}{1pc}%1pc
	\setlength{\parskip}{5pt}%0pt
}
\@tufte@reset@par

% Paragraph indentation and separation for marginal text
\renewcommand{\@tufte@margin@par}{%
	\setlength{\RaggedRightParindent}{0pc}%0.5pc
	\setlength{\JustifyingParindent}{0pc}%0.5pc
	\setlength{\parindent}{0.5pc}%
	\setlength{\parskip}{5pt}%0pt
}
\makeatother



%%%%%%%%%%%%%%%%%%%%%%%%%% Define an orangebox command %%%%%%%%%%%%%%%%%%%%%%%%
%o\usepackage[most]{tcolorbox}

\newtcolorbox{orangebox}{
	colframe=ocre,
	colback=mygray,
	boxrule=0.8pt,
	arc=0pt,
	left=2pt,
	right=2pt,
	width=\linewidth,
	boxsep=4pt
}


\newtcolorbox{redbox}{
	colframe=red,
	boxrule=0.8pt,
	arc=0pt,
	left=2pt,
	right=2pt,
	width=\linewidth,
	boxsep=4pt
}
%%%%%%%%%%%%%%%%%%%%%%%%%%%%%%%%%%%%%%%%%%%%%%%%%%%%%%%%%%%%%%%%%%%%%%%%%%%%%%%

%%%%%%%%%%%%%%%%%%%%%%%%%%%% English Environments %%%%%%%%%%%%%%%%%%%%%%%%%%%%%
\newtheoremstyle{mytheoremstyle}{3pt}{3pt}{\normalfont}{0cm}{\rmfamily\bfseries}{}{1em}{{\color{black}\thmname{#1}~\thmnumber{#2}}\thmnote{\,--\,#3}}
\newtheoremstyle{myproblemstyle}{3pt}{3pt}{\normalfont}{0cm}{\rmfamily\bfseries}{}{1em}{{\color{black}\thmname{#1}~\thmnumber{#2}}\thmnote{\,--\,#3}}
\theoremstyle{mytheoremstyle}
\newmdtheoremenv[linewidth=1pt,backgroundcolor=shallowGreen,linecolor=deepGreen,leftmargin=0pt,innerleftmargin=20pt,innerrightmargin=20pt,]{theorem}{Theorem}[section]
\theoremstyle{mytheoremstyle}
\newmdtheoremenv[linewidth=1pt,backgroundcolor=shallowBlue,linecolor=deepBlue,leftmargin=0pt,innerleftmargin=20pt,innerrightmargin=20pt,]{definition}{Definition}[section]
\theoremstyle{myproblemstyle}
\newmdtheoremenv[linecolor=black,leftmargin=0pt,innerleftmargin=10pt,innerrightmargin=10pt,]{problem}{Problem}[section]
%%%%%%%%%%%%%%%%%%%%%%%%%%%%%%%%%%%%%%%%%%%%%%%%%%%%%%%%%%%%%%%%%%%%%%%%%%%%%%%

%%%%%%%%%%%%%%%%%%%%%%%%%%%%%%% Plotting Settings %%%%%%%%%%%%%%%%%%%%%%%%%%%%%
\usepgfplotslibrary{colorbrewer}
\pgfplotsset{width=8cm,compat=1.9}
%%%%%%%%%%%%%%%%%%%%%%%%%%%%%%%%%%%%%%%%%%%%%%%%%%%%%%%%%%%%%%%%%%%%%%%%%%%%%%%

%%%%%%%%%%%%%%%%%%%%%%%%%%%%%%% MISC %%%%%%%%%%%%%%%%%%%%%%%%%%%%%%%%%%%%%%%%%%
\usepackage[acronym]{glossaries}
\usepackage{hyperref} % Setup: https://www.overleaf.com/learn/latex/Hyperlinks
\hypersetup{
	colorlinks=true,
	%citecolor=deepGreen,
	citecolor=maroon,
	linkcolor=persianBlue,
	filecolor=persianGreen,
	urlcolor=persianBlue,
	pdfpagemode=FullScreen,
}

%%%%%%%%%%%%%%%%%%%%%%%%%%%%%%%%%%%%%%%%%%%%%%%%%%%%%%%%%%%%%%%%%%%%%%%%%%%%%%%
\setcounter{tocdepth}{2}
\setcounter{secnumdepth}{2}

\newcommand{\hlred}[1]{\textcolor{Maroon}{#1}} % Print text in maroon
\newcommand{\hlgreen}[1]{\textcolor{persianGreen}{#1}} % Print text in green
\newcommand{\hlocre}[1]{\textcolor{ocre}{#1}} % Print text in green

\newenvironment{greenenv}{\color{Green}}{\ignorespacesafterend}  % Create green environment
\newenvironment{commentenv}{\color{ocre}}{\ignorespacesafterend}  % Create comment environment


\titleformat{\part}[display]
{\filleft\fontsize{40}{40}\selectfont\scshape}
{\fontsize{90}{90}\selectfont\thepart}
{20pt}
{\thispagestyle{epigraph}}

\setlength\epigraphwidth{.6\textwidth}

%\makeatletter
%\epigraphhead
%{\let\@evenfoot}
%{\let\@oddfoot\@empty\let\@evenfoot}
%{}{}
%\makeatother


%%%%%%%%%%%%%%%%%%%%%%%%%%%%%%%%%%%%%%%%%%%%%%%%%%%%%%%%%%%%%%%%%%%%%%%%%%%%%%%
%TODO LIST
\newlist{todolist}{itemize}{2}
\setlist[todolist]{label=$\square$}
\newcommand{\cmark}{\ding{51}}%
\newcommand{\xmark}{\ding{55}}%
\newcommand{\done}{\rlap{$\square$}{\raisebox{2pt}{\large\hspace{1pt}\cmark}}%
	\hspace{-2.5pt}}
\newcommand{\wontfix}{\rlap{$\square$}{\large\hspace{1pt}\xmark}}

%%%%%%%%%%%%%%%%%%%%%%%%%%%%%%%%%%%%%%%%%%%%%%%%%%%%%%%%%%%%%%%%%%%%%%%%%%%%%%%
\newcommand{\greensquare}{\marginnote{\fcolorbox{green}{green}{\rule{0pt}{3mm}\rule{3mm}{0pt}}\quad}}
\newcommand{\yellowsquare}{\marginnote{\fcolorbox{yellow}{yellow}{\rule{0pt}{3mm}\rule{3mm}{0pt}}\quad}}
\newcommand{\redsquare}{\marginnote{\fcolorbox{red}{red}{\rule{0pt}{3mm}\rule{3mm}{0pt}}\quad}}



\usepackage{tcolorbox}
%%%%%%%%%%%%%%%%%%%%%%%%%%%%%%% Title & Author %%%%%%%%%%%%%%%%%%%%%%%%%%%%%%%%
%\title{Changing Preferences for Representation and the Support for Niche Parties; A Comparative Analysis Across EU Countries}
\title{From Stages to Squares: Representation by the niche parties and the
political presence (\hlred{Role of Performativity?})}
\author{Utku B. Demir}
%%%%%%%%%%%%%%%%%%%%%%%%%%%%%%%%%%%%%%%%%%%%%%%%%%%%%%%%%%%%%%%%%%%%%%%%%%%%%%%%

\begin{document}
\maketitle

\chapter{Introduction}\label{chap:Introduction} % (fold)
\epigraph{the opposite of representation is not participation}{\cite[19]{plotke1997}}

% chapter Introduction (end)

\chapter{Theoretical Framework}\label{chap:Theoretical Framework and Current Debates} % (fold)

% chapter Theoretical Framework and Current Debates (end)
\section{Literatur Analysis and Current Debates}\label{chap:Literatur Analysis} % (fold)
%TODO: Refer to Stiers2024, he has a detailed explanation

% \textcite{blumenau2024}
% 
% Altough the research about niche parties, and their representation approaches
% var, there has been very little research about on the niche party support on an
% individual level, 2 recent publications focused on this topic first through the
% protest oriented spirit of the niche party voters \parencite{nonnemacher2023},
% and a more general look at why the reasons for a niche party support on
% individual level might be \parencite{stiers2024}.
% 
% This paper is not disregarding the policy orientation of the niche parties. It
% is by now a well-established argument that some voters find out about that
% their political perspective very well align with the issues "owned" by specific
% niche parties \parencite{meguid2005}. Nor it is overlooking that a part of the
% votes are simply a protest reaction to the mainstream parties as Nonnemacher
% \cite*{nonnemacher2023} argues.
% 
% However, I argue that \cite{nonnemacher2023}'s claim that the niche party voters are
% going to the streets once they realise their party's influence is negligible is
% only partly true. Nice party voters are also likely to realise even after
% their party get a position in the government, the fundamental attributes of the
% electoral democracy keeping them away from what operation of the square are as
% effective as before.
% 
% Niche party voters are not just policy-driven extremists or disaffected protestors.
% 
% 
% \hlred{New Claim}: I argue the opposite of the Nonnemacher is claiming, rather
% than niche party supporters being the protest oriented individuals, I argue
% their support is much more related to the nature of the representation they are
% getting in the democracies they are living in. I think
% they develop their supports through the closer feeling of engagements with the
% (to-be) representatives of the niche parties and the reason for it is not
% merely (or not only) niche parties' more involvement in the protest or social
% events but their reflection of the Saward's concept of the \textit{Square}.
% %TODO: Define Square
% Even if the notion of the Square may not apply to the connection between the
% niche party members and their (potential) constitutiencies, the engagement
% itself simulates a different construction method for their representation.

% chapter Literatur Analysis (end)
\section{Methodology and Research Question}\label{sec:Methodology and Research Question} % (fold)

% section Methodology and Research Question (end)

\chapter{Main Chapter}\label{chap:Main Chapter} % (fold)

%What is a niche party? The definition of the niche parties are far away from
%being standardised. While research on niche parties often show differences in the selection of the subset of parties (e.g. between \cite{adams2006, nonnemacher2023}) niche parties in the past can also become mainstream in the future. The debate around a precise definition is ongoing bute there is a general agreement on that the niche parties are empasising some specific, often non-economic and/or not prioritised, issues that are largely ignored by their mainstream counterparts (\cite[see 30]{nonnemacher2023} and \cite[1]{stiers2024}). Much like their supporters, their characteristic issues are unaddressed or not prioritised.
%
%Over the last decades, we are experiencing a significant vote shift from the mainstream parties to the niche parties (\cite[see 1]{stiers2024} and \cite{spoon2019}).
%
%This literature has shown that niche parties are most successful if they stay true to their more extreme position without moderating their position following shifts in political opinion (Adams et al., 2006; Ezrow, 2008) – although this depends on the issue on which the party focuses (Bergman and Flatt, 2020).
%\cite[2]{stiers2024}


\section{Saward's Square}\label{sec:Saward's Square} % (fold)

\epigraph{There should be no presumption that a Square is a democratic or democratising space, or a hierarchical or non-hierarchical space. Squares may be devised, created or generated for purposes of empowered inclusion or marginalization.}{\cite[11]{saward2024}}

Representatiave claims matter much more when they are developed on the same
level with the constituencie themselves.

% section Saward's Square (end)
\section{Where is exactly the constructivism in the performative claims?}\label{sec:Where is exactly the construction in the construtivist turn?} % (fold)

\epigraph{[\ldots] representation is not something external to its performance.}{\cite[302]{saward2010}}

Constructivist turn \parencite[]{disch2015}  reminds us the multidirectional nature of the representation but where is this other direction again?
From a perspective of the performativity, how do the represented act on the representative claims?
Or as \cite{kim2024} ask, \textit{is a representative claim accepted and reproduced by those addressed?} \parencite[4]{kim2024}.

Saward's introduction of the Square gives us a hint about what the productive interaction
between the claim-maker and object
%TODO:Object or Auidience
, as well as, the audience looks like in a
co-presence setting. As much as the Square is an abstraction \sidenote{Although, Saward tries to define it as explicitly as possible as a concrete space with a set of specifications; he still notes this should help with the abstraction of the concept. %TODO: Citation Needed
}, in the
literal sense, it is also includes settings like smaller party gaterings or
conventions where bodily interactions are convienied in different horizontal
and vertical ways \parencite[see 4]{kim2024}. Yet, to consider representation merely as a structured mechanism that delivers political presence would be misleading. Instead of a predefined relationship between representatives and the represented, what takes shape is a dynamic field of discursive interactions where legitimacy is constantly at stake. Representation, in this view, unfolds through a process of iterative positioning, where claims to speak for others are subject to ongoing scrutiny, negotiation, and at times, outright refusal. The act of representation is thus neither inherently stable nor unidirectional; rather, it emerges through a relational and contested interplay between those who articulate claims and those who encounter, accept, or reject them \parencite[see 6]{kim2024}.

At the end Saward's work releases the performative practice of claim making out
of the institutionalised settings \parencite[4]{kim2024}.

\hlred{We are talking about the people's engagement with the referent
	associated with them, how they accept or reject it and how they shape it.}

% section Where is exactly the construction in the construtivist turn? (end)

\section{Niche Bodies: What forms a constituency for the niche parties}\label{sec:Niche Bodies} % (fold) c
%\begin{quote}
%	Thirdly, aside from the enactment of horizontality/equality, the assembly of bodies has another even more fundamentally performative function, for what we are seeing when bodies assemble on the street, in the square, or in other public venues is the exercise – one might call it performative – of the right to appear, a bodily demand for a more liveable set of lives.20 This ‘right to appear’ is worth highlighting insofar as the right to appear enacts the right to be recognized on one’s own terms. The latter can play a role for movement activists in political parties and can hence be an important element of horizontal politics. The right to appear interlinks with a fourth and final aspect that we would like to highlight in Butler’s work: the question of the intelligibility of a subject. In order to appear in a meaningful way, the performed subjectivity must appear in an intelligible manner – and if it does not, it will be in a precarious position. Butler makes this argument, particularly when she points to a subject’s gender identifiability as the very precondition for recognizing it as a living being:
%\end{quote}{\cite[7]{kim2024}}
%
%%TODO: First tell about the background of nonnemacher
%%TODO: IS RIGHT TO APPEAR RELEVANT? (kim2024)
%%TODO:: Define representational deprivation
%%WARNING: TAKE CARE, ThIS is YOUR MAIN ARGUMENT
%%TODO: Does not feel natural
%Nonnemacher \parencite*{nonnemacher2023} argues that niche party supporters engage in political protest due to \textit{representational deprivation}. His main argument about the niche parties formed around the assumption that the niche party support is caused by a \textit{build-up} of representational deprivation leading to political protests \parencite[see 30]{nonnemacher2023}.
%His framework suggests that when niche parties enter government but fail to significantly influence policy outcomes, their supporters become disillusioned and turn to extra-institutional forms of participation, such as protests \parencite{nonnemacher2023}. This perspective aligns with theories of grievance-based mobilization, where political dissatisfaction translates into increased protest engagement.
%
%However, this argument assumes that niche party support is primarily driven by a \textit{reactive} protest orientation, neglecting the possibility that these voters are not merely responding to deprivation but are \textit{actively seeking} alternative modes of representation. Is the niche vote simply a protest vote to the mainstream parties (see \cite{hong2015, nonnemacher2023, stiers2024}), or is there a deeper dissatisfaction or search for a different alternative in terms of political representation?
%\hlred{Stier's \parencite*{stiers2024} is mainly focusing on the protest
%against the mainstream parties while disregarding it could be a rational
%choice \parencite[2]{stiers2024}.
%\bftext{I argue that the key issue is not just the failure of traditional democratic structures nor a simple protest to the mainstream parties, but the deeper search for new representative forms which niche parties either intentionally or unintentionally capitalise on.}
%
%
%
%which aligns with Saward’s concept of \textit{the Square} \parencite{saward2024}.
%Nonnemacher has the right direction while identifying the tniche voters with
%their dissatisfaction with how they are represented. But have a different
%conclusion, they claim protests are a result of the dissatisfaction.
%
%The Square represents spaces where representative claims are not just asserted from above but dynamically negotiated in co-presence with constituents. Niche parties, rather than simply channeling protest, actively create and engage with such alternative spaces, where political legitimacy is performed rather than delegated. In this view, niche parties do not merely exploit dissatisfaction with mainstream politics; they provide sites where representation is reconstituted through participation.
%
%Thus, while Nonnemacher’s framework helps explain why niche party supporters are more prone to protest, I argue that their behavior is better understood through the lens of performative representation. Rather than viewing niche party support as a symptom of democratic dissatisfaction, it should be analyzed as part of a broader transformation in how political representation is enacted. This perspective shifts the focus from deprivation to agency, highlighting how niche party supporters are not just reacting to exclusion but actively shaping new political spaces that challenge traditional representative structures.
% section Niche Bodies (end)
%
% chapter Main Chapter (end)
\chapter{Case Study or Empirical Work}\label{chap:Case Study or Empirical Work} % (fold)

% chapter Case Study or Empirical Work (end)

\begin{tcolorbox}
	Bloco de Esquerda (BE) in Portugal provides a concrete example of how niche parties operate within and beyond traditional representative institutions. Unlike mainstream parties, which rely on institutional legitimacy, BE blends electoral participation with grassroots activism, continuously fostering alternative representative spaces. Its history of engagement in anti-austerity protests and advocacy for participatory democracy suggests that its supporters are not just deprived of representation but are invested in redefining what representation means.

	BE’s dual engagement in both electoral and extra-institutional politics illustrates how niche parties can function as sites of political experimentation, where alternative models of representation are enacted. Studying BE allows us to examine how niche parties not only reflect dissatisfaction with existing structures but also offer new ways to think about democratic engagement outside conventional institutions.
\end{tcolorbox}

However, there is a need to explain why the effect
\citeauthoryear[see 31]{nonnemaccher2023} is bringing up against, for example,
\citeauthoryear{torcal2016}, namely that the niche party supporters are
indeed engaging in more protests when their party joins a coalition.
\citeauthoryear{Nonnemacher2023}'s argumentation follows the lines that since
the niche parties are often (read always so far) involved as junior partners in coalitions, amount
of the compromises they have to give into leads their supporters to conclude
their representatives are failing \parencite[see 31]{nonnemacher2023}. Since
the compromisses often involve not being able to deliver policy goals set by
the niche parties, the situation is also likely to be perceived more or less as
treason to the policy-oriented voter, especially when a niche party is in
coalition with a mainstream party far away from its political positioning
whereas the compromises are greater
\parencite[see
	32]{nonnemacher2023}.


%TODO:Argument it whereas the compromises are greater
\chapter{quotes}\label{chap:quotes} % (fold)

% chapter quotes (end)
\section{saward}\label{sec:saward} % (fold)

% section saward (end)
\begin{quote}
	In contexts of physical co-presence of would-be representers and would-be represented, the fact of real-world, real-time presence raises the political stakes of representation – who controls the performance and its outcome, who is empowered or disempowered, placed or displaced?
\end{quote}
-- \cite[3]{saward2024}


\begin{quote}
	The bottom line is this: ‘Squares’ for the purposes of this chapter might literally be square, or they might not; it is the physical co-presence of would-be representers and represented that is the focus, and not the exact configuration of the space of that co-presence (which may well change shape in or through a specific process of interaction).
\end{quote}
-- \cite[5]{saward2024}

\begin{quote}
	The nature and consequences of performing representation in the Square are distinctive from more familiar contexts where there is not a physical co-presence between representers and represented. It is a space where you cannot speak about or for without also speaking to. Physical co-presence heightens the sense of contingency and uncertainty (how will these people react to each other, on the spot and in real time?). [...] \textbf{the Squaolre is a space of ‘mutual vulnerability’}.
\end{quote}
-- \cite[5]{saward2024}

\begin{quote}
	Two generalised possibilities for power dynamics in the site are (A) where structures, terms of entry, etc, advantage or privilege formal representative claim-makers, discourage or disable counterclaims by audience members, and set up a sharp division between who can act and who cannot19; and (B) where the opposite is the case – differently positioned actors can move and speak in a more fluid structure, and so on. Case B holds out the promise of being characterised by a greater sense of equality – not as sameness, but as respect and recognition across difference.
\end{quote}
-- \cite[13]{saward2024}




\chapter{main}\label{chap:main} % (fold) n

As Saward notes

\begin{quote}
	Political representation in contexts where there is not physical co-presence – which is most of the time – involves complex power relations between representers and represented. In general terms, representation in such contexts means that claimants to representation do not have the prospect of combined immediate plus face-to-face feedback
\end{quote}

what is lacking in political presentation is establishing a productive feedback
cycle than responsiveness. That means establishing the framework to make the
constitutiencies able to make bodies \sidenote{As much as Saward's Square doesn't have to relate to any of the meanings of the word, the body here in its critical theory related context used liberatively, both as a body and as a substance of an individual.}  of themselves and not a huge chunk of luqified vote mass.

% chapter main (end)
\printbibliography
\end{document}

































