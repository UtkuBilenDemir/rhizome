%%%%%%%%%%%%%%%%%%%%%%%%%%%% Define Article %%%%%%%%%%%%%%%%%%%%%%%%%%%%%%%%%%
\documentclass[nobib, openany, justified, a4paper, 14pt]{tufte-book}
%%%%%%%%%%%%%%%%%%%%%%%%%%%%%%%%%%%%%%%%%%%%%%%%%%%%%%%%%%%%%%%%%%%%%%%%%%%%%%%

%%%%%%%%%%%%%%%%%%%%%%%%%%%%% Citations %%%%%%%%%%%%%%%%%%%%%%%%%%%%%%%%%%%%%%%
%\usepackage[utf8]{inputenc}
\usepackage[style=authoryear-icomp]{biblatex}
%\usepackage[style=apa]{biblatex}
\addbibresource{/Users/ubd/Bibliotheca/bib.bib}
%%%%%%%%%%%%%%%%%%%%%%%%%%%%%%%%%%%%%%%%%%%%%%%%%%%%%%%%%%%%%%%%%%%%%%%%%%%%%%%

%%%%%%%%%%%%%%%%%%%%%%%%%%%%% Using Packages %%%%%%%%%%%%%%%%%%%%%%%%%%%%%%%%%%
\usepackage{newunicodechar}
\newunicodechar{🦜}{[parrot]}
\PassOptionsToPackage{prologue,dvipsnames}{xcolor}
\sloppy  % globally
\usepackage{geometry}
\usepackage{graphicx}
\usepackage{amssymb}
\usepackage{amsmath}
\usepackage{amsthm}
\usepackage{empheq}
\usepackage{mdframed}
\usepackage{booktabs}
\usepackage{lipsum}
\usepackage{graphicx}
\usepackage{color}
\usepackage{psfrag}
\usepackage{pgfplots}
\usepackage{bm}
\usepackage{epigraph}
\usepackage{titlesec}
\usepackage{tcolorbox}
\usepackage{csquotes}
\usepackage{pifont}
\usepackage{enumitem,amssymb}
% \usepackage{spoton} % adds \todo functionality I hope
%%%%%%%%%%%%%%%%%%%%%%%%%%%%%%%%%%%%%%%%%%%%%%%%%%%%%%%%%%%%%%%%%%%%%%%%%%%%%%%

% Other Settings

%%%%%%%%%%%%%%%%%%%%%%%%%% Page Setting %%%%%%%%%%%%%%%%%%%%%%%%%%%%%%%%%%%%%%%

%%%%%%%%%%%%%%%%%%%%%%%%%% Define some useful colors %%%%%%%%%%%%%%%%%%%%%%%%%%
\definecolor{maroon}{RGB}{128,0,0} %hlred
\definecolor{MAROON}{RGB}{128,0,0} %hlred
\definecolor{deepBlue}{RGB}{61,124,222} %url-links
\definecolor{deepGreen}{RGB}{26,111,0} %citations
\definecolor{ocre}{RGB}{243,102,25}
\definecolor{mygray}{RGB}{243,243,244}
\definecolor{shallowGreen}{RGB}{235,255,255}
\definecolor{shallowBlue}{RGB}{235,249,255}
\definecolor{mediumpersianBlue}{rgb}{0.0, 0.4, 0.65}
\definecolor{persianBlue}{rgb}{0.11, 0.22, 0.73}
\definecolor{persianGreen}{rgb}{0.0, 0.65, 0.58}
\definecolor{persianRed}{rgb}{0.8, 0.2, 0.2}
\definecolor{debianRed}{rgb}{0.84, 0.04, 0.33}
%%%%%%%%%%%%%%%%%%%%%%%%%%%%%%%%%%%%%%%%%%%%%%%%%%%%%%%%%%%%%%%%%%%%%%%%%%%%%%%

%%%%%%%%%%%%%%%%%%%%%%%%%% Indentation Settings %%%%%%%%%%%%%%%%%%%%%%%%%%%%%%%
\makeatletter
% Paragraph indentation and separation for normal text
\renewcommand{\@tufte@reset@par}{%
	\setlength{\RaggedRightParindent}{0pc}%1.0
	\setlength{\JustifyingParindent}{0pc}%1.0
	\setlength{\parindent}{1pc}%1pc
	\setlength{\parskip}{5pt}%0pt
}
\@tufte@reset@par

% Paragraph indentation and separation for marginal text
\renewcommand{\@tufte@margin@par}{%
	\setlength{\RaggedRightParindent}{0pc}%0.5pc
	\setlength{\JustifyingParindent}{0pc}%0.5pc
	\setlength{\parindent}{0.5pc}%
	\setlength{\parskip}{5pt}%0pt
}
\makeatother



%%%%%%%%%%%%%%%%%%%%%%%%%% Define an orangebox command %%%%%%%%%%%%%%%%%%%%%%%%
%o\usepackage[most]{tcolorbox}

\newtcolorbox{orangebox}{
	colframe=ocre,
	colback=mygray,
	boxrule=0.8pt,
	arc=0pt,
	left=2pt,
	right=2pt,
	width=\linewidth,
	boxsep=4pt
}


\newtcolorbox{redbox}{
	colframe=red,
	boxrule=0.8pt,
	arc=0pt,
	left=2pt,
	right=2pt,
	width=\linewidth,
	boxsep=4pt
}
%%%%%%%%%%%%%%%%%%%%%%%%%%%%%%%%%%%%%%%%%%%%%%%%%%%%%%%%%%%%%%%%%%%%%%%%%%%%%%%

%%%%%%%%%%%%%%%%%%%%%%%%%%%% English Environments %%%%%%%%%%%%%%%%%%%%%%%%%%%%%
\newtheoremstyle{mytheoremstyle}{3pt}{3pt}{\normalfont}{0cm}{\rmfamily\bfseries}{}{1em}{{\color{black}\thmname{#1}~\thmnumber{#2}}\thmnote{\,--\,#3}}
\newtheoremstyle{myproblemstyle}{3pt}{3pt}{\normalfont}{0cm}{\rmfamily\bfseries}{}{1em}{{\color{black}\thmname{#1}~\thmnumber{#2}}\thmnote{\,--\,#3}}
\theoremstyle{mytheoremstyle}
\newmdtheoremenv[linewidth=1pt,backgroundcolor=shallowGreen,linecolor=deepGreen,leftmargin=0pt,innerleftmargin=20pt,innerrightmargin=20pt,]{theorem}{Theorem}[section]
\theoremstyle{mytheoremstyle}
\newmdtheoremenv[linewidth=1pt,backgroundcolor=shallowBlue,linecolor=deepBlue,leftmargin=0pt,innerleftmargin=20pt,innerrightmargin=20pt,]{definition}{Definition}[section]
\theoremstyle{myproblemstyle}
\newmdtheoremenv[linecolor=black,leftmargin=0pt,innerleftmargin=10pt,innerrightmargin=10pt,]{problem}{Problem}[section]
%%%%%%%%%%%%%%%%%%%%%%%%%%%%%%%%%%%%%%%%%%%%%%%%%%%%%%%%%%%%%%%%%%%%%%%%%%%%%%%

%%%%%%%%%%%%%%%%%%%%%%%%%%%%%%% Plotting Settings %%%%%%%%%%%%%%%%%%%%%%%%%%%%%
\usepgfplotslibrary{colorbrewer}
\pgfplotsset{width=8cm,compat=1.9}
%%%%%%%%%%%%%%%%%%%%%%%%%%%%%%%%%%%%%%%%%%%%%%%%%%%%%%%%%%%%%%%%%%%%%%%%%%%%%%%

%%%%%%%%%%%%%%%%%%%%%%%%%%%%%%% MISC %%%%%%%%%%%%%%%%%%%%%%%%%%%%%%%%%%%%%%%%%%
\usepackage[acronym]{glossaries}
\usepackage{hyperref} % Setup: https://www.overleaf.com/learn/latex/Hyperlinks
\hypersetup{
	colorlinks=true,
	%citecolor=deepGreen,
	citecolor=maroon,
	linkcolor=persianBlue,
	filecolor=persianGreen,
	urlcolor=persianBlue,
	pdfpagemode=FullScreen,
}

%%%%%%%%%%%%%%%%%%%%%%%%%%%%%%%%%%%%%%%%%%%%%%%%%%%%%%%%%%%%%%%%%%%%%%%%%%%%%%%
\setcounter{tocdepth}{2}
\setcounter{secnumdepth}{2}

\newcommand{\hlred}[1]{\textcolor{Maroon}{#1}} % Print text in maroon
\newcommand{\hlgreen}[1]{\textcolor{persianGreen}{#1}} % Print text in green
\newcommand{\hlocre}[1]{\textcolor{ocre}{#1}} % Print text in green

\newenvironment{greenenv}{\color{Green}}{\ignorespacesafterend}  % Create green environment
\newenvironment{commentenv}{\color{ocre}}{\ignorespacesafterend}  % Create comment environment


\titleformat{\part}[display]
{\filleft\fontsize{40}{40}\selectfont\scshape}
{\fontsize{90}{90}\selectfont\thepart}
{20pt}
{\thispagestyle{epigraph}}

\setlength\epigraphwidth{.6\textwidth}

%\makeatletter
%\epigraphhead
%{\let\@evenfoot}
%{\let\@oddfoot\@empty\let\@evenfoot}
%{}{}
%\makeatother


%%%%%%%%%%%%%%%%%%%%%%%%%%%%%%%%%%%%%%%%%%%%%%%%%%%%%%%%%%%%%%%%%%%%%%%%%%%%%%%
%TODO LIST
\newlist{todolist}{itemize}{2}
\setlist[todolist]{label=$\square$}
\newcommand{\cmark}{\ding{51}}%
\newcommand{\xmark}{\ding{55}}%
\newcommand{\done}{\rlap{$\square$}{\raisebox{2pt}{\large\hspace{1pt}\cmark}}%
	\hspace{-2.5pt}}
\newcommand{\wontfix}{\rlap{$\square$}{\large\hspace{1pt}\xmark}}

%%%%%%%%%%%%%%%%%%%%%%%%%%%%%%%%%%%%%%%%%%%%%%%%%%%%%%%%%%%%%%%%%%%%%%%%%%%%%%%
\newcommand{\greensquare}{\marginnote{\fcolorbox{green}{green}{\rule{0pt}{3mm}\rule{3mm}{0pt}}\quad}}
\newcommand{\yellowsquare}{\marginnote{\fcolorbox{yellow}{yellow}{\rule{0pt}{3mm}\rule{3mm}{0pt}}\quad}}
\newcommand{\redsquare}{\marginnote{\fcolorbox{red}{red}{\rule{0pt}{3mm}\rule{3mm}{0pt}}\quad}}




\usepackage{tcolorbox}

%%%%%%%%%%%%%%%%%%%%%%%%%%%%%%% Title & Author %%%%%%%%%%%%%%%%%%%%%%%%%%%%%%%%

\title{From Stages to Squares: Representation by Niche Parties and Political Presence (The Role of Performativity)}

\author{Utku B. Demir}

%%%%%%%%%%%%%%%%%%%%%%%%%%%%%%%%%%%%%%%%%%%%%%%%%%%%%%%%%%%%%%%%%%%%%%%%%%%%%%%%

\begin{document}

\maketitle

\chapter{Introduction}\label{chap:Introduction}

\epigraph{The opposite of representation is not participation}{\cite{plotke1997}}

\chapter{Theoretical Framework}\label{chap:Theoretical-Framework}

\section{Literature Analysis and Current Debates}\label{sec:Literature-Analysis}

\textcite{blumenau2024}

I argue that \textcite{nonnemacher2023}'s claim that niche party voters take to the streets once they realize their party's influence is negligible remains only partially valid. Niche party voters may also recognize that even after their party gains governmental positions, fundamental attributes of electoral democracy continue limiting the effectiveness of square-based political action.

Niche party voters cannot be reduced to policy-driven extremists or disaffected protestors. \hlred{New Claim:} I contest Nonnemacher's core assumption by arguing niche party support stems more fundamentally from supporters' engagement with alternative modes of representation rather than protest orientation. This engagement emerges through perceived closer connections with niche party representatives, reflecting Saward's concept of the \textit{Square} rather than mere involvement in protest activities.

\section{Methodology and Research Question}\label{sec:Methodology}

\chapter{Main Analysis}\label{chap:Main-Analysis}

The definition of niche parties remains contested. While scholars demonstrate variability in party selection criteria \parencite[compare][]{adams2006,nonnemacher2023}, most agree niche parties emphasize specific non-economic issues neglected by mainstream competitors \parencite{nonnemacher2023,stiers2024}. Like their supporters, these issues occupy marginalized positions within political agendas.

Recent decades witnessed significant voter migration from mainstream to niche parties n one’s own terms \parencite{kim2024}.

\textcite{nonnemacher2023} attributes niche party supporters' protest participation to \textit{representational deprivation}, positing governmental participation reveals parties' limited influence, triggering disillusionment \parencite{nonnemacher2023}. While aligning with grievance mobilization theories, this framework underplays supporters' active pursuit of alternative representation modes.

My analysis counters that niche support reflects not merely reactive protest but active engagement with reconstituted representative practices. Supporters seek not policy influence alone but the right to political presence through what \textcite{saward2024} terms the Square's mutual vulnerability. Niche parties' hybrid instituansaction to participatory performance.

Traditional representation's physical separation between representatives and constituents \parencite{saward2024} inhibits the feedback loops essential for responsive democracy. Niche parties' strength lies in creating Squares where constituencies materialize as political bodies rather than abstract vote masses. This performative reconstitution of representation challenges institutional norms while expanding democratic possibilities.


\end{document}
trols the performance and its outcome, who is empowered or disempowered, placed or displaced? \parencite{saward2024}
\end{quote}

\begin{quote}
	The bottom line is this: ‘Squares’ [...] might literally be square, or they might not; it is the physical co-presence [...] that is the focus \parencite{saward2024}
\end{quote}

\begin{quote}
	The Square is a space of ‘mutual vulnerability’ \parencite{saward2024}
\end{quote}

\chapter{Conclusion}\label{chap:Conclusion}

\textcite{saward2024} identifies two Square power dynamics: (A) hierarchical structures privileging formal representatives, and (B) fluid spaces enabling egalitarian participation. Niche parties gravitate toward Scenario B through co-present claim-making that transforms representation from electoral trractices.
\end{tcolorbox}

\textcite{nonnemacher2023} challenges \textcite{torcal2016} by showing niche supporters increase protest participation when their parties join coalitions. As junior partners, niche parties make greater policy compromises, particularly when aligning with ideologically distant mainstream parties \parencite{nonnemacher2023}. These compromises generate perceptions of representational betrayal among policy-oriented voters.

\chapter{Conceptual Foundations}\label{chap:Conceptual}

\section{Saward's Key Propositions}\label{sec:Saward-Quotes}

\begin{quote}
	In contexts of physical co-presence of would-be representers and would-be represented, the fact of real-world, real-time presence raises the political stakes of representation – who con de Esquerda
\end{quote}\label{chap:Case-Study}

\begin{tcolorbox}
Portugal's Bloco de Esquerda (BE) exemplifies niche parties operating across institutional boundaries. Unlike mainstream competitors relying on institutional legitimacy, BE hybridizes electoral politics with grassroots activism, continuously generating alternative representative spaces. Its anti-austerity protest leadership and participatory democracy advocacy demonstrate supporters' investment in redefining representation rather than merely expressing deprivation.

BE's dual engagement illustrates niche parties as political laboratories where representation models get renegotiated. This case reveals how niche parties transcend protest channels to reconfigure democratic engagement through experimental ptional/extra-institutional engagement facilitates this presence.

\hlred{Crucial Argument:} The core issue involves not traditional democratic structures' failure nor simple anti-mainstream protest, but the search for new representative forms that niche parties materialize through performative practices aligning with Saward's Square concept.

\textcite{stiers2024} overemphasizes protest motivation while neglecting rational engagement with alternative representation. Nonnemacher correctly identifies representational dissatisfaction but misattributes outcomes to deprivation rather than agency. The Square concept reveals how niche parties create spaces for legitimacy performance through participation rather than delegation.

\chapter{Case Study: Bloco}
\parencite{disch2015} reveals representation's multidirectional nature, but operationalizing this insight requires examining how represented subjects engage representative claims.

\textcite{kim2024} poses the critical question: \textit{Is a representative claim accepted and reproduced by those addressed?} Saward's Square concept illuminates productive interactions between claim-makers and audiences through physical co-presence. Though abstract, the Square metaphor encompasses concrete spaces like party gatherings where bodily interactions reshape political dynamics \parencite{kim2024}.

Representation constitutes an unstable discursive field where legitimacy remains perpetually contested through iterative positioning. This process involves continuous negotiation between claim articulation and audience reception \parencite{kim2024}. Niche parties' extra-institutional claim-making \parencite{kim2024} facilitates supporter engagement with referents through acceptance, rejection, and reshaping of representative claims.

\section{Niche Constituencies}\label{sec:Niche-Constituencies}

\begin{quote}
Thirdly, aside from the enactment of horizontality/equality, the assembly of bodies has another even more fundamentally performative function, for what we are seeing when bodies assemble on the street, in the square, or in other public venues is the exercise – one might call it performative – of the right to appear, a bodily demand for a more liveable set of lives. This ‘right to appear’ enacts the right to be recognized os2024,\parencite{spoon2019}. This realignment reflects not merely protest voting but deeper transformations in political representation.

\section{Saward's Square}\label{sec:Sawards-Square}

\epigraph{There should be no presumption that a Square is a democratic or democratising space, or a hierarchical or non-hierarchical space. Squares may be devised, created or generated for purposes of empowered inclusion or marginalization}{\cite{saward2024}}

Representative claims gain potency when developed through co-presence with constituencies rather than hierarchical delegation.

\section{Constructivism in Performative Claims}\label{sec:Constructivism}

\epigraph{[\ldots] representation is not something external to its performance}{\cite{saward2010}} The constructivist turn \
