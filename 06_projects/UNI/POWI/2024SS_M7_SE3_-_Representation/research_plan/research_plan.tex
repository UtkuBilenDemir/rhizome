%%%%%%%%%%%%%%%%%%%%%%%%%%%% Define Article %%%%%%%%%%%%%%%%%%%%%%%%%%%%%%%%%%
\documentclass[nobib, openany, justified, a4paper, 14pt]{tufte-book}
%%%%%%%%%%%%%%%%%%%%%%%%%%%%%%%%%%%%%%%%%%%%%%%%%%%%%%%%%%%%%%%%%%%%%%%%%%%%%%%

%%%%%%%%%%%%%%%%%%%%%%%%%%%%% Citations %%%%%%%%%%%%%%%%%%%%%%%%%%%%%%%%%%%%%%%
%\usepackage[utf8]{inputenc}
\usepackage[style=authoryear-icomp]{biblatex}
%\usepackage[style=apa]{biblatex}
\addbibresource{/Users/ubd/Bibliotheca/bib.bib}
%%%%%%%%%%%%%%%%%%%%%%%%%%%%%%%%%%%%%%%%%%%%%%%%%%%%%%%%%%%%%%%%%%%%%%%%%%%%%%%

%%%%%%%%%%%%%%%%%%%%%%%%%%%%% Using Packages %%%%%%%%%%%%%%%%%%%%%%%%%%%%%%%%%%
\usepackage{newunicodechar}
\newunicodechar{🦜}{[parrot]}
\PassOptionsToPackage{prologue,dvipsnames}{xcolor}
\sloppy  % globally
\usepackage{geometry}
\usepackage{graphicx}
\usepackage{amssymb}
\usepackage{amsmath}
\usepackage{amsthm}
\usepackage{empheq}
\usepackage{mdframed}
\usepackage{booktabs}
\usepackage{lipsum}
\usepackage{graphicx}
\usepackage{color}
\usepackage{psfrag}
\usepackage{pgfplots}
\usepackage{bm}
\usepackage{epigraph}
\usepackage{titlesec}
\usepackage{tcolorbox}
\usepackage{csquotes}
\usepackage{pifont}
\usepackage{enumitem,amssymb}
% \usepackage{spoton} % adds \todo functionality I hope
%%%%%%%%%%%%%%%%%%%%%%%%%%%%%%%%%%%%%%%%%%%%%%%%%%%%%%%%%%%%%%%%%%%%%%%%%%%%%%%

% Other Settings

%%%%%%%%%%%%%%%%%%%%%%%%%% Page Setting %%%%%%%%%%%%%%%%%%%%%%%%%%%%%%%%%%%%%%%

%%%%%%%%%%%%%%%%%%%%%%%%%% Define some useful colors %%%%%%%%%%%%%%%%%%%%%%%%%%
\definecolor{maroon}{RGB}{128,0,0} %hlred
\definecolor{MAROON}{RGB}{128,0,0} %hlred
\definecolor{deepBlue}{RGB}{61,124,222} %url-links
\definecolor{deepGreen}{RGB}{26,111,0} %citations
\definecolor{ocre}{RGB}{243,102,25}
\definecolor{mygray}{RGB}{243,243,244}
\definecolor{shallowGreen}{RGB}{235,255,255}
\definecolor{shallowBlue}{RGB}{235,249,255}
\definecolor{mediumpersianBlue}{rgb}{0.0, 0.4, 0.65}
\definecolor{persianBlue}{rgb}{0.11, 0.22, 0.73}
\definecolor{persianGreen}{rgb}{0.0, 0.65, 0.58}
\definecolor{persianRed}{rgb}{0.8, 0.2, 0.2}
\definecolor{debianRed}{rgb}{0.84, 0.04, 0.33}
%%%%%%%%%%%%%%%%%%%%%%%%%%%%%%%%%%%%%%%%%%%%%%%%%%%%%%%%%%%%%%%%%%%%%%%%%%%%%%%

%%%%%%%%%%%%%%%%%%%%%%%%%% Indentation Settings %%%%%%%%%%%%%%%%%%%%%%%%%%%%%%%
\makeatletter
% Paragraph indentation and separation for normal text
\renewcommand{\@tufte@reset@par}{%
	\setlength{\RaggedRightParindent}{0pc}%1.0
	\setlength{\JustifyingParindent}{0pc}%1.0
	\setlength{\parindent}{1pc}%1pc
	\setlength{\parskip}{5pt}%0pt
}
\@tufte@reset@par

% Paragraph indentation and separation for marginal text
\renewcommand{\@tufte@margin@par}{%
	\setlength{\RaggedRightParindent}{0pc}%0.5pc
	\setlength{\JustifyingParindent}{0pc}%0.5pc
	\setlength{\parindent}{0.5pc}%
	\setlength{\parskip}{5pt}%0pt
}
\makeatother



%%%%%%%%%%%%%%%%%%%%%%%%%% Define an orangebox command %%%%%%%%%%%%%%%%%%%%%%%%
%o\usepackage[most]{tcolorbox}

\newtcolorbox{orangebox}{
	colframe=ocre,
	colback=mygray,
	boxrule=0.8pt,
	arc=0pt,
	left=2pt,
	right=2pt,
	width=\linewidth,
	boxsep=4pt
}


\newtcolorbox{redbox}{
	colframe=red,
	boxrule=0.8pt,
	arc=0pt,
	left=2pt,
	right=2pt,
	width=\linewidth,
	boxsep=4pt
}
%%%%%%%%%%%%%%%%%%%%%%%%%%%%%%%%%%%%%%%%%%%%%%%%%%%%%%%%%%%%%%%%%%%%%%%%%%%%%%%

%%%%%%%%%%%%%%%%%%%%%%%%%%%% English Environments %%%%%%%%%%%%%%%%%%%%%%%%%%%%%
\newtheoremstyle{mytheoremstyle}{3pt}{3pt}{\normalfont}{0cm}{\rmfamily\bfseries}{}{1em}{{\color{black}\thmname{#1}~\thmnumber{#2}}\thmnote{\,--\,#3}}
\newtheoremstyle{myproblemstyle}{3pt}{3pt}{\normalfont}{0cm}{\rmfamily\bfseries}{}{1em}{{\color{black}\thmname{#1}~\thmnumber{#2}}\thmnote{\,--\,#3}}
\theoremstyle{mytheoremstyle}
\newmdtheoremenv[linewidth=1pt,backgroundcolor=shallowGreen,linecolor=deepGreen,leftmargin=0pt,innerleftmargin=20pt,innerrightmargin=20pt,]{theorem}{Theorem}[section]
\theoremstyle{mytheoremstyle}
\newmdtheoremenv[linewidth=1pt,backgroundcolor=shallowBlue,linecolor=deepBlue,leftmargin=0pt,innerleftmargin=20pt,innerrightmargin=20pt,]{definition}{Definition}[section]
\theoremstyle{myproblemstyle}
\newmdtheoremenv[linecolor=black,leftmargin=0pt,innerleftmargin=10pt,innerrightmargin=10pt,]{problem}{Problem}[section]
%%%%%%%%%%%%%%%%%%%%%%%%%%%%%%%%%%%%%%%%%%%%%%%%%%%%%%%%%%%%%%%%%%%%%%%%%%%%%%%

%%%%%%%%%%%%%%%%%%%%%%%%%%%%%%% Plotting Settings %%%%%%%%%%%%%%%%%%%%%%%%%%%%%
\usepgfplotslibrary{colorbrewer}
\pgfplotsset{width=8cm,compat=1.9}
%%%%%%%%%%%%%%%%%%%%%%%%%%%%%%%%%%%%%%%%%%%%%%%%%%%%%%%%%%%%%%%%%%%%%%%%%%%%%%%

%%%%%%%%%%%%%%%%%%%%%%%%%%%%%%% MISC %%%%%%%%%%%%%%%%%%%%%%%%%%%%%%%%%%%%%%%%%%
\usepackage[acronym]{glossaries}
\usepackage{hyperref} % Setup: https://www.overleaf.com/learn/latex/Hyperlinks
\hypersetup{
	colorlinks=true,
	%citecolor=deepGreen,
	citecolor=maroon,
	linkcolor=persianBlue,
	filecolor=persianGreen,
	urlcolor=persianBlue,
	pdfpagemode=FullScreen,
}

%%%%%%%%%%%%%%%%%%%%%%%%%%%%%%%%%%%%%%%%%%%%%%%%%%%%%%%%%%%%%%%%%%%%%%%%%%%%%%%
\setcounter{tocdepth}{2}
\setcounter{secnumdepth}{2}

\newcommand{\hlred}[1]{\textcolor{Maroon}{#1}} % Print text in maroon
\newcommand{\hlgreen}[1]{\textcolor{persianGreen}{#1}} % Print text in green
\newcommand{\hlocre}[1]{\textcolor{ocre}{#1}} % Print text in green

\newenvironment{greenenv}{\color{Green}}{\ignorespacesafterend}  % Create green environment
\newenvironment{commentenv}{\color{ocre}}{\ignorespacesafterend}  % Create comment environment


\titleformat{\part}[display]
{\filleft\fontsize{40}{40}\selectfont\scshape}
{\fontsize{90}{90}\selectfont\thepart}
{20pt}
{\thispagestyle{epigraph}}

\setlength\epigraphwidth{.6\textwidth}

%\makeatletter
%\epigraphhead
%{\let\@evenfoot}
%{\let\@oddfoot\@empty\let\@evenfoot}
%{}{}
%\makeatother


%%%%%%%%%%%%%%%%%%%%%%%%%%%%%%%%%%%%%%%%%%%%%%%%%%%%%%%%%%%%%%%%%%%%%%%%%%%%%%%
%TODO LIST
\newlist{todolist}{itemize}{2}
\setlist[todolist]{label=$\square$}
\newcommand{\cmark}{\ding{51}}%
\newcommand{\xmark}{\ding{55}}%
\newcommand{\done}{\rlap{$\square$}{\raisebox{2pt}{\large\hspace{1pt}\cmark}}%
	\hspace{-2.5pt}}
\newcommand{\wontfix}{\rlap{$\square$}{\large\hspace{1pt}\xmark}}

%%%%%%%%%%%%%%%%%%%%%%%%%%%%%%%%%%%%%%%%%%%%%%%%%%%%%%%%%%%%%%%%%%%%%%%%%%%%%%%
\newcommand{\greensquare}{\marginnote{\fcolorbox{green}{green}{\rule{0pt}{3mm}\rule{3mm}{0pt}}\quad}}
\newcommand{\yellowsquare}{\marginnote{\fcolorbox{yellow}{yellow}{\rule{0pt}{3mm}\rule{3mm}{0pt}}\quad}}
\newcommand{\redsquare}{\marginnote{\fcolorbox{red}{red}{\rule{0pt}{3mm}\rule{3mm}{0pt}}\quad}}




\makeatletter
\renewcommand{\maketitlepage}{%
  \begingroup%
  \thispagestyle{empty}
  \setlength{\parindent}{0pt}

  {\fontsize{12}{12}\selectfont{\@author}\par}

  \vspace{2cm}
  {\fontsize{14}{14}\selectfont\textbf\@title\par}

  \vspace{3cm}

  \endgroup
}
\makeatother

%%%%%%%%%%%%%%%%%%%%%%%%%%%%%%% Title & Author %%%%%%%%%%%%%%%%%%%%%%%%%%%%%%%%
\title{Research Plan: Changing Preferences for Representation and the Support for Niche Parties; A Comparative Analysis Across EU Countries}
\author{Utku B. Demir - 1033 Words}
%%%%%%%%%%%%%%%%%%%%%%%%%%%%%%%%%%%%%%%%%%%%%%%%%%%%%%%%%%%%%%%%%%%%%%%%%%%%%%%%


\begin{document}
\maketitle

\textbf{RQ:} How does the change in voters’ representative preferences and dissatisfaction with the electoral process relate to the support for niche parties in comparison with the salience of the issues they are characterised with?

Often, the support for niche parties is explained by their specific and less flexible \parencite[525]{adams2006} ideological tendencies and the issues they prioritize. The research question aims to examine whether and how changing preferences regarding how citizens wish to be represented play a more prominent role in increasing or decreasing support for these parties.

% section Research Question (end)

\section{Existing Literature}\label{sec:Existing Literature} % (fold)
From their definition to their nature, studying niche parties is multifaceted, often lacking standardized approaches. \textcite{adams2006} has played a prominent role in identifying and characterizing niche parties. Their work emphasizes niche parties’ less flexible structures, more stable and well-defined supporter bases, and the comparatively harsher punishment they face from their constituencies for ideological shifts compared to mainstream parties.

The dimension of adaptation was later explored by \textcite{bischof2020}, who examined the question: \textit{What makes parties adapt themselves to voter preferences?} While they partially confirm Adams et al.’s claim that policy-seeking parties—a term often used to describe niche parties \parencite[515]{adams2006}—are less likely to track changes among the median voter \parencite[9]{bischof2020}, they also show that niche parties are less likely to adjust their positions along the left-right ideological spectrum.

While these examinations primarily approach niche parties from the perspective of party dynamics, other studies shift the focus to voter preferences and electoral support. For example, \textcite{wolak2017, blumenau2024, costa2021} examine voters’ expectations and preferences regarding representation. Particularly, the work of \textcite{blumenau2024} inspires the inclusion of constituencies’ preferences regarding their representatives in this study. This allows for an analysis of how changes in these preferences may affect support for niche parties.

% section Existing Literature (end)

\section{Theory}\label{sec:Theory} % (fold)
I rely on \textcite{blumenau2024} to analyze voter preferences in relation to their support for representatives. This approach examines whether and how support for niche parties is influenced by changing voter preferences and general dissatisfaction with the perceived lack of representation by incumbents. Returning to Sabl's \parencite*[]{sabl2015} critique of a solely responsiveness-oriented approach while analysing representation in quantitative political science, and considering the constructivist understanding of representation (e.g., \textcite{saward2010} and \textcite{disch2015}), this study capitalizes on the potential benefit of analysing party support also from the perspective of constituency preferences regarding their political representation.

This theoretical framework leads to the testing of the following hypotheses:

\begin{itemize}
	\item [H1:] Changing representational preferences and expectations have a significant effect on the increasing (or decreasing) support for niche parties.
	\item [H2:] Dissatisfaction with incumbent representation over time has a significant effect on support for niche parties.
	\item [H3:] Changing representational preferences and expectations have a stronger effect on niche party support than the salience of the issues associated with these parties.
	\item [H3a:] The joint effect of changing representational preferences and dissatisfaction with incumbent representation has a stronger influence on niche party support compared to the salience of the issues associated with these parties.
	\item [H4:] There is a significant difference between the representational preferences of mainstream party voters and niche party voters.
\end{itemize}

% section Theory (end)

\section{Empirics}\label{sec:Empirics} % (fold)
This study relies on the foundational frameworks established by \textcite{adams2006} and \textcite{bischof2020} for the identification of niche parties. These frameworks emphasize the unique policy-seeking nature of niche parties, their ideological rigidity, and their distinct relationship with their voter bases. For the empirical analysis, three to four EU countries will be selected based on evidence of consistent increases in electoral support for at least one niche party over the past decade. Selection criteria will ensure variation in party type, electoral system, and political context to enable comparative insights.

\subsection*{Operationalization of Key Concepts}
\begin{itemize}
	\item \textbf{Voter Preferences:}
	      Changes in voters’ representational preferences will be analyzed using the six dimensions outlined by \textcite{blumenau2024}: substantive representation, descriptive representation, surrogation, personalization, justification, and responsiveness. Longitudinal opinion surveys, such as the European Social Survey (ESS) \parencite*{ess2023} and national election studies in the selected countries, will be utilized to capture shifts in these preferences over time.

	\item \textbf{Niche Party Characteristics:}
	      The identification of the characterizing issues of the niche parties will draw from the Manifesto Project dataset \parencite*[]{manifestoproject}, focusing on issue salience and policy emphasis in party platforms. Salience of these issues will be further validated using Eurobarometer surveys \parencite*{eurobarometer2023}, which provide longitudinal data on public concern for specific policy areas.

	\item \textbf{Dissatisfaction with Incumbents:}
	      Voter dissatisfaction will be measured using Eurobarometer \parencite*{eurobarometer2023} questions on trust in national governments, satisfaction with democracy, and perceived responsiveness of incumbents. Country-level variations will be explored to assess how dissatisfaction correlates with niche party support.

	\item \textbf{Electoral Support:}
	      Electoral trends will be analyzed using constituency-level election results. Sources such as ParlGov \parencite*{parlgov2019} and national electoral commissions will be consulted to gather data on changes in niche party vote shares.
\end{itemize}

\subsection*{Methodology}
The empirical analysis will adopt the following approaches:
\begin{itemize}
	\item \textbf{Descriptive Analysis:}
	      Track changes in representational preferences across the selected countries, focusing on variations between voters of mainstream and niche parties.

	\item \textbf{Multivariate Regression Models:}
	      Examine the relationship between changing representational preferences, dissatisfaction with incumbents, and niche party support. Test interactions between voter dissatisfaction and issue salience to determine joint effects on niche party support.

	\item \textbf{Difference-in-Differences (DiD):}
	      Analyze cases with electoral shocks to assess pre- and post-election dynamics in niche party support and voter preferences.

	\item \textbf{Case Studies:}
	      Provide in-depth qualitative analyses of specific niche parties in each country, focusing on campaigns, rhetoric, and issue salience.
\end{itemize}

\subsection*{Expected Outputs}
\begin{itemize}
	\item Identification of the dimensions of representation most strongly associated with niche party support.
	\item Quantitative estimates of the relative impact of dissatisfaction with incumbents versus issue salience on niche party voting behavior.
	\item Cross-country comparisons to highlight variations in voter behavior and party dynamics.
\end{itemize}

By combining survey data, manifesto analysis, and election results, the methodology aims to provide a comprehensive understanding of the factors driving niche party support and the changing dynamics of political representation.

% section Empirics (end)

\section{Implications}\label{sec:Implications} % (fold)
The study seeks to highlight that, while niche parties are often characterized by their rigid ideological orientations and less flexible constituencies, the changing preferences of voters regarding representation and their dissatisfaction with incumbents may significantly influence support for these parties. This could suggest that representational dynamics play an even greater role than previously recognized, challenging existing theories about niche party support.

% section Implications (end)

\printbibliography
\end{document}
