%%%%%%%%%%%%%%%%%%%%%%%%%%%% Define Article %%%%%%%%%%%%%%%%%%%%%%%%%%%%%%%%%%
\documentclass[nobib, openany, justified, a4paper, 14pt]{tufte-book}
%%%%%%%%%%%%%%%%%%%%%%%%%%%%%%%%%%%%%%%%%%%%%%%%%%%%%%%%%%%%%%%%%%%%%%%%%%%%%%%

%%%%%%%%%%%%%%%%%%%%%%%%%%%%% Citations %%%%%%%%%%%%%%%%%%%%%%%%%%%%%%%%%%%%%%%
%\usepackage[utf8]{inputenc}
\usepackage[style=authoryear-icomp]{biblatex}
%\usepackage[style=apa]{biblatex}
\addbibresource{/Users/ubd/Bibliotheca/bib.bib}
%%%%%%%%%%%%%%%%%%%%%%%%%%%%%%%%%%%%%%%%%%%%%%%%%%%%%%%%%%%%%%%%%%%%%%%%%%%%%%%

%%%%%%%%%%%%%%%%%%%%%%%%%%%%% Using Packages %%%%%%%%%%%%%%%%%%%%%%%%%%%%%%%%%%
\usepackage{newunicodechar}
\newunicodechar{🦜}{[parrot]}
\PassOptionsToPackage{prologue,dvipsnames}{xcolor}
\sloppy  % globally
\usepackage{geometry}
\usepackage{graphicx}
\usepackage{amssymb}
\usepackage{amsmath}
\usepackage{amsthm}
\usepackage{empheq}
\usepackage{mdframed}
\usepackage{booktabs}
\usepackage{lipsum}
\usepackage{graphicx}
\usepackage{color}
\usepackage{psfrag}
\usepackage{pgfplots}
\usepackage{bm}
\usepackage{epigraph}
\usepackage{titlesec}
\usepackage{tcolorbox}
\usepackage{csquotes}
\usepackage{pifont}
\usepackage{enumitem,amssymb}
% \usepackage{spoton} % adds \todo functionality I hope
%%%%%%%%%%%%%%%%%%%%%%%%%%%%%%%%%%%%%%%%%%%%%%%%%%%%%%%%%%%%%%%%%%%%%%%%%%%%%%%

% Other Settings

%%%%%%%%%%%%%%%%%%%%%%%%%% Page Setting %%%%%%%%%%%%%%%%%%%%%%%%%%%%%%%%%%%%%%%

%%%%%%%%%%%%%%%%%%%%%%%%%% Define some useful colors %%%%%%%%%%%%%%%%%%%%%%%%%%
\definecolor{maroon}{RGB}{128,0,0} %hlred
\definecolor{MAROON}{RGB}{128,0,0} %hlred
\definecolor{deepBlue}{RGB}{61,124,222} %url-links
\definecolor{deepGreen}{RGB}{26,111,0} %citations
\definecolor{ocre}{RGB}{243,102,25}
\definecolor{mygray}{RGB}{243,243,244}
\definecolor{shallowGreen}{RGB}{235,255,255}
\definecolor{shallowBlue}{RGB}{235,249,255}
\definecolor{mediumpersianBlue}{rgb}{0.0, 0.4, 0.65}
\definecolor{persianBlue}{rgb}{0.11, 0.22, 0.73}
\definecolor{persianGreen}{rgb}{0.0, 0.65, 0.58}
\definecolor{persianRed}{rgb}{0.8, 0.2, 0.2}
\definecolor{debianRed}{rgb}{0.84, 0.04, 0.33}
%%%%%%%%%%%%%%%%%%%%%%%%%%%%%%%%%%%%%%%%%%%%%%%%%%%%%%%%%%%%%%%%%%%%%%%%%%%%%%%

%%%%%%%%%%%%%%%%%%%%%%%%%% Indentation Settings %%%%%%%%%%%%%%%%%%%%%%%%%%%%%%%
\makeatletter
% Paragraph indentation and separation for normal text
\renewcommand{\@tufte@reset@par}{%
	\setlength{\RaggedRightParindent}{0pc}%1.0
	\setlength{\JustifyingParindent}{0pc}%1.0
	\setlength{\parindent}{1pc}%1pc
	\setlength{\parskip}{5pt}%0pt
}
\@tufte@reset@par

% Paragraph indentation and separation for marginal text
\renewcommand{\@tufte@margin@par}{%
	\setlength{\RaggedRightParindent}{0pc}%0.5pc
	\setlength{\JustifyingParindent}{0pc}%0.5pc
	\setlength{\parindent}{0.5pc}%
	\setlength{\parskip}{5pt}%0pt
}
\makeatother



%%%%%%%%%%%%%%%%%%%%%%%%%% Define an orangebox command %%%%%%%%%%%%%%%%%%%%%%%%
%o\usepackage[most]{tcolorbox}

\newtcolorbox{orangebox}{
	colframe=ocre,
	colback=mygray,
	boxrule=0.8pt,
	arc=0pt,
	left=2pt,
	right=2pt,
	width=\linewidth,
	boxsep=4pt
}


\newtcolorbox{redbox}{
	colframe=red,
	boxrule=0.8pt,
	arc=0pt,
	left=2pt,
	right=2pt,
	width=\linewidth,
	boxsep=4pt
}
%%%%%%%%%%%%%%%%%%%%%%%%%%%%%%%%%%%%%%%%%%%%%%%%%%%%%%%%%%%%%%%%%%%%%%%%%%%%%%%

%%%%%%%%%%%%%%%%%%%%%%%%%%%% English Environments %%%%%%%%%%%%%%%%%%%%%%%%%%%%%
\newtheoremstyle{mytheoremstyle}{3pt}{3pt}{\normalfont}{0cm}{\rmfamily\bfseries}{}{1em}{{\color{black}\thmname{#1}~\thmnumber{#2}}\thmnote{\,--\,#3}}
\newtheoremstyle{myproblemstyle}{3pt}{3pt}{\normalfont}{0cm}{\rmfamily\bfseries}{}{1em}{{\color{black}\thmname{#1}~\thmnumber{#2}}\thmnote{\,--\,#3}}
\theoremstyle{mytheoremstyle}
\newmdtheoremenv[linewidth=1pt,backgroundcolor=shallowGreen,linecolor=deepGreen,leftmargin=0pt,innerleftmargin=20pt,innerrightmargin=20pt,]{theorem}{Theorem}[section]
\theoremstyle{mytheoremstyle}
\newmdtheoremenv[linewidth=1pt,backgroundcolor=shallowBlue,linecolor=deepBlue,leftmargin=0pt,innerleftmargin=20pt,innerrightmargin=20pt,]{definition}{Definition}[section]
\theoremstyle{myproblemstyle}
\newmdtheoremenv[linecolor=black,leftmargin=0pt,innerleftmargin=10pt,innerrightmargin=10pt,]{problem}{Problem}[section]
%%%%%%%%%%%%%%%%%%%%%%%%%%%%%%%%%%%%%%%%%%%%%%%%%%%%%%%%%%%%%%%%%%%%%%%%%%%%%%%

%%%%%%%%%%%%%%%%%%%%%%%%%%%%%%% Plotting Settings %%%%%%%%%%%%%%%%%%%%%%%%%%%%%
\usepgfplotslibrary{colorbrewer}
\pgfplotsset{width=8cm,compat=1.9}
%%%%%%%%%%%%%%%%%%%%%%%%%%%%%%%%%%%%%%%%%%%%%%%%%%%%%%%%%%%%%%%%%%%%%%%%%%%%%%%

%%%%%%%%%%%%%%%%%%%%%%%%%%%%%%% MISC %%%%%%%%%%%%%%%%%%%%%%%%%%%%%%%%%%%%%%%%%%
\usepackage[acronym]{glossaries}
\usepackage{hyperref} % Setup: https://www.overleaf.com/learn/latex/Hyperlinks
\hypersetup{
	colorlinks=true,
	%citecolor=deepGreen,
	citecolor=maroon,
	linkcolor=persianBlue,
	filecolor=persianGreen,
	urlcolor=persianBlue,
	pdfpagemode=FullScreen,
}

%%%%%%%%%%%%%%%%%%%%%%%%%%%%%%%%%%%%%%%%%%%%%%%%%%%%%%%%%%%%%%%%%%%%%%%%%%%%%%%
\setcounter{tocdepth}{2}
\setcounter{secnumdepth}{2}

\newcommand{\hlred}[1]{\textcolor{Maroon}{#1}} % Print text in maroon
\newcommand{\hlgreen}[1]{\textcolor{persianGreen}{#1}} % Print text in green
\newcommand{\hlocre}[1]{\textcolor{ocre}{#1}} % Print text in green

\newenvironment{greenenv}{\color{Green}}{\ignorespacesafterend}  % Create green environment
\newenvironment{commentenv}{\color{ocre}}{\ignorespacesafterend}  % Create comment environment


\titleformat{\part}[display]
{\filleft\fontsize{40}{40}\selectfont\scshape}
{\fontsize{90}{90}\selectfont\thepart}
{20pt}
{\thispagestyle{epigraph}}

\setlength\epigraphwidth{.6\textwidth}

%\makeatletter
%\epigraphhead
%{\let\@evenfoot}
%{\let\@oddfoot\@empty\let\@evenfoot}
%{}{}
%\makeatother


%%%%%%%%%%%%%%%%%%%%%%%%%%%%%%%%%%%%%%%%%%%%%%%%%%%%%%%%%%%%%%%%%%%%%%%%%%%%%%%
%TODO LIST
\newlist{todolist}{itemize}{2}
\setlist[todolist]{label=$\square$}
\newcommand{\cmark}{\ding{51}}%
\newcommand{\xmark}{\ding{55}}%
\newcommand{\done}{\rlap{$\square$}{\raisebox{2pt}{\large\hspace{1pt}\cmark}}%
	\hspace{-2.5pt}}
\newcommand{\wontfix}{\rlap{$\square$}{\large\hspace{1pt}\xmark}}

%%%%%%%%%%%%%%%%%%%%%%%%%%%%%%%%%%%%%%%%%%%%%%%%%%%%%%%%%%%%%%%%%%%%%%%%%%%%%%%
\newcommand{\greensquare}{\marginnote{\fcolorbox{green}{green}{\rule{0pt}{3mm}\rule{3mm}{0pt}}\quad}}
\newcommand{\yellowsquare}{\marginnote{\fcolorbox{yellow}{yellow}{\rule{0pt}{3mm}\rule{3mm}{0pt}}\quad}}
\newcommand{\redsquare}{\marginnote{\fcolorbox{red}{red}{\rule{0pt}{3mm}\rule{3mm}{0pt}}\quad}}



%%%%%%%%%%%%%%%%%%%%%%%%%%%%%%% Title & Author %%%%%%%%%%%%%%%%%%%%%%%%%%%%%%%%
\title{[M13-Expose] The Soft Machine of Control: Generative AI, Dividuals, and the Modulative Power}
\author{Utku B. Demir}
\date{\today}
%%%%%%%%%%%%%%%%%%%%%%%%%%%%%%%%%%%%%%%%%%%%%%%%%%%%%%%%%%%%%%%%%%%%%%%%%%%%%%%%

\begin{document}
\maketitle

\section{Introduction}

%In the rapidly evolving landscape of artificial intelligence (AI), particularly with the rise of generative AI (genAI), questions concerning the intersection of the operation of power, the construction of subjectivity, and the transformation of institutional structures gain new breadth.  This study explores the current operation of power within generative AI algorithms, situating them within Gilles Deleuze's theoretical framework describing the transition from Michel Foucault's \textit{disciplinary power} \parencite[]{Foucault1995} to Deleuze's concept of \textit{societies of control} \parencite[]{deleuze1995a}.

Foucault’s foundational work on discipline and governance maps the biopolitical mechanisms through which institutions historically shaped subjects: individuals are moulded through enclosure and surveillance, confined within well-defined spaces such as schools, factories, and prisons \parencite[]{foucault2008a, Foucault1995}. These institutions act as regulatory spaces, or “moulds,” producing what Foucault termed “docile bodies”, subjects conditioned to conformity through systematic observation and normalisation. In contrast, Deleuze contends in his seminal essay \textit{Postscript on the Societies of Control} \parencite[]{deleuze1995a} that contemporary power operates through a different modality altogether, shifting from fixed, enclosed spaces to fluid, networked systems. Within these “societies of control,” power becomes a continuous, adaptive force immanent to the flows of information that constitute modern digital environments. Control, in this sense, no longer functions by confining individuals within specific spaces but by continuously modulating behaviour, subjectivities through real-time data analytics. Deleuze’s concept of \textit{control} thus provides a compelling framework for analysing the operation of power in what has been termed the \textit{computational turn} \sidenote{A term popularised by \cite{Hildebrandt2013} to mark the emergence of big data and advanced data analytics.}.

With the emergence of digital surveillance and the participatory web \sidenote{Also known as Web 2.0 \parencite[]{Oreilly2009}}, the visibility of bodies and subjectivities has transformed dramatically \parencite[]{Krasmann2017}. Machine learning (ML) and AI algorithms, particularly generative AI, represent an advanced stage of this computational evolution, warranting a renewed investigation into how these systems embody Deleuze’s notion of \textit{control}. Early algorithmic models categorised and filtered information based on simple parameters, but today’s ML algorithms deploy \textit{dividualising} processes, breaking down identities into atomised data points, or “dividuals,” which are then reassembled into probabilistic associations and behavioural predictions. As ML algorithms govern content visibility, assess relevance, and curate personalised feeds on social media, they do so by continuously analysing individualised behavioural patterns, possibly binding users within feedback loops of past behaviour and group associations \parencite[]{Otterlo2013, Cheney2011}. The \textit{dividualisation} process represents a shift from explicit control over information access to a more pervasive modulation of subjectivities, where the boundaries of agency are constantly reshaped by algorithmic processes.  As \cite{Rouvroy2013} argues, these feedback loops generate a subtle yet pervasive normalisation process that lacks a specific endpoint or directive. Thus, it is a matter of research whether the aspect of control is as emergent as the concept of modulation, given that these feedback loops lack a fixed direction.

While this first stage of sophisticated AI models used to regulate information flow is highly relevant to Deleuze’s analysis of the \textit{control society}, recent breakthroughs in generative AI mark a significant leap. The development of generative models, particularly Large Language Models (LLMs), shifts AI beyond mere categorisation and filtering, embedding these technologies within the act of content creation. These models, drawing from vast linguistic, statistical, and computational datasets, extend computational power into the realm of creation and knowledge generation. LLMs, by autonomously producing text, images, and other digital artefacts, transcend traditional epistemological boundaries, thereby introducing a novel logic for mapping and representing human knowledge \parencite[]{amoore2024}. Generative AI systems, therefore, are not merely tools for steering subjectivity; they actively participate in the production of digital reality.  By perceiving, categorising, and modelling the world probabilistically, these models shape the socio-political landscape in distinct ways. \cite{amoore2024} describe generative AI’s political logic as “distributional,” where the politics of generative AI emerges from probabilistic estimation, allowing it to shape decision-making and action without an explicit foundation \parencite[]{amoore2024}. Engaging with human desires, cultural narratives, and social structures, these models position themselves as influential agents within the socio-political domain, shaping knowledge, identity, and values based on probabilistic “world modelling” \parencite[]{amoore2024}.

Thus, this study investigates how generative AI extends Deleuze’s logic of control societies, examining its implications for digital subjectivity, social norms, and the constitution of power within computational networks. It interrogates whether generative AI functions as an instrument of a specific governmentality or whether its further development is capable of deviation, producing lines of flight and deploying nomadic subjectivities. Along those lines, the exploration also begins with the assumption that Deleuze’s other works \sidenote{For example, \textit{Anti-Oedipus} \cite*{deleuze1983} and \textit{A Thousand Plateaus} \cite*{deleuze1987}} may offer more comprehensive and accurate tools to analyse the current operation of power compared to the concept of \textit{Control Societies}.

%\section{Research Question}

%\begin{center}
%
%	\textit{RQ: How does the development of the current AI algorithms, particularly those of generative AI models (genAI), embody, extend, or contradict Gilles Deleuze’s concept of societies of control by modulating subjectivity through its probabilistic, non-linear operation structures?}
%\end{center}
%
%\begin{itemize}
%	\item [\textbf{Hypothesis 1:}] The distributional structure of generative AI, which produces content based on joint probability distributions, creates a dynamic of non-linear causality that reflects modulation without control, where subjectivity is formed through continuous, data-driven modulation rather than direct causative action in the absence of directional control. \sidenote{In line with Deleuze's own argument in his other works: Capitalism decodes, but does not necessarily recode. The dividualising process also functions as a mechanism to keep movements built on these parts within specific thresholds or asymptotes; "Your ethnicity has been recognised, buy a flag".}
%	\item [\textbf{Hypothesis 2:}] Despite their current state, future genAI models
%	      offer unique ways for individuals to create lines of
%	      flight and support the becoming of nomadic subjects. \sidenote{This aligns with
%		      the construction of schizo connections as described in \textit{Anti-Oedipus} \cite*{deleuze1983}.}
%\end{itemize}
%
%\section{State of the Art}
%While there are various attempts to adapt the notion of control to modern
%digital developments, such as those by \cite{brusseau2020}, or partially or completely rejecting the concept of control as an inadequate explanation of current digital power structures (see, for example, \cite{hui2015}), these works either overlook the novelty and potential of genAI mechanisms or fail to explore the \textit{dividual} dimension.  Furthermore, earlier works focusing specifically on the aspect of dividualisation (see e.g. \cite{Cheney2011, Otterlo2013}) are, unfortunately, temporally limited as they were unable to analyse generative models that had not yet reached the level of advancement we see today.
%
%Other theorists focus on various aspects of genAI models, with a common tendency to analyse ethical considerations, their application under neoliberal governmentality, and their role within surveillance capitalism (see e.g. \cite{zuboff2019, gillespie2024, Haggerty2000}). While these aspects are not outside the scope of this research, they were central to my previous bachelor's thesis.  This current study specifically focuses on the structure of probabilistic models, particularly examining their construction at present (and speculatively in the future), without delving into how their misuse (or intentional use) may play out.
%
%Furthermore, while this study addresses the political implications of current generative AI usage, its focus is primarily on the analysis of power operations and, within this context, relevant to questions of democracy only insofar as they relate to the operation of power.  This study does not, however, aim to provide a comprehensive analysis of democratic risks associated with AI, as explored by others (see e.g. \cite{zarkadakes2020, coeckelbergh2024}).
%
%This research, therefore, is focused on three primary areas of debate: (1) an analysis of genAI models as entities deploying power, examining their mechanisms (see e.g. \cite{amoore2024, konik2015, mackenzie2021}); (2) modern reflections on Deleuze's theory (see e.g. \cite{mischke2021c, poster2010}); and (3) sources analysing algorithmic structures with technical expertise (see e.g. \cite{vaswani, bender2021b}).

% While Deleuze notes Foucault's awarenes of the transformation of the disciplinary institutions as early as the end of the second world war, in his own formulation Deleuze marks the following state of this transformation as the modulation. The \textit{molded} individuals of the disciplinary societies are produced through confinement, whereas the \textit{modulation} traverses in an ubiqutous sese "both the social field and the life-course of individuals" \parencite[6]{mackenzie2021}.

%In relation to the former,  Deleuze returns to the brief examples of how
%disciplinary institutions are breaking  down and being supplanted by mechanisms
%of control located in the continual  modulation of dividuals. \parencite[7]{mackenzie2021}



%Galloway provides a striking case in point: ‘the society  of control is
%characterized not by the power of the institutions of modernity, or
%pre-modernity, the army, the prison, the university, the church, but instead by
%what he [Deleuze] called the ultra-rapid forms of free-floating control that
%are  inherent in distributed networks’ (Galloway, 2006, pp. 318-9). \parencite[8]{mackenzie2021}

\section{Theoretical and Conceptual Framework}
This study focuses primarily on Gilles Deleuze's short essay on
control societies \parencite[]{deleuze1992a}, as well as his earlier works (see
\cite{deleuze1983, deleuze1987}), with selected relevant sections from
Foucault's concept of biopolitics (see \cite{foucault2008a, foucault1988}).

After analysing the concept of \textit{control societies} and further concepts from Deleuze’s work with Félix Guattari, the analysis will continue with a temporal exploration of the development of AI models in general, including a comprehensive examination of current and potential capabilities of genAI models. After incorporating insights from selected works related to current debates, the hypotheses will be examined to analyse how Deleuze and Guattari’s theory of control should be interpreted in the context of the current AI landscape and what forms of resistance mentioned in their work might find some realisation within these algorithmic structures.


\section{Expected Results and Contributions to Current Debates}

\begin{itemize}
	\item While the current power the digital constellation deploy does not fully reflect Deleuze's control society, modulation seems to be going much further than this concept. Modulation, however, both in terms of how the subjectivity is produced and what kind of modulative capacities genAI models prevail are to be better analysed with other sources from Deleuze \& Guattari. While modulation seems to be directionless, it is structurally preserves a power structure.
	\item Through analysing the operational structures of genAI models, such as LLMs, I expect to uncover the latent probabilistic distributions operating within a specific governmentality and deploying feedback loops steering human subjectivity in a specific way. While these models are completely dependant on the new human generated data, they also carry the risk to reduce the politically imaginary capabilities of the individuals.
	\item  I also aim to identify potential for these approaches to foster creativity through different methods of application and training, potentially aligning with the forms of resistance that Deleuze and Guattari describe.
\end{itemize}

\section{Preliminary Schedule}
\begin{itemize}
	\item [\textbf{November - Early December:}] Literature research
	\item [\textbf{December - End of January}] Writing process, draft
	\item [\textbf{February}] Literature research in response to feedback
	\item [\textbf{February - Early March}] Writing process
	\item [\textbf{Mid-March}] Final delivery
\end{itemize}


\nocite{wiley2019}
\nocite{amoore2024}
\nocite{bender2021b}
\nocite{brusseau2020}
\nocite{calvo2024}
\nocite{chatterjee2024}
\nocite{cohen2018}
\nocite{dominguezgonzalez2023}
\nocite{gillespie2024}
\nocite{Haggerty2000}
\nocite{hardt2017}
\nocite{hui2015}
\nocite{jiang2024}
\nocite{konik2015}
\nocite{kruger2021}
\nocite{lecun}
\nocite{mackenzie2021}
\nocite{mischke2021c}
\nocite{poster2010}
\nocite{ryan2024}
\nocite{vaswani}
\nocite{Deleuze1992}
\nocite{ettlinger2018}
\nocite{Deleuze1998}
\nocite{wiley2019}

\printbibliography[title={Preliminary Bibliography}]
\end{document}
