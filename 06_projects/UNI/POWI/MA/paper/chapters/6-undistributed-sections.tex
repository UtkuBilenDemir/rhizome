\chapter{Undistributed sections}\label{chap:Undistributed sections} % (fold)

% section Transition to GenAI (end)
\section{From external force to meaning-making}\label{sec:From external force to meaning-making} % (fold)

% section From external force to meaning-making (end)
Dishon (\citeyear{dishon2024}) underlines it as the transition from a Frankesteinian
imaginary to a kafkaesque process meaning-creation stallmate.

%TODO: Explain AI imagineries with the help of Dishon
→ Thus, artificial life is portrayed as a discrete entity, largely mirroring
human agency. In the Frankenstein imaginary, the construct is external to the
what is human with human-like abilities. The concern of agency in this case
reflects the possibility of the construct taking over.

\begin{quote}
	Though AI is depicted as possessing superior capabilities—both physically and mentally (the creature quickly learns to talk and read)—the overall logic governing pattens of meaning-making remains stable. This overall similarity serves as the background according to which certain differences can be appreciated and highlighted. Hence, it is not so much the rationale of agency or meaning-making that shifts, it is the actor who holds the privileged author position—the fear that machines will replace humans is embedded within current structures of meaning-making \parencite[966]{dishon2024} .
\end{quote}

This is an anthropomorphic depiction of AI, and similarly associated with an
anthropomorphic threat. The threat is as much as it is in depiction a discrete
one. An external agent is getting in contact with the human manifesting
anthropomorphic qualities, and the threat is either perfecting or advancing beyond
anything anthropomorphic nature. The discrete threat introduces immediate
concerns about humans' responsibilities of their own making. The question
becomes one of survival, or at the very least about losing humanity's place on
the \emph{food chain}. The concern is that the AI can develop human desires,
the threat is derived from AI possibly being too much of human nature than vice
versa.

%TODO: Introduce the Trial by Kafka analogy here
Dishon \citeyear{dishon2024} puts the process of Kafka's Trial as a counter
analogy here. Contrary to the anthropomorpistic imlplications of the
Frankenstein analogy, the court of the \emph{Trial} "does not have a
well-defined system or a reference to \emph{truth}" \parencite[970]{dishon2024}. It is
completely driven by the qualities, interactions, and commitment %TODO: commitment?
of the subjectivity in interaction. The source of meaning-making in this sense
is completely free of any form of agency. Moreover, as in the interaction of K. with the court
the mechanism is working against any identifiable agency of the interactor
themselves \parencite[see 970]{dishon2024}. The operation is rather blurring
the existing agency itself instead of producing one.

The Court contradicts the externality of the Frankenstein metaphor in its
nature of interaction with the subjectivity engaged with it. The interaction
with The Court is necessary, but the effect of this interaction seems to be
futile.

\begin{quote}
	Thus, paradoxically, the best way to arrive at a definite outcome is to make sure the process never ends. Interactions with the court are necessary and require constant maintenance, yet they cannot be controlled, predicted, or even expected to progress towards a resolution. The novel could be viewed as turning the meaning of a trial upside down—it is not meant to arrive at a verdict or outcome, rather it is the process itself on which one must center. [...] The Trial offers a depiction of control that relies on a different understanding of meaning-making—shifting from a stable and general model of meaning to an idiosyncratic and personalized one.

	-- \textcite[970-972]{dishon2024}
\end{quote}

The meaning-making process K. is subjected to asking for his submission to the
produced truth as it is tailored exactly for him, and there is no specific way
to test its genuinity. And the product of the process seems to be signifying an
inaccessability. This is a shift between internal and external aspects of
agency, a blurring in between layers as well as the nature and perception on
both sides' potential agency.

Through kafkaesque analogy
%TODO: Other analogies?
GenAI becomes an internal agent instead of external entity leads to a blurring
of agency and machinic intentionality. The human's struggle in the process is
to constantly having to assess, to decipher AI's agency. Furthermore, the GenAI
model is not necessarily trying to offer a right answer but to generate content
without necessarily relying to veracity or accuracy leading to an amplification
of human's endless search for meaning \parencite[see
	977]{dishon2024}.

\begin{quote}
	The Trial is not about humans losing control over their creations, if they ever had control in the first place. Instead, it foreshadows GenAI’s capacity to generate content that is personalized to every actor (and thus shaped by humans) yet is not amenable to control through explicit choices. This model of meaningmaking undermines the dichotomy between choice and coercion, no longer positioning the two as mutually exclusive. [...] Therefore, as in Before the Law, personalization does not inherently entail increased control over meaning-making, but rather its increased mediation according to what GenAI identifies, or perhaps determines, as our personal preferences.

	-- \textcite[974]{dishon2024}
\end{quote}

GenAI offers in this sense an endless variation of meaning specifically
personalised. What does it mean in term's of human's choice?
%TODO: Does it really personlaise answers?
%

\section{Instance}\label{sec:Instance} % (fold)
%TODO: For example any subjectivity is an instance on a network of partical
%objects (or larval subjects) -> Refer to the chapter 1 in both buchanan2008
%and AO

% section Instance (end)
\section{Enregistrement \parencite[4]{deleuze1983} }\label{sec:Enregistrement \parencite[4]{deleuze1983} } % (fold)
%TODO: Refer to the double entendre of French meaning in the footnotes. It
%explains the BwO pretty well.
%

\section{Process}\label{sec:Process} % (fold)

% section Process (end)


% section Enregistrement \parencite[4]{deleuze1983}  (end)

% chapter Undistributed sections (end)
