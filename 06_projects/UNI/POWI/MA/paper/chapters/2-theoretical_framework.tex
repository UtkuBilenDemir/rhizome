\chapter{Theoretical Framework}\label{chap:Theoretical Framework} % (fold)
One could argue that artificial intelligence is \emph{no longer an engineering discipline} \parencite[206]{dignum2020}, if it has ever truly been one.

\section{Research Question}

\begin{center}

	\textit{RQ: How does the development of the current AI algorithms, particularly those of generative AI models (genAI), embody, extend, or contradict Gilles Deleuze’s concept of societies of control by modulating subjectivity through its probabilistic, non-linear operation structures?}

	\textit{RQ (Alternative): To what extent can the processes of meaning-production in generative AI algorithms, particularly the transformer architecture be understood through Gilles Deleuze’s concepts of control and modulation, especially in relation to subjectivity construction via their probabilistic, non-linear operational structures?}
\end{center}

\begin{itemize}
	\item [\textbf{Hypothesis 1:}] The distributional structure of generative AI, which produces content based on joint probability distributions, creates a dynamic of non-linear causality that reflects modulation without control, where subjectivity isiiiii formed through continuous, data-driven modulation rather than direct causative action in the absence of directional control. \sidenote{In line with Deleuze's own argument in his other works: Capitalism decodes, but does not necessarily recode. The dividualising process also functions as a mechanism to keep movements built on these parts within specific thresholds or asymptotes; "Your ethnicity has been recognised, buy a flag".}
	\item [\textbf{Hypothesis 2:}] Despite their current state, future genAI models
	      offer unique ways for individuals to create lines of
	      flight and support the becoming of nomadic subjects. \sidenote{This aligns with the construction of schizo connections as described in \textit{Anti-Oedipus} \cite*{deleuze1983}.}
\end{itemize}

\section{State of the Art}

\marginnote{
	\begin{orangebox}
		Mention;

		\begin{itemize}
			\item \cite{rouvroy}
			\item \cite{pasquinelli2023}
		\end{itemize}


	\end{orangebox}
}

While there are various attempts to adapt the notion of control to modern
digital developments, such as those by \cite{brusseau2020}, or partially or completely rejecting the concept of control as an inadequate explanation of current digital power structures (see, for example, \cite{hui2015}), these works either overlook the novelty and potential of genAI mechanisms or fail to explore the \textit{dividual} dimension.  Furthermore, earlier works focusing specifically on the aspect of dividualisation (see e.g. \cite{Cheney2011, Otterlo2013}) are, unfortunately, temporally limited as they were unable to analyse generative models that had not yet reached the level of advancement we see today.

Other theorists focus on various aspects of genAI models, with a common tendency to analyse ethical considerations, their application under neoliberal governmentality, and their role within surveillance capitalism (see e.g. \cite{zuboff2019, gillespie2024, Haggerty2000}). While these aspects are not outside the scope of this research, they were central to my previous bachelor's thesis.  This current study specifically focuses on the structure of probabilistic models, particularly examining their construction at present (and speculatively in the future), without delving into how their misuse (or intentional use) may play out.

Furthermore, while this study addresses the political implications of current generative AI usage, its focus is primarily on the analysis of power operations and, within this context, relevant to questions of democracy only insofar as they relate to the operation of power.  This study does not, however, aim to provide a comprehensive analysis of democratic risks associated with AI, as explored by others (see e.g. \cite{zarkadakes2020, coeckelbergh2024}).

This research, therefore, is focused on three primary areas of debate: (1) an analysis of genAI models as entities deploying power, examining their mechanisms (see e.g. \cite{amoore2024, konik2015, mackenzie2021}); (2) modern reflections on Deleuze's theory (see e.g. \cite{mischke2021c, poster2010}); and (3) sources analysing algorithmic structures with technical expertise (see e.g. \cite{vaswani, bender2021b}).


% chapter Theoretical Framework (end)
