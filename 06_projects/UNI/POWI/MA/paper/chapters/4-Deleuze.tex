\chapter{Deleuze; Disjuntive Synthesis}\label{chap:Deleuze} % (fold)

\section{An Account about Micropolitics}

\section{IDEA:Institution}
\begin{commentenv}
	There is an idea about institutions I do not know how to explore just yet.
	The roots are in the following part of AO:


\end{commentenv}

\begin{quote}
	The two kinds of fantasy, or rather the two regimes, are therefore distinguished according to whether the social production of "goods" imposes its rule on desire through the intermediary of an ego whose fictional unity is guaranteed by the goods themselves, or whether the desiring-production of affects imposes its rule on institutions whose elements are no longer anything but drives. If we must still speak of utopia in this sense, a la Fourier, it is most assuredly not as an ideal model, but as revolutionary action and passion. In his recent works Klossowski indicates to us the only means of bypassing the sterile parallelism where we flounder between Freud and Marx: by discovering how social production and relations of production are an institution of desire, and how affects or drives form part of the infrastructure itself. For they are part of it, they are present there in every way while creating within the economic forms their own repression, as well as the means for breaking this repression.\cite[63]{deleuze1983}.
\end{quote}



%\section{Desire}\label{sec:Desire} % (fold)
%The history of power is the history of the regulation of desire.
%%TODO: Citation needed
%%
%\begin{quote}
%	desire is revolutionary in its essence-desire, not left-wing holidays !-and
%	no society can tolerate a position of real desire without its structures of
%	exploitation, servitude, and hierarchy being compromised.
%	\cite[116]{deleuze1983}
%\end{quote}
%
%In Deleuze and Guattari's argumentation desire emerges as one of the most radical and transformative concepts, fundamentally challenging traditional psychoanalytic models, particularly those of Freud and Lacan. Where psychoanalysis tends to define desire through the lens of lack (a longing for something absent) \gls{dg} reorient desire towards abundance as a framework of production and creation. Desire, for Deleuze, is not a negative force driven by deficiency but a generative, affirmative movement that actively constructs the real.
%%%TODO: Citation needed
%Desire is revolutionary in itself along \gls{dg}'s argumentation, "but it is
%constantly being shackled, or worse converted into interest which is
%suspectible to capture, domestication, and pacification"
%\parencite[11]{buchanan2008}. A pure manifestation of desire without any
%alteration, decoding, deterritorialisation would be enough to undermine
%capitalist society's fundamental structures, but pure manifestations of desire
%are rare, even in revolutionary situations \parencite[11]{buchanan2008}.
%

%\subsection{Creativity; discrete vs. continuous}
%
%\begin{orangebox}
%	What is the assumption regarding \gls{dg}'s assumption about desire. Is desire
%	an initiation of creativity, is shizoprocess a form of releasing initial (and
%	essential) creativity in human potential?
%
%	The association of desire with the creativity is very well pronounced in
%	\gls{dg}'s work, however desire or the schizoprocess in itself are initiators
%	or garrantors of desire. The schizz is the process of how desire is
%	produced, it is the nature and source of the desiring-production, and the
%	production of desire is not something initiated by mechanism reaching towards
%	creativity. Human consciousness is entirely productive, productive of desiring-production, productive of production. Desire's primary role is production, production of production, it is abundance \parencite[49]{buchanan2008e}. Desiring-production binds partial objects that are by nature \textbf{fragmentary and fragmented}, desire constantly couples countinous flows and breaks/interrupts other flows \parencite[5]{deleuze1983} .
%
%\end{orangebox}

%\section{Dividuation}
%\begin{quote}
%	What is a dividual? A dividual is a bundle of elements held together in variation  rather than in reference to a unitary subject. Where disciplinary institutions  segmented the life-course of individuals into separate subjective roles and  functions, control modulates elements of subjectivity across the entire social field.
%	\parencite[5]{mackenzie2021}
%\end{quote}
%
% section Desire (end)
%\section{Modulation}\label{sec:Modulation} % (fold)
%%TODO: Write an entry for Modulation
%
%Picking up from Foucault's steps of civilisation, \emph{the society of sovereignity}
%
%Deleuze's introduction of the modulation against the concept of moulding of the
%discipline marks a transition from the "form-imposig mode to a self-regulating
%mode" \parencite[74]{hui2015}.
%
% section Modulation (end)
%\section{Control}\label{sec:Control} % (fold)
%%TODO: When you are writing this part, rely heavily on Burroughs, which Deleuze
%%does not do
%
%\begin{quote}
%	what Deleuze terms the “societies of  control” that inhabit the late twentieth century—these are based on protocols, logics of “modulation,” and the “ultrarapid forms of free-floating control”
%	\parencite[86]{galloway2004}
%\end{quote}
%
%\begin{quote}
%	Hardt and Negri confirm this position by writing elsewhere that “the passage to the society of control does not in any way mean the end of discipline. In fact, the immanent exercise of discipline . . . is extended even more generally in the society of control” (330)
%	\parencite[83]{galloway2001}
%\end{quote}
%% section Control (end)
%%

\section{Subordination to Voice}
\begin{orangebox}
	https://www.youtube.com/watch?v=5t1vTLU7s40
\end{orangebox}


%\section{Meaning-Making}
%\subsection{Saussurean Approach}
%\begin{quote}
%	To the extent that LLMs excel at conversation, they verify Saussure’s insight that meaning emerges from the interplay of signs in a formal system. There is no inherent need for actual sensory grounding. If “a sign stands in the place of something else” (Saussure, 1959, p. 66), then for an LLM, the “something else” could be another cluster of words, or a swirl of pixels if it is visually enabled, all existing within the confines of digital memory. Meanwhile, Peirce’s emphasis on iconic signs — signs that resemble their object — and indexical signs — signs that point to or are causally connected with their object — seems, on the surface, less relevant to an AI that navigates text tokens rather than the physical world. Without a body to roam or eyes to see, the Peircean structure appears incomplete inside the machine’s domain.
%	\parencite[]{phd2025}
%\end{quote}


%\subsection{Language in LLM}
%
%\begin{quote}
%	It represents nothing, but it produces. It means nothing, but it works.
%	Desire makes its entry with the general collapse of the question "What does
%	it mean?" No one has been able to pose the problem of language except to the
%	extent that linguists and logicians have first eliminated meaning; and the
%	greatest force of language was only discovered once a work was viewed as a
%	machine, producing certain effects, amenable to a certain use.
%	\cite[109]{deleuze1983}
%\end{quote}

%\section{Difference, Repetition, Singularity (Potential discussion about the
%  need for sensory input for the genAI (LeCUn?)}
%The role of the imagination, or the mind which contemplates in its multiple and fragmented states, is to draw something new from repetition, to draw difference from it. For that matter, repetition is itself in essence imaginary, since the imagination alone here forms the “moment” of the vis repetitiva from the point of view of constitution: it makes that which it contracts appear as elements or cases of repetition. Imaginary repetition is not a false repetition which stands in for the absent true repetition: true repetition takes place in imagination. \cite{deleuze1994, kruger2021}
%
%Once manifested as thought, furthermore, the thinking that happens is divergent and ramifying rather than convergent and identifying. \cite[175]{kruger2021}
%
%Thought emerges out of an evanescent materiality. It is exactly at this point where Deleuze parts ways with Kant. While the latter accepted the existence of a priori categories of mind that would stabilise and universalise the thought of the thinking subject, Deleuze maintains the radically empirical nature of the emergence of any transcendental structures. Thought emerges out of experience and can only ever be a response to experience. Experience, in turn, is bound up with matter, in the non-identical repetition of material intensities. \cite[178]{kruger2021}

%\section{Undistributed}
%
%%TODO:Consider this, what to do with this?
%%
%From a classical theoretical perspective, notably that of Max Weber, modern Western societies are increasingly structured by processes of rationalization. For Weber, the bureaucratic apparatus exemplifies this rationalizing tendency, organizing social life through formalized procedures, calculability, and technical efficiency. In this view, bureaucracy becomes the paradigmatic form of modern rational order \parencite[46]{kivisto2013} . Within the context of algorithmic systems and AI infrastructures, such rationalization is not only extended but intensified: decision-making is delegated to technical systems whose logic of control operates beneath the surface of human institutions, suggesting that contemporary forms of automation represent an advanced stage of this bureaucratic rationality.
%

% chapter Deleuze (end)
