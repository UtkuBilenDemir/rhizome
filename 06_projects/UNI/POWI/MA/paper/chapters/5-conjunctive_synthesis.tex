\chapter{Disjunctive Synthesis}

\sidenote{\begin{commentenv}
		https://youtu.be/1LtAHUtW8sk?si=s7mi6FXbyQC7qW1l
	\end{commentenv}


}

We should definitely ask ourselves whether the giant AI corporations would baulk at putting the levers of mass correlation at the disposal of regimes seeking national rebirth through rationalised ethnocentrism \parencite[]{mcquillan2019}. Both in terms of capability, resources and power, they are the ones running the show after all. However, even in the currently breadcrumb collecting process of the common involvement in open development in AI models, we also need to be capable of having a genealogy and critique in the very mechanism of the generative core of the \gls{ai} models. After all, their only meaning is not just the profatibility for the humanity, through its historical imaginary and contemporary discussions we are seeing possibility of our downfall but also contingency, and resistance; finally we also haven't completely forgot about its implication of emancipation in some part of the fiction.

How do the algorithm create meaningless attributes, like not stepping on the
lines between tiles, or a tic.

Can AI curse in a sense as reachtion?

As Slavoj Zizek exclaims, "Hegel mentions we live in language but we are not comfortable, we are not at
home in language" \parencite{berggrueninstitute2025} .

Deleuze notes that the transition between the primitive machine and the
despotic machine notes the subordination of a sign system to the
\textit{voice}.
%TODO: Citation needed.
Primitive societies had a symbolic system but it wasn't used to
mimic what the voice was saying, societal inscription was separated under oral
tradition and a system of signs. Writing was the ultimate subordination of a
sign system to the voice becoming the unified method of inscription, a
reterritorialisation of signs. \gls{genai} is subordination in a different
sense.
%TODO: Explain the nature of this subordination, the voice is claiming
%everything. A structure of voice taking command of the meaning. Eliminating
%meaning through the structural patterns in languate, or reterritorialising the
%meaning after taking out the history.
%
\section{Revolutionary possibilities?}
The experiencer aspect of the system, however, emerges through its engagement with external feedback loops. For example, the AI receives feedback from scientific evaluations of its predictions, critiques from researchers, and integrations of previously unseen data sets. Through this feedback, the AI refines its internal representations, enhancing the accuracy and scope of its models. As it adapts, the system begins to map these localized understandings onto a more abstract, global framework—capturing phenomena that may not align neatly with human interpretative schemas, such as non-linear and chaotic interactions between various ecological systems. In this iterative process, the AI system constructs transjective subjectivity: its ability to generate meaning that bridges internal, computational representations and external, relational dynamics. [[@rijos]]
These dynamics underscores the potential of computational systems to engage in a co-creative process with humans, not merely solving problems but redefining the frameworks of meaning itself. [[@rijos]]

Given the dispersed nature of these relationships, a singular weight doesn’t hold any conceptual content; instead, it obtains relevance through extensive interaction patterns. [@maas] -> He is referring to Cilliers(2002) but it also connects with Deleuze

A central aspect of social movement formation involves using language strategically to destabilize dominant narratives and call attention to underrepresented social perspectives. Social movements produce new norms, language, and ways of communicating. This adds challenges to the deployment of LMs, as methodologies reliant on LMs run the risk of ‘value-lock’, where the LM-reliant technology reifies older, less-inclusive understandings. \cite[614]{bender2021b}

\section{Follow-up on discreteness and continuity}
Human Consciousness is entirely productive
It process links partial objects together fragmented by history into lines of rational thoughts and belief that become expressed as some whole that we kind of say in whatever we do, in arts, statements, dogmas opinions…And this whole is made by desire. Lines of reason, lines of belief are flows of desire.
Reality is the productive flow that connects what is fragmanented. This is the process, this is the human mind.
**AI** prominent shift that eliminates desire. It fills all spaces with information/knowledge/data. And what are those fillings? Those are dogmas of state, science of status-quo, capital as a body without  Hegemonic Reproduction..
Revolutionary politics loses its impetus. **All that remains is reception, the acceptance of State, Science, Capital**
\cite{creativephilosophy2023}


I disagree, the **AI** answers desire, it claims the answer was always there. It doesn't block desire, it definitely doesn't kill desire, it rewires it, in a loop that feeds itself.

Not that the desire doesn't want to fulfil itself, but the creative flows benefit from the desire being fluid. The subject of desire shouldn't be this well-defined in order to breed creation.




As \cite{mcquillan2019} notes
\begin{quote}
	It needs to be more than debiasing datasets because that leaves the core of AI untouched. It needs to be more than inclusive participation in the engineering elite because that, while important, won't in itself transform AI. It needs to be more than an ethical AI, because most ethical AI operates as PR to calm public fears WHILE INDUSTRY GETS ON WITH IT. It needs to be more than ideas of fairness expressed as law, because that imagines society is already an even playing field and obfuscates the structural asymmetries generating the perfectly legal injustices we see deepening every day. \cite{mcquillan2019}
	\[...\]
	Real AI matters not because it heralds machine intelligence but because it confronts us with the unresolved injustices of our current system. An antifascist AI is a project based on solidarity, mutual aid and collective care. We don't need autonomous machines but a technics that is part of a movement for social autonomy.
\end{quote}
