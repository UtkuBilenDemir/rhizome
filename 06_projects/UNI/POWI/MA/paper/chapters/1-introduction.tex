\chapter{Introduction}\label{chap:Introduction} % (fold)

%%INFO: Raise of the genAI
%In recent years, there have been substantial advancements in the field of \gls{ai}, driven in particular by progress in \gls{nn} and \gls{dnn} architectures. These developments have enabled the deployment of predictive \gls{ai} models across a wide array of domains, ranging from social media platforms and search engines to natural language processing tasks such as text classification and topic modelling. While these applications primarily focused on analysis and prediction, a new paradigm has emerged in the form of \gls{genai}. Once a relatively silent front in \gls{nlp} research, \gls{genai}'s history goes back to 1950s \parencite[4]{cao2023a} . Unlike traditional predictive models, \gls{genai} systems are capable of producing novel outputs—such as text, images, or code—by extracting and operationalizing intent from human-provided instructions. This marks a significant shift in the goals and capabilities of \gls{ai}, moving beyond automation and decision support to enabling more efficient, scalable, and creative content generation processes.

%INFO: Generative AI Models deploy a specific governing rationality
%Although the meaning-making processes of the current \gls{genai} models are coming from a long history of development in various areas scientific areas and statistical approaches, new architectures deploy a novel interpretation, representation of the world via the inroduction of political model, a governing rationality (see \cite[2]{amoore2024})
%\sidenote{The governing rationality of \gls{genai} models (see e.g.	\cite{amoore2024}) is chosen to express	the underlying generational structure of algorithmic meaning-making without	initiating any confusion with \cite{rouvroy}'s concept of Algorithmic	Governmentality. While Rouvroy refers to an algorithmic turn in the	governmentality in neoliberal governance, governing rationality refers to the rationality established in the algorithmic models.}


%%INFO: Technical parts explained
%My work focuses especially on the architectural components that made the
%meaning-making process of the \gls{genai} Models as comprehensive as today,
%especially those of transformer architecture like "attention", "latency",
%"gradient descent".
%%TODO: Add more

%INFO: Could be better in the end
%The rapidly evolving landscape of \gls{ai}, particularly with the emergence of \gls{genai} and \gls{aigc}, brings renewed urgency to questions surrounding the entanglement of algorithmic systems with power, subjectivity, and institutional transformation. As \gls{genai} systems increasingly mediate communication, creativity, and decision-making, this study critically examines their role in the contemporary operation of power. Drawing on Gilles Deleuze’s theoretical account of the shift from Michel Foucault’s \textit{disciplinary societies} \parencite{Foucault1995} to what Deleuze terms \textit{societies of control} \parencite{deleuze1995a}, the analysis situates \gls{genai} as emblematic of a broader transformation in the modes through which governance, normativity, and subject-formation are algorithmically enacted.

%My work especially focuses on the political theory of the generative AI
%algorithms.
%
%Capitalism is distinctively characterised with the ability to decode,
%deterritorialise the flows, but not to recode, reterritorialise. The
%reterritorialisation in capitalism is delegated to to some institutional
%frameworks, or protocologicalised processes.

\section{utku}
I will try to focus on specific bits wethout concerning about the network they
are or might be connected to.

%One often urgently needed disclaimer in this cluster is that the concept of
%disciplinary societies is a definition of the specific operation of
%(bio-)power, speaking about control or post-disciplinary societies is not speaking about the absence or
%replacement of discipline, one can freely emphasise that the control is
%discipline (see \cite{kelly2015a} )

\section{(U) A word about institution}
Capitalism reterritorialises with one hand what it deterriorialises with the
other. But it is the delegation to specific institutional entities doing the
reterritorialisation. AI in its application is a part of the
reterritorialisation, but in its own unique structure by keeping the borders of
the socius in its own past and structure.

\cite{mackenzie2021} argument that in thw institutional framework of the
control societies, subjectivity construction operates on the basis of
dividualised subjects without the formation of a subjective centre \parencite[14]{mackenzie2021}.

This work only partly relies on the works of Antoniette Rouvroy
%TODO: Citation needed
and Gerald Raunig
%TODO: Citation needed
because both the exploration of the Algorithmic Governmentality and the
institutional formation in control societies does not include a structural
analysis, especially an analysis on the level of the algorithm itself,
neither any of them are looking for a "bit element" in the current formation of
capitalism

\sidenote{Although this statement somewhat aligns with \citeyear{mackenzie2021}'s claim, I do not agree that the 2 authors are not exploring the concepts they've initiated far enough. The problem rather lies in still not distinguished branches in the algorithmic critique, that would allow us to position ourselves around completely different disciplines and goals despite being on the same surface.}

%TODO: Introduce the abbreviation D&G

\section{Cognitive Capitalism}
If this is cogntive capitalism and this is the attention economy, then we need
to immediately turn to the subjectivity construction, as it is what defines the
nature of the attention.

\gls{genai}


% chapter Introduction (end)
