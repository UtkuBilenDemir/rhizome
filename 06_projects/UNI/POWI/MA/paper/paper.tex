%%%%%%%%%%%%%%%%%%%%%%%%%%%% Define Article %%%%%%%%%%%%%%%%%%%%%%%%%%%%%%%%%%
\documentclass[nobib, openany, justified, a4paper, 14pt]{tufte-book}
%%%%%%%%%%%%%%%%%%%%%%%%%%%%%%%%%%%%%%%%%%%%%%%%%%%%%%%%%%%%%%%%%%%%%%%%%%%%%%%

%%%%%%%%%%%%%%%%%%%%%%%%%%%%% Citations %%%%%%%%%%%%%%%%%%%%%%%%%%%%%%%%%%%%%%%
%\usepackage[utf8]{inputenc}
\usepackage[style=authoryear-icomp]{biblatex}
%\usepackage[style=apa]{biblatex}
\addbibresource{/Users/ubd/Bibliotheca/bib.bib}
%%%%%%%%%%%%%%%%%%%%%%%%%%%%%%%%%%%%%%%%%%%%%%%%%%%%%%%%%%%%%%%%%%%%%%%%%%%%%%%

%%%%%%%%%%%%%%%%%%%%%%%%%%%%% Using Packages %%%%%%%%%%%%%%%%%%%%%%%%%%%%%%%%%%
\usepackage{newunicodechar}
\newunicodechar{🦜}{[parrot]}
\PassOptionsToPackage{prologue,dvipsnames}{xcolor}
\sloppy  % globally
\usepackage{geometry}
\usepackage{graphicx}
\usepackage{amssymb}
\usepackage{amsmath}
\usepackage{amsthm}
\usepackage{empheq}
\usepackage{mdframed}
\usepackage{booktabs}
\usepackage{lipsum}
\usepackage{graphicx}
\usepackage{color}
\usepackage{psfrag}
\usepackage{pgfplots}
\usepackage{bm}
\usepackage{epigraph}
\usepackage{titlesec}
\usepackage{tcolorbox}
\usepackage{csquotes}
\usepackage{pifont}
\usepackage{enumitem,amssymb}
% \usepackage{spoton} % adds \todo functionality I hope
%%%%%%%%%%%%%%%%%%%%%%%%%%%%%%%%%%%%%%%%%%%%%%%%%%%%%%%%%%%%%%%%%%%%%%%%%%%%%%%

% Other Settings

%%%%%%%%%%%%%%%%%%%%%%%%%% Page Setting %%%%%%%%%%%%%%%%%%%%%%%%%%%%%%%%%%%%%%%

%%%%%%%%%%%%%%%%%%%%%%%%%% Define some useful colors %%%%%%%%%%%%%%%%%%%%%%%%%%
\definecolor{maroon}{RGB}{128,0,0} %hlred
\definecolor{MAROON}{RGB}{128,0,0} %hlred
\definecolor{deepBlue}{RGB}{61,124,222} %url-links
\definecolor{deepGreen}{RGB}{26,111,0} %citations
\definecolor{ocre}{RGB}{243,102,25}
\definecolor{mygray}{RGB}{243,243,244}
\definecolor{shallowGreen}{RGB}{235,255,255}
\definecolor{shallowBlue}{RGB}{235,249,255}
\definecolor{mediumpersianBlue}{rgb}{0.0, 0.4, 0.65}
\definecolor{persianBlue}{rgb}{0.11, 0.22, 0.73}
\definecolor{persianGreen}{rgb}{0.0, 0.65, 0.58}
\definecolor{persianRed}{rgb}{0.8, 0.2, 0.2}
\definecolor{debianRed}{rgb}{0.84, 0.04, 0.33}
%%%%%%%%%%%%%%%%%%%%%%%%%%%%%%%%%%%%%%%%%%%%%%%%%%%%%%%%%%%%%%%%%%%%%%%%%%%%%%%

%%%%%%%%%%%%%%%%%%%%%%%%%% Indentation Settings %%%%%%%%%%%%%%%%%%%%%%%%%%%%%%%
\makeatletter
% Paragraph indentation and separation for normal text
\renewcommand{\@tufte@reset@par}{%
	\setlength{\RaggedRightParindent}{0pc}%1.0
	\setlength{\JustifyingParindent}{0pc}%1.0
	\setlength{\parindent}{1pc}%1pc
	\setlength{\parskip}{5pt}%0pt
}
\@tufte@reset@par

% Paragraph indentation and separation for marginal text
\renewcommand{\@tufte@margin@par}{%
	\setlength{\RaggedRightParindent}{0pc}%0.5pc
	\setlength{\JustifyingParindent}{0pc}%0.5pc
	\setlength{\parindent}{0.5pc}%
	\setlength{\parskip}{5pt}%0pt
}
\makeatother



%%%%%%%%%%%%%%%%%%%%%%%%%% Define an orangebox command %%%%%%%%%%%%%%%%%%%%%%%%
%o\usepackage[most]{tcolorbox}

\newtcolorbox{orangebox}{
	colframe=ocre,
	colback=mygray,
	boxrule=0.8pt,
	arc=0pt,
	left=2pt,
	right=2pt,
	width=\linewidth,
	boxsep=4pt
}


\newtcolorbox{redbox}{
	colframe=red,
	boxrule=0.8pt,
	arc=0pt,
	left=2pt,
	right=2pt,
	width=\linewidth,
	boxsep=4pt
}
%%%%%%%%%%%%%%%%%%%%%%%%%%%%%%%%%%%%%%%%%%%%%%%%%%%%%%%%%%%%%%%%%%%%%%%%%%%%%%%

%%%%%%%%%%%%%%%%%%%%%%%%%%%% English Environments %%%%%%%%%%%%%%%%%%%%%%%%%%%%%
\newtheoremstyle{mytheoremstyle}{3pt}{3pt}{\normalfont}{0cm}{\rmfamily\bfseries}{}{1em}{{\color{black}\thmname{#1}~\thmnumber{#2}}\thmnote{\,--\,#3}}
\newtheoremstyle{myproblemstyle}{3pt}{3pt}{\normalfont}{0cm}{\rmfamily\bfseries}{}{1em}{{\color{black}\thmname{#1}~\thmnumber{#2}}\thmnote{\,--\,#3}}
\theoremstyle{mytheoremstyle}
\newmdtheoremenv[linewidth=1pt,backgroundcolor=shallowGreen,linecolor=deepGreen,leftmargin=0pt,innerleftmargin=20pt,innerrightmargin=20pt,]{theorem}{Theorem}[section]
\theoremstyle{mytheoremstyle}
\newmdtheoremenv[linewidth=1pt,backgroundcolor=shallowBlue,linecolor=deepBlue,leftmargin=0pt,innerleftmargin=20pt,innerrightmargin=20pt,]{definition}{Definition}[section]
\theoremstyle{myproblemstyle}
\newmdtheoremenv[linecolor=black,leftmargin=0pt,innerleftmargin=10pt,innerrightmargin=10pt,]{problem}{Problem}[section]
%%%%%%%%%%%%%%%%%%%%%%%%%%%%%%%%%%%%%%%%%%%%%%%%%%%%%%%%%%%%%%%%%%%%%%%%%%%%%%%

%%%%%%%%%%%%%%%%%%%%%%%%%%%%%%% Plotting Settings %%%%%%%%%%%%%%%%%%%%%%%%%%%%%
\usepgfplotslibrary{colorbrewer}
\pgfplotsset{width=8cm,compat=1.9}
%%%%%%%%%%%%%%%%%%%%%%%%%%%%%%%%%%%%%%%%%%%%%%%%%%%%%%%%%%%%%%%%%%%%%%%%%%%%%%%

%%%%%%%%%%%%%%%%%%%%%%%%%%%%%%% MISC %%%%%%%%%%%%%%%%%%%%%%%%%%%%%%%%%%%%%%%%%%
\usepackage[acronym]{glossaries}
\usepackage{hyperref} % Setup: https://www.overleaf.com/learn/latex/Hyperlinks
\hypersetup{
	colorlinks=true,
	%citecolor=deepGreen,
	citecolor=maroon,
	linkcolor=persianBlue,
	filecolor=persianGreen,
	urlcolor=persianBlue,
	pdfpagemode=FullScreen,
}

%%%%%%%%%%%%%%%%%%%%%%%%%%%%%%%%%%%%%%%%%%%%%%%%%%%%%%%%%%%%%%%%%%%%%%%%%%%%%%%
\setcounter{tocdepth}{2}
\setcounter{secnumdepth}{2}

\newcommand{\hlred}[1]{\textcolor{Maroon}{#1}} % Print text in maroon
\newcommand{\hlgreen}[1]{\textcolor{persianGreen}{#1}} % Print text in green
\newcommand{\hlocre}[1]{\textcolor{ocre}{#1}} % Print text in green

\newenvironment{greenenv}{\color{Green}}{\ignorespacesafterend}  % Create green environment
\newenvironment{commentenv}{\color{ocre}}{\ignorespacesafterend}  % Create comment environment


\titleformat{\part}[display]
{\filleft\fontsize{40}{40}\selectfont\scshape}
{\fontsize{90}{90}\selectfont\thepart}
{20pt}
{\thispagestyle{epigraph}}

\setlength\epigraphwidth{.6\textwidth}

%\makeatletter
%\epigraphhead
%{\let\@evenfoot}
%{\let\@oddfoot\@empty\let\@evenfoot}
%{}{}
%\makeatother


%%%%%%%%%%%%%%%%%%%%%%%%%%%%%%%%%%%%%%%%%%%%%%%%%%%%%%%%%%%%%%%%%%%%%%%%%%%%%%%
%TODO LIST
\newlist{todolist}{itemize}{2}
\setlist[todolist]{label=$\square$}
\newcommand{\cmark}{\ding{51}}%
\newcommand{\xmark}{\ding{55}}%
\newcommand{\done}{\rlap{$\square$}{\raisebox{2pt}{\large\hspace{1pt}\cmark}}%
	\hspace{-2.5pt}}
\newcommand{\wontfix}{\rlap{$\square$}{\large\hspace{1pt}\xmark}}

%%%%%%%%%%%%%%%%%%%%%%%%%%%%%%%%%%%%%%%%%%%%%%%%%%%%%%%%%%%%%%%%%%%%%%%%%%%%%%%
\newcommand{\greensquare}{\marginnote{\fcolorbox{green}{green}{\rule{0pt}{3mm}\rule{3mm}{0pt}}\quad}}
\newcommand{\yellowsquare}{\marginnote{\fcolorbox{yellow}{yellow}{\rule{0pt}{3mm}\rule{3mm}{0pt}}\quad}}
\newcommand{\redsquare}{\marginnote{\fcolorbox{red}{red}{\rule{0pt}{3mm}\rule{3mm}{0pt}}\quad}}



\usepackage{glossaries}
\makeglossaries
\newglossaryentry{GenAI}
{
  name =  Generative Artificial Intelligence,
  description = {GenAI description},
}


%%%%%%%%%%%%%%%%%%%%%%%%%%%%%%% Title & Author %%%%%%%%%%%%%%%%%%%%%%%%%%%%%%%%
\title{Master's Thesis}
\author{Utku B. Demir}
%%%%%%%%%%%%%%%%%%%%%%%%%%%%%%%%%%%%%%%%%%%%%%%%%%%%%%%%%%%%%%%%%%%%%%%%%%%%%%%%

\begin{document}
\maketitle

\chapter{Introduction}\label{chap:Introduction} % (fold)

My work especially focuses on the political theory of the generative AI
algorithms.

Capitalism is distinctively characterised with the ability to decode,
deterritorialise the flows, but not to recode, reterritorialise. The
reterritorialisation in capitalism is delegated to to some institutional
frameworks, or protocologicalised processes.

\section{utku}
I will try to focus on specific bits wethout concerning about the network they
are or might be connected to.

One often urgently needed disclaimer in this cluster is that the concept of
disciplinary societies is a definition of the specific operation of
(bio-)power, speaking about control or post-disciplinary societies is not speaking about the absence or
replacement of discipline, one can freely emphasise that the control is
discipline (see \cite[]{kelly2015a} )

\section{(U) A word about institution}
Capitalism reterritorialises with one hand what it deterriorialises with the
other. But it is the delegation to specific institutional entities doing the
reterritorialisation. AI in its application is a part of the
reterritorialisation, but in its own unique structure by keeping the borders of
the socius in its own past and structure.

\cite{mackenzie2021} argument that in thw institutional framework of the
control societies, subjectivity construction operates on the basis of
dividualised subjects without the formation of a subjective centre \parencite[14]{mackenzie2021}.

This work only partly relies on the works of Antoniette Rouvroy
%TODO: Citation needed
and Gerald Raunig
%TODO: Citation needed
because both the exploration of the Algorithmic Governmentality and the
institutional formation in control societies does not include a structural
analysis, especially an analysis on the level of the algorithm itself,
neither any of them are looking for a "bit element" in the current formation of
capitalism \sidenote{Although this statement somewhat aligns with
	\citeauthoryear{mackenzie2021}'s claim, I do not agree that the 2 authors are
	not exploring the concepts they've initiated far enough. The problem rather
	lies in still not distinguished branches in the algorithmic critique, that
	would allow us to position ourselves around completely different disciplines
	and goals despite being on the same surface. } .

%TODO: Introduce the abbreviation D&G
%

\section{Cognitive Capitalism}
If this is cogntive capitalism and this is the attention economy, then we need
to immediately turn to the subjectivity construction, as it is what defines the
nature of the attention.

\gls{GenAI}

% chapter Introduction (end)
\chapter{Theoretical Framework}\label{chap:Theoretical Framework} % (fold)
One could argue that artificial intelligence is \emph{no longer an engineering
	discipline} \parencite[206]{dignum2020}, if it has ever truly been one.

\chapter{Algorithm}\label{chap:Algorithm} % (fold)
\epigraph{A computer would deserve to be called intelligent if it could deceive a human into believing that it was human.}{\textcite{turing1950}}

%INFO: Edgy much? 
%It is a much out-reaching question to consider after which threshold a
% meaning-producing entity to be called intelligent, or sentient considering
% animals and sometimes other humans were prevented to be in this category.

\section{Introduction to Algorithm}\label{sec:Introduction to Algorithm} % (fold)

As the profile of algorithms has grown and as their actions become the source of discussion, we might want to avoid thinking of them as good and bad algorithms and think instead about how these media forms mesh human with machine agency and what this means.
% section Introduction to Algorithm (end)
% chapter Algorithm (end)


% chapter Theoretical Framework (end)
\chapter{Deleuze}\label{chap:Deleuze} % (fold)

\section{IDEA:Institution}
\begin{commentenv}
	There is an idea about institutions I do not know how to explore just yet.
	The roots are in the following part of AO:


\end{commentenv}

\begin{quote}
	The two kinds of fantasy, or rather the two regimes, are therefore distinguished according to whether the social production of "goods" imposes its rule on desire through the intermediary of an ego whose fictional unity is guaranteed by the goods themselves, or whether the desiring-production of affects imposes its rule on institutions whose elements are no longer anything but drives. If we must still speak of utopia in this sense, a la Fourier, it is most assuredly not as an ideal model, but as revolutionary action and passion. In his recent works Klossowski indicates to us the only means of bypassing the sterile parallelism where we flounder between Freud and Marx: by discovering how social production and relations of production are an institution of desire, and how affects or drives form part of the infrastructure itself. For they are part of it, they are present there in every way while creating within the economic forms their own repression, as well as the means for breaking this repression.\cite[63]{deleuze1983}.
\end{quote}


\section{Desire}\label{sec:Desire} % (fold)
The history of power is the history of the regulation of desire.
%TODO: Citation needed
%
\begin{quote}
	desire is revolutionary in its essence-desire, not left-wing holidays !-and
	no society can tolerate a position of real desire without its structures of
	exploitation, servitude, and hierarchy being compromised.
	\cite[116]{deleuze1983}
\end{quote}

In Deleuze and Guattari's argumentation desire emerges as one of the most radical and transformative concepts, fundamentally challenging traditional psychoanalytic models, particularly those of Freud and Lacan. Where psychoanalysis tends to define desire through the lens of lack (a longing for something absent) D\&G reorient desire towards abundance as a framework of production and creation. Desire, for Deleuze, is not a negative force driven by deficiency but a generative, affirmative movement that actively constructs the real.
%%TODO: Citation needed
Desire is revolutionary in itself along D\&G's argumentation, "but it is
constantly being shackled, or worse converted into interest which is
suspectible to capture, domestication, and pacification"
\parencite[11]{buchanan2008}. A pure manifestation of desire without any
alteration, decoding, deterritorialisation would be enough to undermine
capitalist society's fundamental structures, but pure manifestations of desire
are rare, even in revolutionary situations \parencite[11]{buchanan2008}.



% section Desire (end)
\section{Modulation}\label{sec:Modulation} % (fold)
%TODO: Write an entry for Modulation

Picking up from Foucault's steps of civilisation, \emph{the society of sovereignity}

Deleuze's introduction of the modulation against the concept of moulding of the
discipline marks a transition from the "form-imposig mode to a self-regulating
mode" \parencite[74]{hui2015}.

% section Modulation (end)
\section{Control}\label{sec:Control} % (fold)
%TODO: When you are writing this part, rely heavily on Burroughs, which Deleuze
%does not do

\begin{quote}
	what Deleuze terms the “societies of  control” that inhabit the late twentieth century—these are based on protocols, logics of “modulation,” and the “ultrarapid forms of free-floating control”
	\parencite[86]{galloway2004}
\end{quote}

\begin{quote}
	Hardt and Negri confirm this position by writing elsewhere that “the passage to the society of control does not in any way mean the end of discipline. In fact, the immanent exercise of discipline . . . is extended even more generally in the society of control” (330)
	\parencite[83]{galloway2001}
\end{quote}
% section Control (end)
% chapter Deleuze (end)

\chapter{Undistributed sections}\label{chap:Undistributed sections} % (fold)
\section{Production}
\begin{quote}
	A means something else in the digital age. It means that we must also descend into the somewhat immaterial technology of modern-day computing, and examine the formal qualities of the machines that constitute the factory loom and industrial Colossus of our age. The factory was modernity’s site of production. The “non-place” of Empire refuses such an easy localization. For Empire, we must descend instead into the distributed networks, the programming languages, the computer protocols, and other digital technologies that have transformed twenty-first-century production into a vital mass of immaterial flows and instantaneoustransactions. Indeed, we must read the never ending stream of computer code as we read any text (the former having yet to achieve recognition as a “natural language”), decoding its structure of control as we would a film or novel.nd just as Capital begins somewhat synchronically with an examination of the commodity, then shifts into a more grounded examination of production, Empire decides also to descend “into the hidden abode of production”2 after laying out the form of imperial world order. The reason for my essay is this curious “descending into the hidden abode of production.” The hidden abode of production means many things in the age of Empire. It means descending into the real conditions of Third World chip-making factories populated by the “destitute, excluded, repressed, exploited—and yet living!” working poor (156), just as it means descending into the boardrooms of dotcom start-ups. These are material referents. Still, descending into the hidden abode of production
	\parencite[82]{galloway2001}
\end{quote}


\section{Dividuation}
\begin{quote}
	What is a dividual? A dividual is a bundle of elements held together in variation  rather than in reference to a unitary subject. Where disciplinary institutions  segmented the life-course of individuals into separate subjective roles and  functions, control modulates elements of subjectivity across the entire social field.
	\parencite[5]{mackenzie2021}
\end{quote}

\section{Saussurean Approach}
\begin{quote}
	To the extent that LLMs excel at conversation, they verify Saussure’s insight that meaning emerges from the interplay of signs in a formal system. There is no inherent need for actual sensory grounding. If “a sign stands in the place of something else” (Saussure, 1959, p. 66), then for an LLM, the “something else” could be another cluster of words, or a swirl of pixels if it is visually enabled, all existing within the confines of digital memory. Meanwhile, Peirce’s emphasis on iconic signs — signs that resemble their object — and indexical signs — signs that point to or are causally connected with their object — seems, on the surface, less relevant to an AI that navigates text tokens rather than the physical world. Without a body to roam or eyes to see, the Peircean structure appears incomplete inside the machine’s domain.
	\parencite[]{phd2025}
\end{quote}


\section{Language in LLM}

\begin{quote}
	It represents nothing, but it produces. It means nothing, but it works.
	Desire makes its entry with the general collapse of the question "What does
	it mean?" No one has been able to pose the problem of language except to the
	extent that linguists and logicians have first eliminated meaning; and the
	greatest force of language was only discovered once a work was viewed as a
	machine, producing certain effects, amenable to a certain use.
	\cite[109]{deleuze1983}
\end{quote}

\section{Transition to GenAI}\label{sec:Transition to GenAI} % (fold)
%INFO: This section is all about the transition from the AI algorithms that
%profile associate relevance to those producing "meaning"

An in-depth analysis of the steps in development of AI models is rarely
meaningful in its whole extention outside of the areas dealing with AI development.
While only few relatively distinctive steps had a relevance for the
uninitiated, also, some categorical differences might sound arbitrary
\sidenote{
	E.g. difference between emph{neural networks} to \emph{deep neural networks},
	or \emph{language model} to \emph{large language model}.
}
%TODO: DO we need the previous paragraph?
%TODO: Introduce an AI categorisation

Previous AI algorithms that assesed relevance and association.
%TODO: Take the AI definition from the BA

The GenAI evolves the AI operation from the association of context, and
association of agents into the \emph{meaning creation}
\parencite[964]{dishon2024}.

% section Transition to GenAI (end)
\section{Pre-Training to Fine-Tuning}\label{sec:Pre-Training} % (fold)
%TODO: Refer to @dishon2024 on thiese

\begin{quote}
	Here, it is vital to distinguish two stages in GenAI model development: pre-training and fine-tuning. During pre-training, GenAI learns from vast datasets to identify patterns and generate similar content. This stage is marked by its unstructured approach and unpredictability. In fine-tuning, the model’s capabilities are tailored by applying targeted reinforcement learning (usually via human feedback) to refine and direct its behaviors towards more specific contexts and outcomes \parencite[964]{dishon2024}
\end{quote}



Generative AI marks a transformative shift, as algorithms now transcend mere analysis and refinement, stepping into the domain of crafting independent representations—an area traditionally regarded as the unique province of human creativity and expression.
% section Pre-Training Fine-Tuning(end)

\section{From external force to meaning-making}\label{sec:From external force to meaning-making} % (fold)

% section From external force to meaning-making (end)
Dishon (\citeyear{dishon2024}) underlines it as the transition from a Frankesteinian
imaginary to a kafkaesque process meaning-creation stallmate.

%TODO: Explain AI imagineries with the help of Dishon
→ Thus, artificial life is portrayed as a discrete entity, largely mirroring
human agency. In the Frankenstein imaginary, the construct is external to the
what is human with human-like abilities. The concern of agency in this case
reflects the possibility of the construct taking over.

\begin{quote}
	Though AI is depicted as possessing superior capabilities—both physically and mentally (the creature quickly learns to talk and read)—the overall logic governing pattens of meaning-making remains stable. This overall similarity serves as the background according to which certain differences can be appreciated and highlighted. Hence, it is not so much the rationale of agency or meaning-making that shifts, it is the actor who holds the privileged author position—the fear that machines will replace humans is embedded within current structures of meaning-making \parencite[966]{dishon2024} .
\end{quote}

This is an anthropomorphic depiction of AI, and similarly associated with an
anthropomorphic threat. The threat is as much as it is in depiction a discrete
one. An external agent is getting in contact with the human manifesting
anthropomorphic qualities, and the threat is either perfecting or advancing beyond
anything anthropomorphic nature. The discrete threat introduces immediate
concerns about humans' responsibilities of their own making. The question
becomes one of survival, or at the very least about losing humanity's place on
the \emph{food chain}. The concern is that the AI can develop human desires,
the threat is derived from AI possibly being too much of human nature than vice
versa.

%TODO: Introduce the Trial by Kafka analogy here
Dishon \citeyear{dishon2024} puts the process of Kafka's Trial as a counter
analogy here. Contrary to the anthropomorpistic imlplications of the
Frankenstein analogy, the court of the \emph{Trial} "does not have a
well-defined system or a reference to \emph{truth}" \parencite[970]{dishon2024}. It is
completely driven by the qualities, interactions, and commitment %TODO: commitment?
of the subjectivity in interaction. The source of meaning-making in this sense
is completely free of any form of agency. Moreover, as in the interaction of K. with the court
the mechanism is working against any identifiable agency of the interactor
themselves \parencite[see 970]{dishon2024}. The operation is rather blurring
the existing agency itself instead of producing one.

The Court contradicts the externality of the Frankenstein metaphor in its
nature of interaction with the subjectivity engaged with it. The interaction
with The Court is necessary, but the effect of this interaction seems to be
futile.

\begin{quote}
	Thus, paradoxically, the best way to arrive at a definite outcome is to make sure the process never ends. Interactions with the court are necessary and require constant maintenance, yet they cannot be controlled, predicted, or even expected to progress towards a resolution. The novel could be viewed as turning the meaning of a trial upside down—it is not meant to arrive at a verdict or outcome, rather it is the process itself on which one must center. [...] The Trial offers a depiction of control that relies on a different understanding of meaning-making—shifting from a stable and general model of meaning to an idiosyncratic and personalized one.

	-- \textcite[970-972]{dishon2024}
\end{quote}

The meaning-making process K. is subjected to asking for his submission to the
produced truth as it is tailored exactly for him, and there is no specific way
to test its genuinity. And the product of the process seems to be signifying an
inaccessability. This is a shift between internal and external aspects of
agency, a blurring in between layers as well as the nature and perception on
both sides' potential agency.

Through kafkaesque analogy
%TODO: Other analogies?
GenAI becomes an internal agent instead of external entity leads to a blurring
of agency and machinic intentionality. The human's struggle in the process is
to constantly having to assess, to decipher AI's agency. Furthermore, the GenAI
model is not necessarily trying to offer a right answer but to generate content
without necessarily relying to veracity or accuracy leading to an amplification
of human's endless search for meaning \parencite[see
	977]{dishon2024}.

\begin{quote}
	The Trial is not about humans losing control over their creations, if they ever had control in the first place. Instead, it foreshadows GenAI’s capacity to generate content that is personalized to every actor (and thus shaped by humans) yet is not amenable to control through explicit choices. This model of meaningmaking undermines the dichotomy between choice and coercion, no longer positioning the two as mutually exclusive. [...] Therefore, as in Before the Law, personalization does not inherently entail increased control over meaning-making, but rather its increased mediation according to what GenAI identifies, or perhaps determines, as our personal preferences.

	-- \textcite[974]{dishon2024}
\end{quote}

GenAI offers in this sense an endless variation of meaning specifically
personalised. What does it mean in term's of human's choice?
%TODO: Does it really personlaise answers?
%

\section{Instance}\label{sec:Instance} % (fold)
%TODO: For example any subjectivity is an instance on a network of partical
%objects (or larval subjects) -> Refer to the chapter 1 in both buchanan2008
%and AO

% section Instance (end)
\section{Enregistrement \parencite[4]{deleuze1983} }\label{sec:Enregistrement \parencite[4]{deleuze1983} } % (fold)
%TODO: Refer to the double entendre of French meaning in the footnotes. It
%explains the BwO pretty well.
%

\section{Process}\label{sec:Process} % (fold)

% section Process (end)


% section Enregistrement \parencite[4]{deleuze1983}  (end)

% chapter Undistributed sections (end)
\printglossaries
\printbibliography
\end{document}

































