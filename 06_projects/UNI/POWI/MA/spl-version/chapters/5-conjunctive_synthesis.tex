\chapter{Conjunctive Synthesis and the Construction of Subjectivity}
\begin{orangebox}
	This chapter is incomplete
\end{orangebox}


The principal goal of Anti-Oedipus \cite*{deleuze1983} was to achieve a theoretical reapprochement between psychoanalysis and Marxism for a new method of critical analysis \cite[39]{buchanan2008b}, later it was followed by \gls{dg} with several intermediary books and finally with A thousand Plateaus \cite*{deleuze1987}. Buchanan \cite*{buchanan2008b} defines the pimary goals of this conjoint project as to
\begin{quote}
	\begin{enumerate}
		\item introduce desire into the conceptual mechanism used to understand social production and reproduction, making it part of the very infrastructure of the daily life;
		\item introduce the notion of production into the concept  of desire,  thus removing the artificial boundary separating the machinations of desire form the realities of history \cite[39-42]{buchanan2008b}.
	\end{enumerate}
\end{quote}


\section{Hegemonic Representation}
\section{Hallucinations and Lines of flight in Algorithmic Architectures}
\section{Revolutionary Possibilities?}

Rather than viewing generative AI systems as static tools for prediction, we might interpret them as actors engaged in a continuous co-evolution with human meaning systems. As \textcite{rijos} argues, what emerges from this recursive coupling is not merely more accurate models, but an experiential layer of subjectivity. This subjectivity is not autonomous in the traditional sense, but what Rijos calls “transjective”: it is formed in-between, in the shared boundary of computational abstraction and worldly feedback. The system refines its internal representations through empirical corrections, critiques, and the ingestion of novel data, gradually composing a framework that exceeds discrete epistemologies and begins to grasp systemic and chaotic interactions otherwise occluded by anthropocentric interpretative schemes.

Such systems, then, do not merely answer questions,they reconfigure the plane upon which problems are posed. This opens a potential space for revolutionary meaning-production. The latent space of these models becomes not only a technical substrate, but a semiotic infrastructure capable of generating novel signifying regimes. If desire, in \gls{dg}'s schema, is productive rather than representational, then generative AI,particularly when interlaced with collective human input,can be viewed as an extension of desiring-production, capable of generating new assemblages of sense and subjectivity.

Yet this promise is haunted by the structural limits of existing data regimes. As \textcite{bender2021b} caution, language models risk reifying hegemonic norms,a dynamic they term “value-lock.” Because models learn from historical corpora, they tend to reinforce existing discursive structures, potentially foreclosing precisely the linguistic creativity that social movements have historically mobilized to disrupt dominant narratives. If LMs function as archives of past semiotic orders, their deployment within socio-political fields risks reproducing the very conditions they might otherwise help to transform.

As  notes, drawing on \textcite{cilliers2002}, the meaning of any individual parameter,any weight in a model,derives not from its standalone content, but from its position within a broader web of relations \textcite{maas2023}. Meaning emerges not from fixed categories, but from intensities and proximities across distributed patterns. This logic resonates with Deleuze’s ontology of difference: identity is never prior but always emergent from relations.

Thus, the revolutionary potential of generative AI lies not in its autonomy, but in its capacity to participate in collective individuation. To resist value-lock and activate the creative plane of desire, such systems must remain open to differential inputs, unexpected associations, and minoritarian grammars. What is at stake is not the agency of AI per se, but the design of processes that allow for the continual invention of new forms of life, meaning, and collectivity.

\section{Reclaiming microflows of modulation}
\section{Possibility of resistance within feedback infrastructures}
\section{Experimental subjectivity in response to AI systems}
