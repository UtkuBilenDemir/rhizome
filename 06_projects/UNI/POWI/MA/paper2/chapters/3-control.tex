\chapter{Subjectivity and the Shift from Discipline to Control}

\begin{quote}
	The society of control is characterized not by the power of the institutions of modernity, or  pre-modernity, the army, the prison, the university, the church, but instead by what he Deleuze called "the ultra-rapid forms of free-floating control that are  inherent in distributed networks".

	— \cite[318-319]{galloway2004}, \cite{mackenzie2021}
\end{quote}


\marginnote{\textbf{TODO}:
	\begin{todolist}
		\item Introduce the critique of the Postscript, also reflect on
		\cite{mackenzie2021}
		\item "Every regime introduces some form of governance of desire"


		\item tell how control replaces the institutional formation of Foucault


		\item add Burroughs: Burroughs never thought about the control as
		bidirectional thing (mayan shaman citation)

		\item \textbf{To be completed with following}:
		Disciplinary inst. -> Control -> Postscript -> Modulation
		\item \gls{genai}  as an institution


		\item But how does the AI relate to the subjectivisation etc.?

	\end{todolist}
}


Deleuze's short and speculative essay, \citetitle{deleuze1992a} \parencite*{deleuze1992a}, introduced a fragmentary but generative diagnosis of late-modern power structures. It describes a social assemblage with a specific machinery, an institutional framework constructed via the digital technologies, governed via the functionalities made available by computational advancements. A computational dispositif that do not merely maintain or articulate, but also shape the political, and economic architectonics; realising the shift into a more fluid and flexible operation of power. It is an uncanny tendency to associate the current in and around the novel computer technologies, it is reaching beyond the plain understanding of \textit{cyberspace} or the \textit{virtual}, we are dealing with artificial systems capable of meaning-making (see \cite[]{kazakov2025, dishon2024}). \Gls{ai} systems are mediating how we structure the reality while expanding their existence both in terms of capability, their spectrum of actions and their presence in vastly differing areas of operation (see \cite[]{kazakov2025}). The current \gls{ai} scene is exposed to an accelarating domination by \gls{genai} models, and particularly \glspl{llm} operating on a strict paradigm of expansion and rapid scaling model features, increasing data ingestion, and allocation of more and more resources. It is a matter of debate if the rapid accelaration in every aspect of the \gls{ai} development is also translated into the development towards a more sophisticated future with \gls{agi}, singularity while we are still debating what human-like intelligence means, but nonetheless, these models are getting more capable, comprehensive %under performing their intended functions 
and becoming more present by the day.

\redsquare
Kazakov \parencite*{kazakov2025} situates the contemporary \gls{ai} development within what he calls scalar Darwinism: a phase of development defined by relentless quantitative expansion rather than qualitative transformation. \Glspl{llm}  and other \gls{genai}  systems advance primarily by scaling; more parameters, larger datasets, greater computational resources, without fundamental architectural innovation. This mode of growth reinforces existing capitalist logics, where data is treated as a resource to be extracted and leveraged, and market advantage derives from size rather than capability. As in Michel Foucault's (see \cite*[131]{foucault2008}) definition of neoliberal governmentality whereas \textit{the market} claimed to be the methaphysical plane producing the best possible solutions for diverse issues without any \textit{directional} intervention, in the age of rapid \gls{ai} development the scaling of the data and models claimed to be paving the ways to the ultimate solutions humanity direly in need of \sidenote{
	Not always this obvious, but the technolsolutionist propagation is especially strong in tech. development fronts:
	\begin{minipage}{0.9\marginparwidth}
		\centering
		\includegraphics[width=\linewidth]{images/musk2.png}
		\vspace{0.3em} % Optional spacing
	\end{minipage}

	— \cite[]{musk2025}
}.


\yellowsquare
Whether the truth abouth our world is contained in the data waiting for the
right model to extract the correct distribution, the \gls{genai} models have
already shown surprising capabilities. While, especially \glspl{llm} often
defined as simple next word predictors (see e.g. \cite[]{dalvi2025}), or
plagiarism machines (see e.g. \cite[]{chomsky2023}) \gls{genai} models continue to
show surprisingly well performance in different areas of expertise (see e.g. \cite[]{sultanow2024}). These surprising performances are not simply technical feats; they signal a qualitative shift in how artificial systems engage with reality. Modern \glspl{llm}  and other \gls{genai} models no longer operate solely as analytic tools that classify or retrieve as their previous \gls{ai} model counterparts; they actively participate in the production and reproduction of reality by governing the flow of information, assisting various purposes of knowledge creation, generating outputs that are treated as meaningful, actionable, and often authoritative (see e.g. \cite{montanari2025, dishon2024}). While their presence is causing a profound transformation in the ecology of information and the mediation of human knowledge, they structure what is visible and legible, filter which forms of knowledge circulate, and frame how subjects encounter and relate to information. In this sense, the processes of subjectivisation today is tightly bound to this new mode of algorithmic meaning production. Other modes of information mediation is increasingly replaced by the \gls{genai} models, To understand the societal and political implications of \gls{genai}, we must first interrogate its substance of subjectivisation of these meaning making entities as distributed institutions over information.

%this leads to the analysis of these meaning making entities as distributed institutions over information, where probabilistic models do not only organize and rank content but also shape the conditions under which meaning appears. %This institutional dimension is not peripheral, it is central to how contemporary power operates. %As these systems mediate perception, govern flows of communication, and gradually normalize their outputs as authoritative, they take on a role analogous to, yet fundamentally different from, classical institutions like the school, the archive, or the newspaper. They enact what Deleuze would describe as continuous modulation: an ongoing recalibration of knowledge, behavior, and expectation that operates at infrastructural depth rather than within enclosed spaces \parencite{deleuze1992a, mackenzie2021}.
%To understand the societal and political implications of \gls{genai}, we must first interrogate its institutional substance.

\marginnote{\textbf{TODO}:
	\begin{todolist}
		\item Last part is quite weak, update

		\item Write a better transition

	\end{todolist}
}


\section{Subjectivity}

\begin{quote}
	Knowledge is not innocent. It is not so much about discovering the truth, but rather about producing certain truths; it produces objects, or subjects, like the delinquent, and thus spaces and strategies of intervention. The individual, as Foucault has it, became knowable and thus accessible to power.

	— \cite[11]{Krasmann2017}
\end{quote}

\begin{orangebox}
	\textbf{CCC}
\end{orangebox}


\greensquare
The question of subjectivity; its emergence, its production has long haunted Western thought, morphing with shifts in different branches of philosophy like epistemology, and metaphysics. Its philosophical genealogy stretches back to ancient concerns with soul and selfhood (see e.g. \citeauthor{aristotle1986}'s \citetitle{aristotle1986} \cite*{aristotle1986}), but its modern formulation takes decisive shape with Descartes’s cogito, which installs the thinking subject as the indubitable ground of knowledge \parencite{descartes2008}. From there, Kant’s \textit{Copernican revolution} redirects philosophical inquiry toward the a priori structures of the subject that condition possible experience \parencite{kant2009}. These moves solidified the notion of the autonomous subject as the bedrock of Enlightenment thought, closely linked to emerging political imaginaries of agency, reason, and rights \parencite{taylor1989}. Yet even in these rationalist formulations, tensions persist around the relationship between the interiority of the subject and its formation through language, culture, and material institutions.

\greensquare
It is these tensions that would later be unraveled by structuralism and post-structuralism. Structuralists such as Lévi-Strauss, Saussure, and Althusser shifted attention from interior experience to the impersonal systems; language, myth, ideology that precede and produce subjectivity \parencite{levi-strauss1963, saussure2011, althusser1977}. The subject, in this sense, becomes an effect of signifying structures; it is interpolated by ideological apparatuses and made legible within symbolic orders. Post-structuralist thought does not simply reverse this move but radicalizes it: in the theories of Barthes, Derrida, and Kristeva foreground the instability, iterability, and difference at the heart of these structures themselves \parencite{derrida2016, barthes1977, kristeva1980}. In post-structuralism, the autonomous subject of modernity dissolves into the relational field of discourse and social practice. What appears as “subjectivity” is only an instance, a provisional effect emerging from the entangled formations of language, power, and social structures. In this sense, there is no pre-given subject behind the text; the subject exists only as a position produced within and by the text of the social. Subjectivity, in this view, is neither natural nor given; it is produced, fractured, and dynamic. Institutions, in turn, do not merely constrain subjects but participate in their ongoing fabrication. This shift opens the way for analyzing how subjects are continuously assembled across milieus like for example linguistic, institutional, technological surfaces \sidenote{\textbf{NOTE}: Not quite sure about this prologue, consider removing.}

\redsquare
Post-structuralist accounts of power challenge the assumption that subjectivity is an original or pre-social essence. Instead, they situate the subject as a product of institutionalised practices that operate across everyday life. From the school to the prison, the factory to the family, institutions do not simply govern behaviour; they organise routines, structure spaces, assign functions, and produce bodies capable of carrying out specified functions. To become a subject is to be made visible and intelligible; seen, heard, corrected, and positioned within patterned social routines. These processes are not merely ideological; they are materially embedded in environments that stabilise what kinds of subject-positions are possible. Spatial arrangements such as the classroom desk, the assembly line, or the domestic threshold do not simply house activity; they condition the very terms under which individuals come to know themselves and others. What emerges is a subjectivity that is not interior or fixed, but modulated through institutional dispositifs that shape the conduct, perception, and agency of the self over time \parencite{foucault1995, hardt1998}.

\greensquare
In Foucault's description of \textit{careful fabrication of the subjectivities} \parencite[215]{foucault1995} the institutions were formed as efficient enclosures operating on surveillance machinery to engrave specific forms of subjectivities on bodies (see e.g. \cite[783]{foucault1982}). This forging process on bodies in disciplinary societies was allowing to delegate the correction of behaviour to the individuals themselves through being exposed to a \textit{constant state of conscious and permanent visibility} \parencite[202-203]{foucault1995}. The disciplinary machinery appears as an effective and economic machinery in comparison with the operation of power in the previous \textit{sovereign societies} whereas the power was imposed on the bodies through public spectacle, punishment, and sovereign's right over subjects' lives; with Deleuze's words \textit{to tax rather than to organize production, to rule on death rather than to administer life}\parencite*[3]{deleuze1992a}. Discipline, the successor of the sovereign power, instead, inverts the gaze in the subjects, deploys the knowledge gathered over bodies and works as a \textit{political technology of the body} \parencite[26]{foucault1995}; it operates in a distributed manner, without forming monolithic centres emitting the power. Yet, constantly active in its diffuse structure, nests in societal functions and practices, manifested extended by its own subjects, a \textit{microphysical manifestation of power} (see \cite[26-27]{foucault1995}).


\yellowsquare
In disciplinary societies, as Foucault famously outlined, power is exerted through the molding of individual bodies and behaviors via normative institutions \parencite{foucault2008}. The logic was architectural and corporeal: subjects were shaped within bounded spaces, subjected to surveillance, and trained into docility through routines and assessment. Control societies, by contrast, operate through the flexible recomposition of dividual data traces,discrete fragments of identity, behavior, or preference,that can be extracted, modelled, and recombined by algorithmic infrastructures. This understanding frames the conceptual shift that underpins the present analysis: the transition from disciplinary societies to control societies. Both operate as mechanisms for the production of subjectivity, but they differ in how they structure space, time, and the flow of power.


\section{Postscript: Updating the Societies of Control}

\marginnote{\textbf{TODO}:
	\begin{todolist}
		\item \cite{hardt1998, brusseau2020, mackenzie2015, galloway2001, kazakov2025, hui2015}

	\end{todolist}
}



\begin{quote}
	You see control can never be a means to any practical end\ldots It can never be a means to anything but more control\ldots Like junk\ldots

	— \cite[81]{burroughs1979}
\end{quote}

%INFO:\subsection{1. From Disciplinary to Control Societies}

Following Michel Foucault’s genealogy of power, from sovereign societies to disciplinary regimes, Gilles Deleuze introduces a third historical configuration: \textit{the society of control} \parencite[]{deleuze1992a} . Deleuze points to a crisis in disciplinary institutions and starts charting the replacement of enclosed institutional spaces; schools, factories, prisons, but also family\sidenote{ Which \gls{dg} were eager to emphasise its institutional nature of (see \citetitle{deleuze1983} \cite*{deleuze1983}).}  by diffuse, pervasive mechanisms of flexible forms of control. The analysis starts with a historical account to draw the difference between the discipinary and control societies, Deleuze notes that the 20. Century marks the transition form one society to another and the disciplinary institutions were already fading out after the WWII \parencite[3]{deleuze1992a}.
While disciplinary regimes operated through enclosures, segregating individuals into clearly defined spaces associated with specific functions, contemporary forms of control rely on more fluid mechanisms; instead of physical boundaries, social organisation is achieved by tracking, directing, and modulating movement and behaviour across interconnected and permeable environments (see \cite[3]{brusseau2020}). \textit{The walls of the institutions are breaking down in such a way that their  disciplinary logics do not become ineffective but are rather generalized in fluid  forms across the social field} \parencite[139]{hardt1998}.

%INFO:\subsection{2. Control as Modulation and Continuous Surface, The Dividual and Datafied Subjectivity}

%%%\begin{orangebox}
%%%
%%%
%%%	The shift from disciplinary societies to societies of control, as articulated by Deleuze \parencite{deleuze1995a}, marks a transformation not only in the mechanisms of governance, but in the infrastructures through which power circulates. Where disciplinary institutions,prisons, schools, hospitals,enclosed individuals and shaped them through spatial and temporal segmentation, control societies operate through continuous modulation. The walls of the disciplinary institutions are breaking down, their disciplinary logics do not become necessarily ineffective, but are rather generalized in fluid forms across the social field; the striated space of the institutions turns into the smooth space of the societies of control \parencite[139]{hardt1998}. Power becomes immanent to processes of circulation; it no longer functions by enclosure but by coding and recoding flows of information, behavior, and subjectivity in real time.
%%%
%%%\end{orangebox}
%%%
%%%
%%%Control, a term Deleuze coined from \citeauthor{burroughs1979} \parencite*[]{burroughs1979},  in his formulation, signifies both an institutional shift and a fundamental change in the production of subjectivity. Disciplinary institutions were operating on discrete subjectivites \textit{molded} in individual enclosures, the individual jumps from one enclosure to another. The institutions of the disciplinary societies are discrete entities, \textit{independent variables, in each of the enclosures one has to start from zero although there is a common language} (see \cite[4]{deleuze1992a}); family, school, factory etc., all play into the training of the subject with their distinct contextes with equivalent machinery. Control  on the other hand introduces a smooth surface of continuous adjustment through sophisticated unification of the knowledge on bodies; databases, ubiquotus computing\sidenote{Control and its modulating nature are not necessarily digital, the terms are referring to a specific operation that became the new paradigm of capitalism. Deleuze finds a characterisation of this phenomenon in the transition from the factory to corporation structure (see \cite[6]{deleuze1992a}), one that seemingly gives more freedom to the worker on one side, but control the flow of work with other means.
%%%	%, or from focusing from production to unified interest on trading products, stocks, a rapid exchange of commodities for surplus as a direct form of Marx' \(MCM'\) formula without any production. 
%%%	However, this stage is very much characterised with the apparatuses that enable rapid, flexible, highly adaptable and adapting forms of operation, hence the emphasis on the digital means that enable the operation.}. The individual is not jumping between fixed identities the institutions characterised with, instead she is split into dividuals, data particles, micro traces, and acted upon through these partial qualities.  The parsing and recomposition of dividuals across data flows (see \cite[4]{deleuze1992a}) powered by advanced computational, statistical inference are the ground for a \textit{personalised} output, evaluation, feedback, response, and planning like a \textit{self-deforming cast} adjusted specifically and dymanically (see \cite[4]{deleuze1992a}). Docility in this new regime is no longer imposed through the explicit codes of an institution. Instead, control operates by creating a space that feels open and permissive, as if the individual is free to tangle, to explore, and to create. Yet both the processes of production and the ends they serve are subtly guided by intangible, underlying forces (see \cite[75]{hui2015}).  One doesn't have to begin from the start in every single enclosure in control societies, the analogical distinctions between the institutional spaces are no more, they all converge towards one  (see \cite[6]{deleuze1992a}) and one is never finished with any of the spaces of subjectivisation.
%%%
%%%\marginnote{\textbf{TODO}: Title
%%%	\begin{todolist}
%%%		\item The end seems weak
%%%
%%%	\end{todolist}
%%%}
%%%
%%%\begin{orangebox}
%%%	A parallel tension animates the role of \textit{modulation}
%%%	, a concept central to Deleuze's diagnosis. Drawing on Simondon, Hui \parencite*{hui2015} traces modulation beyond its repressive deployment, uncovering its ontological roots in processes of individuation. Modulation, in Hui's reading, is not inherently co-opted by control; it remains a contested terrain, one that can either reinforce algorithmic governance or open pathways for new collective forms.
%%%\end{orangebox}

%%%The shift from disciplinary societies to societies of control, as articulated by Deleuze \parencite{deleuze1995a}, marks a transformation in both the mechanisms of governance and the infrastructures through which power circulates. Disciplinary institutions, once enclosed individuals and shaped them through spatial and temporal segmentation. Control societies, by contrast, operate through continuous modulation: the walls of institutions break down, and their disciplinary logics are not abolished but generalized into fluid forms across the social field \parencite[139]{hardt1998}. Power becomes immanent to processes of circulation; it no longer functions through strict disciplinary planning in enclosures but through the coding and recoding of flows of information, behavior, and subjectivity in real time.
%%%
%%%Control, a term Deleuze borrows from \citeauthor{burroughs1979} \parencite*[]{burroughs1979}, signifies both this institutional shift and a fundamental change in the production of subjectivity. In disciplinary societies, institutions were discrete places, each molding the individual anew according to their own nature of training: family, school, factory, and prison operated as independent sites of training, with a shared language but segmented machinery \parencite[4]{deleuze1992a}, the process in one bare got translated into the life in the other one. The subject moved between them in sequence, starting from zero in each.

The shift from disciplinary societies to societies of control, as articulated by Deleuze \parencite{deleuze1995a}, marks a profound transformation in the mechanisms of governance and the infrastructures of power. In disciplinary societies, power functioned through discrete institutions; family, school, factory, and prison etc. each molding individuals within enclosed spaces, imposing fixed norms, and segmenting life into sequences of spatial and temporal regimes \parencite[4]{deleuze1992a}. Movement between institutions required the subject to start anew in each, as the machinery of discipline operated in parallel but remained compartmentalized. Control, a term Deleuze borrows from \citeauthor{burroughs1979} \parencite*[]{burroughs1979}, signifies both this institutional shift and a fundamental change in the machinery of subjectivisation. A new post-disciplinary dispositif powered by the emerging forms of computational advancement whereas power becomes immanent to processes of circulation; it no longer functions through strict enclosure but through the real-time coding and recoding of flows of information, behavior, and subjectivity. \textit{Modulation}, a concept Deleuze introduces instead of the molds that characterise disciplinary instituions, is central to this transition. Disciplinary \textit{molding} imposes form from fixed sites, \textit{modulation} is a flexible, dynamic regime that enacts control through adaptation via continuous feedback; a shift from a \textit{form-imposing mode to a self-regulating mode} \parencite[74]{hui2015} in the production of subjectivity. In this sense, control is not simply the replacement of discipline but its generalization and internalization within the continuous processes that traverse the social field.

The unitary subject operating in different designated roles in disciplinary institutions is going through a different kind of splitting this time. The self in control societies is unified across the smooth surface of control society but the individual's qualities and behaviour is getting analysed an and acted upon in smaller separated parts (see \cite[5]{mackenzie2021}) . The individual is broken down, fragmented into dividuals; data particles, micro‑traces, partial qualities circulating across digital and organizational infrastructures \parencite{deleuze1992a}, this \textit{bundle of elements held together} replaces the unitary subject \parencite[6]{mackenzie2021}.
\sidenote{Control and its modulating nature are not necessarily digital, the terms are referring to a specific operation that became the new paradigm of capitalism. Deleuze finds a characterisation of this phenomenon in the transition from the factory to corporation structure, where the site of production is replaced by an abstract field of work with rapid changes in definition of the salaries, and the work itself surrounded by a \textit{spirit} of the work place. Similarly the never ending education is another indicator Deleuze mentions, the education is not coming to an end in the enclosure of scholl but transendences into all branches of life (see \cite[6]{deleuze1992a}). Nonetheless, the modulation is characterized by the machinery that enables discpline to transform itself to this fluid form of control, hence the emphasis on the novel operation and digital aspects of control.} Through databases, ubiquitous computation, and advanced statistical inference, these dividual traces are parsed, recomposed, and acted upon, generating personalised evaluations, outputs, and interventions. The effect is akin to a self‑deforming cast, continuously adjusting to the subject in motion \parencite[4]{deleuze1992a}.

Docility under this new regime is no longer enforced by explicit institutional intentionality, operationalized as rigid codes. Instead, control operates by creating spaces that feel open and permissive, as if the individual were free to explore, create, and tangle with possibilities. Yet both their production and its ends are subtly governed by intangible, underlying forces \parencite[75]{hui2015}. In this sense, control converges the previously separate spaces of subjectivation into a single, fluid field: one no longer leaves an institution behind, and one is never fully done with the spaces that act upon the self \parencite[6]{deleuze1992a}. In a dividualised society, the masses are not analysed as collection of individuals anymore, partial dividual characteristics open trajectories to analyse collevtive behaviour on; the variables are biometric information, social media history, purchasing tendencies, political leaning, voting patterns, and other elements of subjectivity, a new way of grouping working in the logic of marketing (see \cite[7]{deleuze1992a}).


%Deleuze claims that the stretching, continuation of education in other parts of the life is a form of control

% \subsection{3. Immanence and the Extension of Discipline}
Control does not abolish discipline; its difference is bound in intesification and hypereffectivity. As Hardt and Negri \parencite*{hardt2003}  observe, \textit{the passage to the society of control does not in any way mean the end of discipline. In fact, the immanent exercise of discipline [...] is extended even more generally in the society of control} \parencite[83]{galloway2001}. This immanence is central: control is no longer imposed from above but embedded within the continuous flows of communication, code, and affect. It operates through protocols, feedback loops, and algorithmic infrastructures, an open-ended regime of governance that infiltrates the very capacities of subjects to act, perceive, and desire
\sidenote{

	\begin{todolist}

		\item use \cite{cheney-lippold2024} to talk about dividuals

		\item Reconsider
	\end{todolist}

}. Yet while Foucault never postulated a stage beyond disciplinary societies, Deleuze’s Postscript on the Societies of Control offers only a sparse sketch of what comes after enclosure-based institutionalisation. As Hardt \parencite*[139]{hardt1998}  points out, Deleuze says remarkably little about the institutional architecture of control societies themselves; the form remains vague, its contours merely suggested through keywords like \textit{modulation} or \textit{dividuation}. We are being exposed to a fundamental change in the training of subjectivity, but the implications of the new operational novelty, as well as, it is new forms and machinery are only partly explored. Although, we are having a glimpse into a world governed by code and the \textit{spirit} of corporation, the question about the operative principles especially regarding the emerging novel machineries remains to be explored.




\section{\Gls{ai} as Institutional Framework}
How can we operationalise the institutional analysis of the control societies? How can we expand the exploration to the \gls{ai} models governing and creating or patch-working information? There are no shortage of adaptations of Deleuze's short account to contemporary advancements. Brusseau extends these debates into the era of big data and predictive analytics, where the logic of control intensifies through hyper-personalized algorithmic environments \parencite{brusseau2020}. Predictive technologies, entwined with generative AI systems, instantiate what Hui terms \textit{disindividuation}: the fracturing of subjectivity into calculable, governable fragments.

\marginnote{\textbf{TODO}:
	\begin{todolist}
		\item Could be better to introduce this above, when we are talking about
		the dividuals

	\end{todolist}
}



\begin{orangebox}
	Yet, as MacKenzie and Porter emphasize, such developments also provoke new modalities of critique. Their notion of \textit{counter-sequencing}, the rearrangement or disruption of institutional sequences,suggests avenues for resistance that do not rely on outdated ideals of autonomous subjectivity but engage the very logics of dividuation and modulation from within.
\end{orangebox}

The imaginary of the computational future had a discrete nature in literature, the
algorithmic governance was much more about blocking flows, denying access and
keeping boundaries intact. The central question Deleuze asks is \textit{how can
	there be control if nothing is forbidden?} \parencite[2]{brusseau2020}. The
answer to the question with the predictive analytics; data-driven marketing and social media strategies that regulate through incentives, soft control over the flow of consumers, recommendation systems, filters, and relevance associations; not necessarily a containment, no blockage, no enforcement, but correction through personal information, profiling, anticipation (see \cite[2]{brusseau2020}). Now, we have another medium to analyse in the same manner, one that is able to talk back.


While Deleuze's account remains foundational, its brevity has spurred diverse and sometimes conflicting interpretations. As MacKenzie and Porter \parencite*{mackenzie2021} observe, much of the subsequent literature has overemphasized technological dimensions, portraying control as an exclusively computational or algorithmic phenomenon, detached from institutional life.


\begin{orangebox}


	Yet, they argue, institutions have not disappeared; rather, they have been transformed into totalizing structures that sequence and redistribute dividuals across domains. This process of sequencing constitutes a key mechanism by which control operates in contemporary society, bridging the technological and institutional logics.

\end{orangebox}



\section{Connection to AI}



\begin{quote}
	Power brings the subject into being, but power does not exist independent of its enactment. It is immanent and only takes shape at a point of resistance.46 The subject is such a point of resistance that recasts, redirects and sometimes reverts power. Subjectivation, however, always involves wrestling with oneself; it is governing the self and self-government: the subject is bound to power as it is to him- or herself.47 How then to conceive of a political subject as a fold of power as well as a “line of flight”? 48 How to imagine a challenge to the current regime of visibility?

	— \cite[18]{Krasmann2017}
\end{quote}

In modulation, power is no longer exercised through strict categories or final forms but through elastic processes that track, nudge, and reshape behavior in real time. This logic is foundational to the algorithmic infrastructures of contemporary societies, especially those driven by \gls{genai}, where subjectivation occurs not through fixed norms but through continuous calibration against probabilistic expectations. In this regime, what is governed is not the subject as a stable identity, but the flow of tendencies, preferences, and predictions.



\section{Turn in Capitalism and its institutions}
With \citetitle{deleuze1992a}, Deleuze is not (just) trying to define the
framework of a series of computational advancements, he is referring to a new
turn in capitalism. One that immediately to be observed also by the methodology
and dispositifs it came to life with. In the post-structural analysis of
powerm, we come back looking into apparatuses and finding the parts or a
complete miniature of the operation of the power

\Gls{genai} systems can be read as a part of this shift. Their architectures do not discipline a subject within the context of an enclosure; they are not necessarily designed to achieve a specific inscription. However, they have the capability to modulate meaning, affect, and behavior by operating upon statistical representations of language, vision, and interaction. In place of rules or norms, \gls{genai} systems govern through probabilistic inference: they do not enforce a fixed logic, but generate outputs that are dynamically aligned with the distributional patterns of their training data. And they are going through processes of fine-tuning (see Section \ref{finetuning} for a reflection)
to adjust models' behaviour to some degree. This represents arguably a post-disciplinary mechanism of control,one that governs not by exclusion or correction, but by continuous recalibration.

\marginnote{\textbf{TODO}
	\begin{todolist}
		\item Mention symbolic, non-symbolic AI
		\item Some weak lines above

	\end{todolist}
}




\section{Capitalism and Schizophrenia}

\marginnote{\textbf{TODO}: Title
	\begin{todolist}
		\item Consider opening a discussion between the postscript and \gls{dg}'s project

	\end{todolist}
}


\section{Krassman Quotes: Algorithm \& Control}
\begin{quote}
	The digital subject, at first sight, is a fictive subject, both in that it is about doubling reality in “data doubles”23 – as Deleuze observes: language becomes “numerical”, individuals morph into “dividuals” and masses into “samples, data, markets”24 – and in that the individual is no longer of primary interest in those procedures of data production. Instead, patterns of behavior and the movements of data are gathered to predict and shape future possibilities. There are criminal ambitions to be anticipated and forestalled but also consumer desires to be addressed and invoked. \[...\] Algorithms do not simply apply norms, but generate new norms of suspicion.26 They present results we did not reckon with and could not anticipate. They help us to envision the unimaginable and perhaps to preempt the incalculable. \[...\] Power is no longer merely inscribed into the environment, the architecture, the order of light, as was the case with the Panopticon. Rather, the environment itself, the algorithms, appear to be the source of power, as they are able to process data and produce information. \[...\] They thereby produce their own truth effects. Rather than predict truthful probabilities, algorithms preempt reality. Confronting us with our desires and aspirations, they always already seem to know our wishes – precisely because drawing on a seemingly incomprehensible amount of disparate data. There is no representation and no simulation of the world, as what could have been said seems to have always already been said: there is no possibility for difference to emerge,37 and in this sense, no space for the political to be challenged.

	— \cite[15-16]{Krasmann2017}
\end{quote}



\section{\Gls{genai} Modulation }

\marginnote{\textbf{TODO}: Enter Foucault -> Deleuze, Societies of Control
	\begin{todolist}
		\item [\done] Foucault
		\item Societies of Control
		\item Control in Burroughs

	\end{todolist}
}


Amoore \parencite*{amoore2024}, for example, argues that this modulation is not neutral, the generative capacity of these systems is embedded in a \textit{governing rationality} \parencite{amoore2024}, one that renders plausible what counts as intelligible, actionable, or true. By learning and operationalizing joint probability distributions across vast corpora, \gls{genai} systems instantiate regimes of verisimilitude,offering outputs that appear coherent not because they adhere to a symbolic rule set, but because they resonate statistically. In doing so, they encode a specific politics of what can be thought, said, or imagined.

Whereas disciplinary power sought to impose order through hierarchies and segmentation, control operates by managing flows. In the case of \gls{genai}, this entails the modulation of user input, system response, and contextual adaptation in a closed feedback loop. Each prompt, response, and correction contributes to the model’s ongoing refinement, a continuous, real-time inscription of preferences and expectations into the probabilistic substrate of the system.

\Gls{genai} models therefore represent a paradigmatic case of modulation-as-governance. Their architecture is not only technical, but institutional: a site where subjectivity is shaped not through fixed norms, but through dynamic adaptation. They do not dictate, but suggest; they do not enforce, but align. Yet in this very flexibility lies a form of power that is more pervasive and less accountable than disciplinary mechanisms,one that operates in the folds of everyday interaction, shaping sense before critique can even begin.

