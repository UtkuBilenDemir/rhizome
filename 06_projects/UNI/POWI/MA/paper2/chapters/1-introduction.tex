\chapter{Introduction}

\greensquare
In recent years, substantial advancements in the field of \gls{ai}, particularly through developments in \gls{nn} and \gls{dnn} architectures, have enabled the deployment of predictive models across a wide array of domains, from social media platforms and search engines to natural language processing tasks such as text classification and topic modelling. While these applications primarily focused on analysis, relevance association, personalisation, and prediction, a new paradigm has emerged in the form of \gls{genai}. Once a relatively silent front in \gls{nlp} research, \gls{genai}'s history dates back to the 1950s \parencite[4]{cao2023a}. Unlike traditional models, \gls{genai} systems are capable of producing novel outputs; such as text, images, or code by extracting and operationalising intent from human-provided instructions. This shift marks a transformation not only in the goals and capabilities of \gls{ai}, but also in its epistemic and operational logics. Particularly in its implementations based on transformer architecture, \gls{genai} now occupies a central role in the production, interpretation, and circulation of information and media, moving beyond automation and decision support to enabling generative processes that raise fundamental questions about agency, subjectivity, and truth.

\greensquare
The analysis and critique of the \gls{ai} models is nothing new, the surveillance capabilities that has been established by the contemporary data
analysis (e.g. \cite{Krasmann2017}), the effect of a completely data based
rationality introduced by the datalogical turn (see \cite{Clough2015}), a
dividualised information flows through the profiling and association by the
models running on the web (see e.g. \cite{Cheney2011}), and the decision-making
systems adopting an algorithmic
governmentality (see e.g.
\cite{rouvroy2007}), and various ethical, as well as, bias related research
(e.g. \cite{kordzadeh2022}) have been a vibrant field in the last years. The
capability \gls{genai} models especially \glspl{llm} to meaning-making
\parencite[]{gretzky2024, mishra2024, dishon2024}, however, , have provoked
renewed inquiry. While these generative processes are rooted in a long history
of statistical and computational development, contemporary architectures with
their interpretation of the vast datasets of productive human legacy introduce
an immediate representational logic, one that claimed to embody a distinct political
model or \emph{governing rationality} \parencite[2]{amoore2024}\sidenote{The governing rationality of \gls{genai} models (see e.g. \cite{amoore2024}) refers to the generative structure of algorithmic meaning-making and should not be confused with \cite{rouvroy}'s concept of Algorithmic Governmentality. Whereas Rouvroy addresses algorithmic turns in neoliberal governance, governing rationality designates the internal logic established by the model itself.}. While this
interpretative substance
enables \gls{genai} models to communicate human-like, also establishes a power
structure over governing information as a governing institution (see e.g.
\cite{mackenzie2021} or \cite{dishon2024}). Beyond their technical capacities, these systems enact a form of governance deeply entangled with power, normativity, and new forms of subjectivisation \parencite{eloff2021}. \Gls{genai} operates through a distributional logic that structures knowledge by modelling statistical regularities in a datafied world \parencite{amoore2023}. These systems \textit{traverse data foundations} to generate outputs that appear plausible within a learned distribution, but are often critiqued with the lack of deterministic causality and transparency in justification. Whether \gls{genai} models blur the line between representation and enactment; especially the outputs form \glspl{llm} , for instance, are increasingly treated as epistemically meaningful, even authorial, despite the large discussion about the lack of intentional agent which brings us to an even richer debate and literature about the agency in human autorship.



\marginnote{\textbf{TODO}: Title
	\begin{todolist}
		\item This discussion line can be followed up on


		\item Mention somewhere soemthing like:

		My work focuses especially on the architectural components that made the
		meaning-making process of the \gls{genai} Models as comprehensive as today,
		especially those of transformer architecture like "attention", "latency",
		"gradient descent".

	\end{todolist}
}



\yellowsquare
As Michel Serres (\cite*[41]{serres2019}) put, \textit{the forces shaping our bodies} increasingly
shifted from natural conditions to environments of our own making; the
constructed world exerts influence more than the given one. The rapid acceleration of technological change has drawn the human condition more deeply into a culturally defined context, shaped by the flow of information and new forms of mediation than into any purely natural domain. The models capable of meaning-making are farther away from being mere classication and filtering tools, they actively participate in structuring the mediation of culture and knowledge, further amplifying their ability to (re-)structure the reality. What \gls{genai} introduces is beyond its technical nature inquires institutional analysis \parencite{mackenzie2021} of the power structure deployed in this new constellation. Framing the question as \emph{in what form of institutional nature the architecture and rationality of the contemporary \gls{genai} algorithms deploy and what conclusions these implicate on agency, subjectivisation, ciritique, and resistance}; this study situates \gls{genai} within the broader transformation of power described by Gilles \textcite{deleuze1992a} as the shift from Michel Foucault's \emph{disciplinary societies} (\cite*{Foucault1977}) to \emph{societies of control}.

Foucault’s account of disciplinary power emphasized enclosure, surveillance, and the moulding of subjects within bounded institutions like prisons, schools, and hospitals \parencite{foucault2008}, Deleuze's postscript outlines a more fluid and continuous, flexible form of control. Foucault’s model of disciplinary power remains foundational for understanding how modern societies shape subjects. Disciplinary power is enclosure-based: it isolates bodies in space, subjects them to surveillance and normalization, and produces docile individuals. Deleuze's description of governance in the \citetitle{deleuze1992a} operates through modulation: subjects are governed not by confinement but through their data traces, captured and recomposed in real-time\sidenote{One frequently necessary clarification in this context is that the concept of disciplinary societies refers to a specific mode of operation of (bio-)power; speaking of control or post-disciplinary societies does not imply the disappearance or replacement of discipline. Rather, one might underscore that control is itself a continuation or transformation of discipline (see \cite{kelly2015a})}.
In such control societies, individuals become \emph{dividuals}, decomposed into discrete, analysable data points recombinable by algorithmic systems \parencite{mackenzie2021} vastly increasing the field of visibility on the bodies \parencite{foucault2008}. Control does not dissolve institutions into flow; rather, it reorganizes them into mechanisms that totalise by sequencing dividuals across domains. Institutions of control no longer discipline by containment, but by aggregating, modelling, and redistributing datafied subjectivities through infrastructures such as \gls{genai} platforms. In a control society, institutions persist, but their logic changes: they no longer require rigid containment to shape subjects. Instead, they operate through networks, feedback loops, and computational infrastructures, producing a totalizing effect not through direct commands but through the continuous sequencing and aggregation of dividual traces. As Mackenzie and Porter \parencite*[]{mackenzie2021} observe, these are computational institutions, structures of power that operate via algorithmic inscription and real-time modulation, rather than disciplinary ritual or spatial enclosure.


\marginnote{\textbf{TODO}: Title
	\begin{todolist}
		\item Rewrite

	\end{todolist}
}

\Gls{genai}  has the potential to exemplify this institutional modulation. \Glspl{llm}  and multimodal models operate on immense corpora of human expression, decomposing language and imagery into tokenized, vectorized microflows. Through self-supervised learning and probabilistic inference, these models produce outputs that feel intentional, expressive, and meaningful, yet at the smae time, are fundamentally statistical orchestrations. Yet, they govern the meaning; shaping the information landscapes we inhabit, mediating how knowledge is encountered, and influencing the formation of subjectivity.
This thesis therefore advances the hypothesis that generative AI models function as \emph{institutions of control}, not metaphorically, but operationally. Their architectures instantiate regimes of truth through statistical inference, acting as epistemic infrastructures that determine what can be said, imagined, or inferred. This shift raises the stakes of critique. In control societies, resistance cannot depend on unmasking ideology or demanding transparency alone. Instead, critique must become processual and counter-sequential: it must trace the operations of sequencing and propose alternative arrangements that disrupt the logic of totalisation \parencite{mackenzie2021}. Accordingly, I adopt a micropolitical perspective, asking not merely what \gls{genai} systems do, but how they do it. What are the machinic elements, attention mechanisms, tokenisation, transformer layers that enable the modulation of information and subjectivity? And where, if anywhere, might one locate lines of flight within these architectures? Can their operation be reappropriated as tools for critique, invention, or resistance? Resistance, in this sense, does not have to be necessarily mean halting development or rejecting these systems outright; rather, it involves maintaining an openness to transformation while actively engaging with and shaping these changes where intervention remains possible (see \cite[227]{tucker2021}).


Rather than positioning \gls{genai} as either emancipatory or repressive, this study approaches it as a complex institutional actor embedded in contemporary capitalism, furthermore develops a ciritical reflection of Deleuze's concept of control by deviating both in terms of the analysis of the \gls{genai} architecture, and in the critique of the institutional formation these models establish other works of \gls{dg}  (see e.g. \cite{deleuze1983}, \cite{deleuze1987}). As \gls{dg} argue, capitalism decodes and deterritorialises flows only to reterritorialise them elsewhere \parencite{deleuze1983}. In this context, \gls{genai} functions not merely as a tool of production or surveillance, but as a mechanism of epistemic reterritorialisation: producing coherence, narrativity, and alignment from fragmented inputs. It governs the production of meaning while also embedding the potential for alternative uses, ones that may exceed or redirect capitalist logics. This study thus offers a re-critique of control societies through the lens of generative architectures. Combining political theory, and technical analysis, it examines how \gls{genai} models operate on both infrastructural and epistemic levels. In doing so, it seeks to develop a renewed account of power, one grounded in probabilistic modulation, infrastructural inscription, and the micropolitics of machine reasoning. Ultimately, I argue that while the generative capabilities challenge the pillars of the control society concept, while finding particularly insightful correspondences in other literature of \gls{dg} .


After setting the scene with the theoretical framework in Chapter \ref{theoretical}, the study delves into the transformation from \textit{Disciplinary Societies} to those \textit{Societies of Control} with the analysis of Deleuze's short essay \citetitle{deleuze1992a} in Chapter \ref{control}. This analysis also includes a overlook on the themes of subjectivisation, institutional formations especially with consideration of the novel computational apparatuses, and micropolitical resistance. Chapter \ref{cha:ai} starts with the ambitious goal of charting both the historical development of \gls{ai}, and the inner machinery of the contemporary \gls{genai} models. A crucial part of the progress in the chapter remains to be to define if and how \gls{ai} models were being used in an institutional manner up until the novelties like \gls{genai} and how the current machinery enables \gls{genai} models, especially \glspl{llm} to be potential agents of subjectivization.

\begin{orangebox}
	Consider the following:


	•	What does it mean when a machine modulates language, rather than represents it?
	•	How does a transformer model participate in the social, not as a metaphor, but as a machinic node?
	•	What kinds of institutions are being formed not through buildings or laws, but through attention-weighted predictions and recursive training?
\end{orangebox}


