\chapter{TO BE DISTRIBUTED}


\sidenote{\textbf{CONJUNCTIVE}: An idea could be mackenzie -> Rouvroy -> Serres}

\section{Techno-Feudalism and the Coils of the Serpent}
Finding the intentions of the global corporates behind the operation of \gls{genai} models is a much less of a sophisticated analysis in comparison, but a point
we cannot possibly overlook in this discussion whatsoever.


\section{Michel Serres?}


The parasite (noise) is not just a disturbance; it can create new orders, as systems reorganize to manage it.
\begin{itemize}
	\item Political implication:
	\item Power and social relations are often mediated by parasitic flows: who interrupts, who feeds, who reorganizes.
	\item There is no pure communication or smooth system—noise is generative.

\end{itemize}

We might need noise/parasite for change in AI

\section{The role of critique?}

\begin{quote}
	Critique as a practice of stepping beyond the limits of possible  knowledge, for some, came to replace the idea that critique should establish the  limits of legitimate knowledge.

	— \cite[17]{mackenzie2021}
\end{quote}

We need the critique ability not just for us, we need the ciritique to even
change models.

Rouvroy puts it succinctly when she says that  algorithmically produced knowledge is no longer produced by humans about the world, rather it is ‘produced from the digital world’ (Rouvroy, 2012, p. 4). This is very much like how Serres emphasises that everything is emerging from the culture itself, and the nature as a ground is only on the periphery
