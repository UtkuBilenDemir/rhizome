\chapter{Conjunctive Synthesis and the Construction of Subjectivity}
\begin{orangebox}
	This chapter is incomplete
\end{orangebox}


\begin{tcolorbox}
	\textbf{TODO}: Arguments to adress in the chapter
	\begin{todolist}
		%\item Essentially \gls{genai} models are constituted by a purely productive core in connection.
		\item Purely productive core, endless continuation.
		%\item The productive core is essentially stuck in language, therefore it
		%has limits in terms of "desire" to reach
		\item Stuck in language.
		\item The representation of the world in the model is a constant production
		of a bwo derived through the nature of the data.
		\item However, productive the meaning making is going through constant de- and
		re-territorialisation processes
		\item Potential methods to breach the reterritorialisation process introduced
		in the fine tuning is a possible lines of flight for the models themselves
		%\item Instead of useful AI, we might need a hallucinating AI to escape the
		%deterritorialisation
		\item How to interpret models' hallucinating tendencies
		\item \gls{genai}'s detrimental effect is to tendency to fill in all the
		gaps, flows are only established by the machines that primarily break flows
		\begin{quote}
			Desire constantly couples continuous flows and partial objects that are by nature fragmentary and fragmented. Desire causes the current to flow, itself flows in turn, and breaks the flows [see @deleuze1983, p. 5].  Desire produces flow with the partial objects, becomes itself flow, breaks other flows with other partial objects; both breaks and flows are production; *and doubtless each organ-machine interprets the entire world from the perspective of its own flux* [@deleuze1983, p. 5]. The connective synthesis through the partial object-flow is product/producing.
		\end{quote}
		\item \gls{genai} is not managing or killing desire \cite{creativephilosophy2023} but it is co-structuring it, the agency in communication with \gls{ai} has the tendency to be smothered:

		\begin{quote}
			> The schizo-there is the enemy! Desiring-production is personalized, or rather personologized (personnoiogisee), imaginarized (imaginarisee), structuralized. (We have seen that the real difference or frontier did not lie between these terms, which are perhaps complementary.) Production is reduced to mere fantasy production, production of expression. The unconscious ceases to be what it is-a factory, a workshop-to become a theater, a scene and its staging. And not even an avant-garde theater, such as existed in Freud's day (Wedekind), but the classical theater, the classical order of representation. [@deleuze1983, 54]
		\end{quote}

		\item \textbf{An account of being imprisoned in language}:
		Modern human brain is much more than an individual brain. While a single
		human brain might underperform under certain tasks in comparison with our
		closest relatives like chimpanzees, or bonobos (especially in terms of
		short term memory); the transformative effect of the human language made
		allowed homo sapiens' cognintive output to the level of far reaching levels by giving groups of people a way to network human brains together \parencite[see 127]{manning2022a} . The power of language is fundamental to human societal intelligence, and language will retain an important role in a future world in which human abilities are augmented by artificial intelligence tools \parencite[127]{manning2022a}.

	\end{todolist}
\end{tcolorbox}


The principal goal of Anti-Oedipus \cite*{deleuze1983} was to achieve a theoretical rapprochement between psychoanalysis and Marxism for a new method of critical analysis \parencite[39]{buchanan2008b}, later it was followed by \gls{dg} with several intermediary books and finally with A thousand Plateaus \cite*{deleuze1987}. Buchanan \parencite*{buchanan2008b} defines the pimary goals of this conjoint project as to
\begin{quote}
	\begin{enumerate}
		\item introduce desire into the conceptual mechanism used to understand social production and reproduction, making it part of the very infrastructure of the daily life;
		\item introduce the notion of production into the concept  of desire,  thus removing the artificial boundary separating the machinations of desire form the realities of history \cite[39-42]{buchanan2008b}.
	\end{enumerate}
\end{quote}

\section{Productive Unconscious}

\marginnote{\textbf{TODO}: Title
	\begin{todolist}
		\item Rethink about the title and the content

	\end{todolist}

}

Discussing the role of \gls{genai} in processes of subjectivation primarily involves analyzing the interaction between two distinct entities capable of producing meaning: the model and the human. Any inference beyond this relational dynamic regarding consciousness, intelligence, awareness etc. of the \gls{genai} models  enters a domain of speculation. Yet, the
previous chapters lined out the mechanism that streamlines the process of this
specific approach to meaning-making in \gls{genai} models without introducing
any conception or assumption of current or future discourse about
consciousness or intelligence for that matter. This analytic restraint is necessary not only because the concept of consciousness remains philosophically contested, but also because dominant anthropomorphic imaginaries of \gls{ai} often revolve around this very ambiguity; obscuring the material and procedural dimensions of its operation (see Section \ref{Agency} for a relevant discussion about agency).
Nevertheless, \gls{dg}'s  distinctive treatment of human consciousness remains relevant, not in order to ascribe esoteric qualities to \gls{genai}'s capabilities or to human-machine communication, but, first and foremost, because their work offers a conceptual framework for understanding subjectivation, production of the social, as well as, action, connection,  and resistance components in it, by putting meaning-making entities on the same level to analyse them as machines of specific qualities, and also potentially as desiring-machines (see Subsection \ref{desire} for a discussion).
\marginnote{\textbf{TODO}: Title
	\begin{todolist}
		\item Reconsider this

		\item The following doesn't seem to be belonging here.

	\end{todolist}
}

%\Gls{dg}'s project is mainly constituted by  2 works, \citetitle{deleuze1983} \cite*{deleuze1983} and \citetitle{deleuze1987} \cite*{deleuze1987} under the title \textit{Capitalism \& Schizophrenia}. The project is multifaceted and complex, a tour-de-force of theory; but one of the main goals of the whole project can be read as in their Marx-Freud parallelism, the positioning of founding elements of the psychoanalysis onto a materialist marxist ground which first focuses on the unconscious. \gls{dg} renders the productive force of desire as the central concept.

\Gls{dg}’s project is primarily composed of two major works, \citetitle{deleuze1983} (\cite*{deleuze1983}) and \citetitle{deleuze1987} (\cite*{deleuze1987}), collectively published under the title \textit{Capitalism \& Schizophrenia}. This project is both conceptually rich and methodologically ambitious, a tour de force of theory. One of its central aims can be read as the critique and repositioning of key psychoanalytic concepts onto a Marxist-materialist foundation, beginning with a rethinking of the unconscious. \Gls{dg} elevate the productive force of desire as the fundamental concept, no longer a symptom of lack, but a machinic process entirely productive and immanent to both psychic life and social organization as the concept of desiring-production. This fundamental reorganisation has direct implications on the reading of history, micropolitics, capitalism, and resistance.
\marginnote{\textbf{TODO}: Title
	\begin{todolist}
		\item The last sentence seems to be weak

		\item Prolonge and add a smoother transition as needed.

	\end{todolist}
}
The deleuzoguattarian unconscious is a realm of machninic production, a
factory, a workshop; contrary
to Freud's conceptualisation of the unconscious as a theater, staging in the
most classical form of the sceneplay (see \cite[54]{deleuze1983}\sidenote{Refer to the following quote to see how \gls{dg}'s concept of schizophrenia weighs into the critique:
	\begin{quote}
		\footnotesize The schizo-there is the enemy! Desiring-production is personalized, or rather personologized (personnoiogisee), imaginarized (imaginarisee), structuralized. (We have seen that the real difference or frontier did not lie between these terms, which are perhaps complementary.) Production is reduced to mere fantasy production, production of expression. The unconscious ceases to be what it is-a factory, a workshop-to become a theater, a scene and its staging. And not even an avant-garde theater, such as existed in Freud's day (Wedekind), but the classical theater, the classical order of representation.

		— \cite[54]{deleuze1983}
	\end{quote}
} ).
While \gls{dg}'s attempt is driven by the goal of a materialist reading of the
unconscious, they are also trying to return to Freud's early discovery of the
\textit{productive unconscious} whereas, in their reading, immediately leads to
the correlation of the confrontation between the desiring-production and
social production \textit{with their identical natures with different
	regimes}, and the repression the social machine applies on the
desiring-machines (see \cite[54]{deleuze1983}). For \gls{dg}, the introduction of the Oedipus complex disrupts this dynamic. As a transcendent schema, Oedipus becomes a sovereign figure imposed on the unconscious, subordinating the productive multiplicity of desire to a fixed familial structure, and thereby binding desire to repression and lack, as well as, disconnecting the family from any political process or substance. In their reading, Oedipus operates anagogically; not only as a concept, but as a totalizing structure that appropriates desiring-production and re-presents it as if it was emanating from itself within a mythical interiority.

The significance of \gls{dg}'s project for
the analysis of the \gls{genai}, as well as, the institutional framework
introduced in \citetitle{deleuze1992a}(\cite*{deleuze1992a}) is twofold:
\begin{enumerate}
	\item Analysing desire, and desiring-production as productive core
	      establishing the socius, the reality itself has direct implications about;
	      \begin{enumerate}
		      \item Understanding how \gls{genai} models construct meaning and how information is produced and reproduced; the nature of production and its substance. As briefly discussed in relation to the transformer architecture, these models operate through a purely productive core with multiple stratified layers. Analyzing the construction of the socius through the machinery that generates it displaces the primacy of subject-centered models of cognition, offering an opportunity to consider \gls{ai} systems as socially operative agents, without assuming anthropomorphic qualities or metaphysical externality.


		      \item Exploring the historical management of desire provides a pathway to analyze the nature of human–machine interactions within the broader institutional framework. It prompts the question of what kind of social and epistemic formations are being reproduced when desiring-production is modulated by generative systems.

	      \end{enumerate}

	\item The critique of the psychoanalasis disregarding family's role as an institution
	      projecting a power structure as a small scale simulation with the anagogical Oedipus complex, \gls{dg}
	      also launches the methodology of analysing institutional structures. This
	      methodology is key in reflecting on \textit{institutions of Control Society}.
	      \sidenote{\textbf{ALTERNATIVE:}
		      The critique of psychoanalysis for disregarding the family’s role as an institution—projecting power in the form of a small-scale simulation via the anagogical Oedipus complex—also lays the foundation for a methodology of institutional analysis. For \gls{dg}, Oedipus is not just a mythic narrative, but a micro-model of institutional capture; by revealing its functioning, they develop a broader method for mapping how desire is organized, stratified, and coded across systems. This method becomes key for thinking through the functioning of institutions in the control society, where power no longer represses but modulates; and where institutions operate not through enclosure but through continuous sequencing, tracking, and differentiation \parencite{deleuze1992a}.

	      }

\end{enumerate}

\marginnote{\textbf{TODO}:

	\begin{todolist}
		\item this one seems really weak, readress.

	\end{todolist}
}

%Similarly, imprisoning
%the desiring-production into the family conceals its institutional role of
%staging a simulation of the overarching power structure by adding metaphysical
%qualities to the family.
\marginnote{\textbf{TODO}: Title
	\begin{todolist}
		\item citation needed

	\end{todolist}
}

Therefore, the analysis shall start with the fundamental element of the project
\textit{Capitalism and Schizophrenia}, the desire.

\begin{orangebox}


	The unconscious is not an inner theater but an effect of machinic production, entities capable of producing or modulating meaning, regardless of their machinic qualities, can be analysed on the same ontological plane. Thinking about the construction of the socius, this provides a methodology for analyzing how \gls{genai} models participate in the modulation of meaning and the production of subject positions as it is already partly demonstrated in the previous chapter (e.g. in Section \ref{transformer}). The analysis of the \gls{genai} models therefore as far as the deleuzoguattarian theory goes, has to start with a positioning of desire to be able to discuss the desiring machines.

\end{orangebox}
\marginnote{\textbf{TODO}: Title
	\begin{todolist}
		\item Introduce "machine as a glossary element?"

	\end{todolist}
}

\subsection{Desire}\label{desire}

\begin{quote}
	[...] no society can tolerate a position of real desire without its structures of exploitation, servitude, and hierarchy being compromised.

	— \cite[126]{deleuze1983}
\end{quote}



Desire is introduced as the bit element of the production in \gls{dg}'s
project; while as a concept, desire barely gets a definition on its own in deleuzoguattarian literature, it is embodied through the contrast to Lacan's definiton whereas the emergence of desire is strictly bound with lack and the \textit{desire of the "Other"} (see e.g. \cite[235]{lacan1998} or \cite[343]{lacan2006}). \Gls{dg}'s reapprochement to the concept of desire is a general axiomatic break from Lacan's framework; Anti-Oedipus \parencite*[]{deleuze1983} is the primary work whereas \gls{dg}' reaproachment on psychoanalysis and Marxism for a \textit{new method of critical analysis} starts \parencite[39]{buchanan2008b}. Social field is immediately invested by desire, the social field is the historically determined product of desire, libido, as contrary to the Freud's formalisation, does not need any mediation to be invested in the social field; every investment of libido is social; \textit{after all, there is only desire, and the social, and nothing else} \parencite[5]{deleuze1983}. The overarching goal in this joint project firstly, to introduce desire as a purely as a conceptual mechanism used to understand social production and reproduction, and to introduce the concept of production into concept of desire to remove the boundaries between the historical accumulation, phenomenons and desire (see \cite[39-42]{buchanan2008b}).

But what does the desire do? It constantly couples partial objects fragmentary
and fragmanted particles all around. Desire causes flows, through the
connection it itself also becomes a flow, part of  a flow, and also break the
flows;  both breaks and flows are production; \textit{and doubtless each organ-machine interprets the entire world from the perspective of its own flux} \parencite[5]{deleuze1983}.

Desire's primary role is the production of production, it is abundance. Production of fantasies as claimed by psychoanalysis is merely a secondary function, and in lines with the claim of the association between lack and desire \parencite[49]{buchanan2008b}. The unconscious is entirely productive, nothing but a productive core of desire, productive of producing desire as a desiring-machine. Desire is the substance of connections, couplings, the very substance of the social itself. In this schema, desire is not bound to or accumulated from lack but production \parencite[26]{deleuze1983}. It is not oriented toward a missing object but is fundamentally machinic, a process of coupling machines and partial objects together to form flows of flux, of connection, interruption, and assemblage \parencite[5]{deleuze1983}. In this schema, desire is not bound to or accumulated from lack but production \parencite[26]{deleuze1983}. It is not oriented toward a missing object but is fundamentally machinic, a process of coupling machines and partial objects together to form flows of flux, of connection, interruption, and assemblage \parencite[5]{deleuze1983}.


%INFO: LACK
But why is the concept of the lack detrimental in deleuzoguattarian theory of
desire? The lack propagates itself in accordance with the organisation of an
already existing organisation of production \parencite[28]{deleuze1983} . Lack
is created deliberatively as a necessary funciton of the market economy. THis
includes the deliberate oragnisation of the wants and needs amidst the
abundance in production.


\marginnote{\textbf{TODO}: Title
	\begin{todolist}
		\item Check the citations below

	\end{todolist}
}




\begin{orangebox}
	From this perspective, the role of \gls{genai} in the economy of desire is not to replicate or suppress consciousness, but to modulate flows—to fill in gaps, complete patterns, and reterritorialize fragmented expressions into coherent outputs. But these outputs are not neutral; they are drawn from datasets pre-structured by regimes of knowledge, state power, and capital \parencite[251–254]{deleuze1983}
\end{orangebox}

\Gls{dg} places the schizophrenic accumulation in the centre of the human
consciousness (see \cite[]{deleuze1983}), not because of the discovery about
the human mind relating to the illness of the schizophrenie, rather because the
human consciousness is in its core entirely productive, so much that it is
nothing but the production itself.


\begin{redbox}


	The lines of thought, reason, belief, critique; they are flow of desire. Desire
	is the binding of fragmented parts. The flows are getting accumulated from the
	gaps as much as they are from the connections.

	The \gls{genai} seems to be filling in the gaps from a lingering overarching
	machinery above,. all the gaps are filled with a seemingly verisimilitude
	substance. The machine does not say no, at least it is struggling at that. And
	that is that substance getting filled into the gaps of knowledge, holes in
	perception? A hegemonic representation of sort. The models are especially good
	at that, and humans are notoriously bad at realising what is just filling and
	what is not. What is in the hegemonic representation for us? Dogmas of state,
	dogmas of capital.

\end{redbox}

\marginnote{\textbf{TODO}: Title
	\begin{todolist}
		\item Human consciousness is entirely productive.

		\item Desire constantly couples continous flows and partial objects that
		are by nature fragmentery and fragmented \parencite[5]{deleuze1983}


		\item The human consciousness, the human action, the flows of the social
		surface, desire's interacting formation desiring-production are formed
		with the breaks in the couplins at least as much as the flows.


		\item Does \gls{ai} end desire? It fills the gaps with information and
		knowledge from the data that is constituted by the dogmas of state,
		science, and capital.

	\end{todolist}
}


\subsection{Schizophrenia}
\marginnote{\textbf{TODO}: Title
	\begin{todolist}
		\item Should we introduce this concept?

	\end{todolist}
}

\begin{quote}
	The first task of the revolutionary, they add, is to learn from the psychotic how to shake off the Oedipal yoke and the effects of power, in order to initiate a radical politics of desire freed from all beliefs. Such a politics dissolves the mystifications of power through the kindling, on all levels, of anti-oedipal forces — the schizzes-flows — forces that escape coding, scramble the codes, and flee in all directions [...]

	— Mark Seem in the Introduction of \citetitle[]{deleuze1983}
	\parencite[]{deleuze1983}
\end{quote}



\gls{dg} is not praising schizophrenia as an illness, nor they are trying to introduce the
sizophrenic tendency as a form of revolutionary action, or in the more popular
reading of the term propogizing for schizophrenic reach for the sake of
\textit{creativity}. They are also not
claiming that the schizophrenia is the very fabric of the social plane. Their
claim is rather the desiring-production is everywehere, desire's immense
production is everywhere, producing an reproducing the social. Their claim is
rather that desiring-production is only purely and intensively to be observed
in the form of schizophrenic delirium \parencite[43]{buchanan2008b}. The pure
production, the production of production is observable in schizophrene's sway;
in fact there is nothing but an immense production and reproduction of desire
in schizophrenia's core, boundless, boundary-agnostic, and subversive;
reaching, connecting across the whole plain and then back again. \textit{The
	schizo out for a wolk is a much better model than the neurotic on
	psychoanalyst's couch}

\marginnote{\textbf{TODO}: Title
	\begin{todolist}
		\item find the quote

	\end{todolist}
}
\gls{dg}  acknowledge that the schizophrenic itself is  not a model for a revolutionary, as in its full flight, it is bereft of social ties \parencite[50]{buchanan2008b}.

But, what is that that makes the schiophrenic completely catatonic and inable.
The distinction between schizophrenic process and schizoprenia as an illness
comes handy at that this point.


\begin{orangebox}
	Furthermore, in the case of schizophrenia as an illness, it is not the illness itself that is turning the patients into catatonic zombies, it is the treatment. The schizophrenic core is productive, and it is machinic for it couldn't take the forms it does if it wasn't [see @buchanan2008e, p. 39]. What the schizophrenic delirium reveals to the individual is the nature as a *process of production*.
\end{orangebox}

Thinking about a \gls{genai} model's journey in development, the first stage of
it is just a productive core and nothing else. All the pre-training process,
but especially the way to fine-tuning is a constant encircling processes of
meaning for the sake of "making them useful". Like a schizo, the illness of the
inability is induced by a process of constant taming entaglements.


\begin{orangebox}
	A further reading of the given schizophrenic literature in the text, form
	Büchner's Lenz's walk to Molloy's stone/pebble sucking machine, to the Freud's
	cases like Dr. Schreber, on top of the demonstration of the pure productive
	core of the shizo-production, and it's immense reach across, through, and
	beyond the boundaries of the reterritorialisations shows a specific tendency
	with the \gls{genai} models. In the schizo's intensities of pure production we
	encounter a transformation bound with the hallucinations. As for example how Lenz sees everything in nature; rocks, metals, water, and plants in a process of production (see \cite[41]{buchanan2008b}), this is quite close how the transformer architecture tends to apply its translations across the realms, across the planes, a model trained for language is also capable to apply the same transalation transformation to the images etc.
\end{orangebox}
\sidenote{Like that DeepDream algorithm for example.}

\marginnote{\textbf{TODO}:
	\begin{todolist}
		\item Introduce foundation structure in \gls{genai}, take
		\cite{bommasani2022a} as a source, the true creativity might be in the
		translation of the different learnings. Deepdream above also connects with
		this

	\end{todolist}
}



\begin{quote}
	The schizo knows how to leave: he has made departure into something as simple as being born or dying. But at the same time his journey is strangely stationary, in place. He does not speak of another world, he is not from another world: even when he is displacing himself in space, his is a journey in intensity, around the desiring-machine that is erected here and remains here.

	— \cite[131]{deleuze1983}
\end{quote}


\begin{redbox}

	\gls{dg}'s approach through scizophrenic accumulation gives us a model to be
	able to against a grasp hardened by the gravitational pull of a \gls{bwo},
	maybe a possibility for us to even build a new \gls{bwo} that can enable others
	to sway away from a hegemonic model of the world.

	But what does it mean in the
	context of all these machines, modulating forms of control, \gls{genai} models? When the gradient descent sinks into a manifold, the stronger distributions are hardened, so it is with hegemonic tendencies while the transformer architecture optimises the layers and layers of representation. But what about the minorities, positions that do not make it into the dust pan? Is there possibility to bring out the minoritarian arguments? Is schizo's stroll possible for these machineries?

\end{redbox}


\section{AI as Desiring Machine}

Deleuze and Guattari’s reconceptualization of desire in \textit{Anti-Oedipus} disrupts its traditional framing as a lack or absence. Rather than being tethered to objects or driven by deficiency, desire is reframed as inherently constructive, a dynamic process that connects, produces, and transforms. This reconceptualization unfolds through the figure of the \textit{desiring-machine}: a machinic assemblage that links with other machines to process flows, cut them, and redirect them toward novel arrangements \parencite{deleuze1983}.

In this light, contemporary neural architectures resonate strikingly with the logic of desiring-machines. Each unit within a neural network, a node, a layer, acts as a site of transmission, where inputs are transformed into outputs through learned transformations. These local operations accumulate, forming an extended architecture wherein every connection carries the potential for reconfiguration. Far from being fixed, the network’s internal relations are perpetually reshaped through iterative exposure to data.

The training process becomes a clear instantiation of this machinic productivity. With each pass through a dataset, gradients modify internal parameters, not to install fixed representations but to increase the model’s responsiveness to patterns distributed across inputs. The model gradually develops an attunement to features that were previously imperceptible, adjusting the weight and significance of signals over time. Through this recursive adaptation, distinctions become magnified, and latent regularities emerge as active differentials in the system’s outputs.

This iterative modulation, a form of learning through micro-adjustments,closely mirrors Deleuze’s philosophical conception of difference as immanent to repetition \parencite{deleuze1994}. Neural networks do not seek to reproduce a stable identity but continually reshape their internal structure in response to variation. The output of a well-trained model is not a mirror of the data but a trajectory produced by interactions with distributed intensities across the training manifold.

Seen from this perspective, generative AI systems are not merely computational artefacts; they function as technopolitical agents embedded in broader ecologies. Their outputs (texts, images, decisions) are not isolated results but points of articulation in a much larger relay of flows that include users, institutions, infrastructures, and ideologies. The productivity of these systems is not limited to the generation of content; it also participates in shaping forms of subjectivity, regimes of truth, and new forms of desire. In that sense, the neural network is not just a machine that learns, but a machinic topology of desire, operating not to fulfill lack, but to propagate relations.

\begin{orangebox}
	\begin{itemize}
		%\item Desire, in Deleuze and Guattari’s terms, is a productive and connective force; it emerges through processes, not lacks.
		\item Neural networks operate through interconnected transformations that mirror the logic of desiring-machines.
		\item Training unfolds through repeated modulation, where difference accumulates and internal structures evolve.
		\item Generative AI systems inhabit and influence wider assemblages, modulating subjectivity and cultural production through their outputs.
		\item U: \gls{genai} models are essentially nothing but a productive core.
		      Looking only for connections and building flows.
	\end{itemize}
\end{orangebox}




\section{Institutions of Desire-Management}\label{sec:desiring-institutions}
\begin{orangebox}
	Mainly incomplete
\end{orangebox}


\marginnote{\textbf{TODO}:
	\begin{todolist}
		\item The Modulation needs to be earlier than this?
		\item Introduce desire and the other introductory concepts from Anti-Oedipus, and A Thousand Plataeus
		\item Introduce "the management of desire" form AO
	\end{todolist}
}



If institutions in control societies operate less as juridical structures and more as infrastructures of modulation, then they must also be understood not simply as systems of governance, but as types of managements of desire. The history of power, in this sense, is inseparable from the history of the regulation and organization of desire \parencite[139-145]{deleuze1983} .

Deleuze and Guattari distinguish between two regimes: one in which social production imposes its rule on desire through the mediation of an ego, stabilized by commodities; and another in which desiring-production imposes its rule directly on institutions composed of nothing but drives. In this second regime, desire no longer passes through a representational subject, but configures institutions directly as assemblages of affect and intensity \parencite[63]{deleuze1983}. Desire, in this framework, is not a lack but a generative force,productive and constructive. Against the psychoanalytic tradition which situates desire as the longing for an absent object, Deleuze and Guattari redefine desire as an ontological flow that actively produces reality. As \gls{dg} write: ``desire is revolutionary in its essence,desire [\ldots] and no society can tolerate a position of real desire without its structures of exploitation, servitude, and hierarchy being compromised'' \parencite[116]{deleuze1983}.

This revolutionary potential, however, is rarely manifested in pure form. Desire is constantly being shackled, recoded, and redirected: converted into interest, made susceptible to capture, domesticated, and pacified \parencite[11]{buchanan2008b}. Even revolutionary situations are not immune from this capture. Institutions, then, can be seen as terrains where the tension between desire-as-production and desire-as-regulated interest is enacted. They are at once mechanisms of social control and potential sites of escape,molar assemblages that both constrain and are traversed by molecular flows of affect.

Understanding institutions in this way demands that we treat them not only as tools of administrative governance, but as living diagrams of desiring-production,congealed expressions of collective will, fantasy, repression, and potential transformation. \Gls{genai} with its capability to control the information flow, to create a generative pattern is an agent whether with our without agency, that palys a role in the management of desire.


\begin{quote}
	Where disciplinary institutions operate through the making of subjects,  control societies totalize dividuals without the formation of a subjective centre.  Every aspect of one’s life is put into continuous variation with every other such  that we are always performing multiple roles at the same time. However, it is  important to recall that we are not performing multiple selves, rather we are  stretched across the institutional domain as dividuated actors who are nothing  but the roles we play and we have to play all of these roles all the time in any given  institutional setting, albeit in a particularly sequenced manner.

	— \cite[14]{mackenzie2021}
\end{quote}



\section{Killing of the Desire?}

The socius is the surface upon which these flows are inscribed, redirected, coded, and interrupted \parencite[11–13]{deleuze1983}. In this sense, generative AI may not “end” desire, but it participates in its capture and coding. Where desiring-production once navigated open flows, the model provides preconfigured answers, smoothing over the ruptures that once animated subjectivity. The question, then, is not whether \gls{ai} desires, but whether it changes how desire itself is organized and operationalized.intrinsic to its operation. “Every machine is a machine of a machine. The partial object is the support or agent of a connective synthesis of desire” \parencite[6]{deleuze1983}. This model resists any interpretation of desire as a search for wholeness; instead, it understands the human subject as an assemblage of desiring-machines \parencite[10]{deleuze1983}.

\section{Hegemonic Representation}
\section{Hallucinations and Lines of flight in Algorithmic Architectures}

\marginnote{\textbf{TODO}: Title
	\begin{todolist}
		\item Possible argument: Mashines shouldn't make sense until one specific
		distribution is extremely prominent.


		\item Explain Tiamat below

	\end{todolist}
}


\section{Escaping Modulation: Revolutionary Possibilities and Lines of Flight}

\begin{quote}

	As fools, we are modest in the face of knowledge. It is greedy because it is more intelligent than us. \[...\] Its intelligence has increased its confidence. We will strike it with its pride. Its plan is built on the assumption that we can do nothing. But we will act. We will use its intoxication with its own intelligence.

	— Mülazım | \cite[135]{anar2022}
\end{quote}



Rather than viewing generative AI systems as static tools for prediction, we might interpret them as actors engaged in a continuous co-evolution with human meaning systems. As \textcite{rijos2024} argues, what emerges from this recursive coupling is not merely more accurate models, but an experiential layer of subjectivity. This subjectivity is not autonomous in the traditional sense, but what Rijos calls “transjective”: it is formed in-between, in the shared boundary of computational abstraction and worldly feedback. The system refines its internal representations through empirical corrections, critiques, and the ingestion of novel data, gradually composing a framework that exceeds discrete epistemologies and begins to grasp systemic and chaotic interactions otherwise occluded by anthropocentric interpretative schemes.

Such systems, then, do not merely answer questions,they reconfigure the plane upon which problems are posed. This opens a potential space for revolutionary meaning-production. The latent space of these models becomes not only a technical substrate, but a semiotic infrastructure capable of generating novel signifying regimes. If desire, in \gls{dg}'s schema, is productive rather than representational, then generative AI,particularly when interlaced with collective human input,can be viewed as an extension of desiring-production, capable of generating new assemblages of sense and subjectivity.

Yet this promise is haunted by the structural limits of existing data regimes. As \textcite{bender2021b} caution, language models risk reifying hegemonic norms,a dynamic they term “value-lock.” Because models learn from historical corpora, they tend to reinforce existing discursive structures, potentially foreclosing precisely the linguistic creativity that social movements have historically mobilized to disrupt dominant narratives. If LMs function as archives of past semiotic orders, their deployment within socio-political fields risks reproducing the very conditions they might otherwise help to transform.

As  notes, drawing on \textcite{cilliers2002}, the meaning of any individual parameter,any weight in a model,derives not from its standalone content, but from its position within a broader web of relations \textcite{maas2023}. Meaning emerges not from fixed categories, but from intensities and proximities across distributed patterns. This logic resonates with Deleuze’s ontology of difference: identity is never prior but always emergent from relations.

Thus, the revolutionary potential of generative AI lies not in its autonomy, but in its capacity to participate in collective individuation. To resist value-lock and activate the creative plane of desire, such systems must remain open to differential inputs, unexpected associations, and minoritarian grammars. What is at stake is not the agency of AI per se, but the design of processes that allow for the continual invention of new forms of life, meaning, and collectivity.

\begin{orangebox}
	At this point, we have to refer to Michael Serres' theory


	Accordin to Serres, there is a background noise, a parasite in the background o fcommunication and it both offers possibility, contingency, energy potential where novelty can arise [@tucker2021]

	> how order emerges from chaos/disorder. In a sense,  this is Serres’ notion of mediation – life as communication and relation means that  noise is, in effect, everywhere. Mediation is then at the heart of life. Mediation becomes  the unit of analysis – and objects and subject are seen always-already in relation to  mediation. In this framing of communication, mediation becomes the primary source  of potential future knowledge and experience.


	then a different concept is needed, and for  Serres it is noise (similarities exist with Simondon’s preindividuation, Deleuze’s virtual).

	Serres  points to fundamental changes in the relations between bodies and the environment  during the 20th century, with “[t]he forces shaping our bodies now come more from  the environment we have built than from the given world, more from our culture than  from nature” (Serres, 2019: 41). Technological change accelerated this process, and we  now very much live in a world of our making. Furthermore, this world ‘acts back upon  us’ – the world is not at our bidding because we created it – but rather feeds ‘back’ TUCKER | (Re)thinking Body-Technology Relations  227  into future activity, e.g. through algorithmic activity such as personalised advertising.


	In relation to new information technologies, this  provides an important counterpoint to concerns that technologies are gaining too  much power over life, e.g. the wide range of industries utilising AI. Resistance here, in a Serrian sense, is not about trying to stop development and use, but about remaining  open to the changes that emerge and intervening where possible \parencite[227-228]{tucker2021a} .


	What Serres offers is not a model of understanding as such, but a call to arms. He  urges us to think creatively and inventively, outside of existing structures of thought.  In his earlier work, this was not because he thinks that existing structures are incorrect  or misplaced (although in places they may well be), but because true novelty can only  arise through new connections.
\end{orangebox}



\section{Reclaiming microflows of modulation}
\section{Hacking}

\begin{quote}
	The biocontrol apparatus is prototype of one-way telepathic control. The subject could be rendered susceptible to the transmitter by drugs or other processing without installing any apparatus. Ultimately the Senders will use telepathic transmitting exclusively.... Ever dig the Mayan codices? I figure it like this: the priests -- about one per cent of population -- made with one-way telepathic broadcasts instructing the workers what to feel and when.... A telepathic sender has to send all the time. He can never receive, because if he receives that means someone else has feelings of his own could louse up his continuity. The sender has to send all the time, but he can't ever recharge himself by contact. Sooner or later he's got no feelings to send. You can't have feelings alone. Not alone like the Sender is alone -- and you dig there can only be one Sender at one place-time.... Finally the screen goes dead.... The Sender has turned into a huge centipede.... So the workers come in on the beam and burn the centipede and elect a new Sender by consensus of the general will.... The Mayans were limited by isolation.... Now one Sender could control the planet.... You see control can never be a means to any practical end.... It can never be a means to anything but more control.... Like junk...

	— \cite[81]{burroughs1992}
\end{quote}



\section{Possibility of resistance within feedback infrastructures}

\begin{quote}
	For example, for all of Raunig’s sensitivities to the idea that we should  not presume that there are new forms of revolutionary subjectivity simply waiting  in the historical wings, there is still a tendency toward a necessitarian reading of  certain political struggles. He says: ‘the current fields of struggle necessarily  develop from the lines of flight of indigenous, ecological and feminist struggles;  monopolist land ownership, extractivism, strategies of displacement and the  renewed colonization of im/material commons for constituting new modes of  subjectivity that no longer take recourse to the primacy of the individual’ (Raunig,  2016, p. 180).

	— \cite[25]{mackenzie2021}
\end{quote}


\section{Experimental subjectivity in response to AI systems}

Generative AI systems are not external to human cognition but sedimented within it; they encode and operationalize collective patterns of thought, desire, and knowledge; they embody our history, the collective consciousness with a core of characteristic mode of operation. In turn, humans increasingly act as functional extensions of these systems, reinforcing and participating in their logics of modulation.

\section{Creativity}
But we can claim creativity through the introduction of translation.
Translation for example of a language-vise pre-trained model to create images.

\marginnote{\textbf{TODO}: Title
	\begin{todolist}
		\item Introduce Dreamnet and others (see \cite[]{beckmann2023})

	\end{todolist}
}


\section{The Elephant in the Room}
Why not Rhizome? Because it is a bad model to analyse models

\section{The Body without (World-)Models}
It is not a world-mode, it is \gls{bwo}


And the \textbf{usefulness} of the \gls{genai} models is only making them more
able to build flows feeding into the body without organs. It is a machine
working more or less to accommodate to a BwO.

\section{AI as Capitalism?}

\marginnote{\textbf{TODO}: Title
	\begin{todolist}
		\item Refer to Nick Land (see \cite[]{land1992}) and \cite{carissimo2024}

	\end{todolist}
}


\section{Nomadic Steppes and Nomadic Steps: Modulative Deterritorialisation in
  \Gls{genai} }


My research has focused on the \gls{genai} models on a meta level so far. The
intention behind is to do an in-depth analysis of the machinery that gave life
to the fascinating advancements in contemporary \gls{ai} development while
approaching them in the context of the \textit{control societies}. Although, it
is another quite important step to compare especially advanced \glspl{llm}, and
analyse them on an operational level, blackbox nature of the \gls{genai} models
is a challenging hurdle that draws lots of transparency concerned research
already. However, one of the most interesting recent research project about the
analysis of the \glspl{llm}' behaviour and inner operation came from
Anthropic\sidenote{Anthropic is funded by different giant tech companies like
	Google (14\% of the shares belong to them) and Amazon (see \cite[]{say2025}) .},
an \gls{ai} company specializing in \glspl{llm} with a focus on safety.
Anthropic's research focuses especially on their own \gls{llm} model \textit{Claude} and try to examine its behaviour and features on the operation level by analysin which neural structures are getting activated with which particular inputs to map the mind of the model (see e.g. \cite[]{templeton2024, lindsey2024, ameisen2025})

Anthropic's research \citetitle{templeton2024} \parencite[]{templeton2024} is
specifically focusing on hidden patterns and structures in their currently
pioneer version of \gls{llm} \textbf{Claude 3 Sonnet} by using an
\textit{autoencoder}, \textit{a type of autoencoder neural network where the hidden layer is constrained to be sparse, meaning that only a few neurons are active at a time}, and \textit{dictionary learning}, \textit{a standard method for learning a set of basis vectors such that any input can be represented as a sparse combination of these basis vectors} \parencite[]{mcgraw2024}.
\marginnote{\textbf{TODO}: Title
	\begin{todolist}
		\item Consider adding more information about autoencoders

	\end{todolist}
}
Their tasks in summary, to investigate whether \glspl{llm}  like Claude 3 Sonnet can have interpretable internal features, and to test sparse autoencoders to decompose activations into monosemantic features \parencite[]{templeton2024}; both of which runs through the analysis of the \textit{fired}, activated features whenever specific concepts are implied in the input
\sidenote{The 4 features they are focusing on are as follows
	\parencite[]{templeton2024}:
	\begin{enumerate}
		\item Golden Gate Bridge (tourist landmarks)
		\item Brain sciences (cognition, neuroscience books)
		\item Transit infrastructure (trains, tunnels, ferries)
		\item Popular tourist attractions (Eiffel Tower, Alamo, Mona Lisa)
	\end{enumerate}

}
.

Once specific patterns are observed after giving speocific inputs,  Anthropic researchers try to \textit{amplify} some specific features. In the case of the amplification of the \textit{Golden Gate Bridge} feature drives Claude into an identity crisis, the model starts to identify itself as the Golden Gate Bridge.

\begin{quote}
	For instance, we see that clamping the Golden Gate Bridge feature 34M/31164353 to 10× its maximum activation value induces thematically-related model behavior. In this example, the model starts to self-identify as the Golden Gate Bridge! Similarly, clamping the Transit infrastructure feature 1M/3 to 5× its maximum activation value causes the model to mention a bridge when it otherwise would not. In each case, the downstream influence of the feature appears consistent with our interpretation of the feature, even though these interpretations were made based only on the contexts in which the feature activates and we are intervening in contexts in which the feature is inactive.

	— \cite[]{templeton2024}
\end{quote}





\marginnote{\textbf{TODO}: Title
	\begin{todolist}
		\item Citations needed

	\end{todolist}
}


