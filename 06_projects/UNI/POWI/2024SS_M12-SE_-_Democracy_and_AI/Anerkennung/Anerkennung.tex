\documentclass[a4paper,11pt]{article}
\usepackage[utf8]{inputenc}
\usepackage{geometry}
\geometry{a4paper, left=25mm, right=25mm, top=30mm, bottom=30mm}
\usepackage{hyperref}

\title{Begründung für das Wahlfach \\ Democracy and Artificial Intelligence}
\author{Utku B. Demir}
\date{\today}

\begin{document}

\maketitle

\section{Begründung für die Wahl des Kurses}

Hiermit beantrage ich die Anrechnung des Kurses \textit{Democracy and Artificial Intelligence} als Wahlfach im Rahmen meines Masterstudiums. Der Kursinhalt überschneidet sich thematisch mit meiner Masterarbeit und ergänzt meine Forschungsfrage, indem er den Einfluss von generativen KI-Algorithmen auf demokratische Prozesse analysiert.

\section{Verbindung zu meiner Masterarbeit}

Meine Masterarbeit trägt den Titel: \textit{The Soft Machine of Control: Generative AI, Dividuals, and the Modulative Power}. Sie untersucht die Verbindung zwischen der Theorie von Gilles Deleuze zu \textit{Kontrollgesellschaften} und den aktuellen Entwicklungen im Bereich generativer KI-Modelle (genAI). Insbesondere befasse ich mich mit der Frage, wie generative KI-Algorithmen nicht nur Subjektivitäten produzieren, sondern auch politische und soziale Strukturen beeinflussen.

Der Kurs \textit{Democracy and Artificial Intelligence} stellt eine thematische Erweiterung meiner Arbeit dar, da er sich mit der Frage beschäftigt, ob und wie KI demokratische Prinzipien untergräbt oder stärken kann. Dies schließt an meine Analyse an, in der ich untersuche, ob genAI als ein Instrument von \textit{governmentality} im Sinne von Foucault fungiert oder ob sich darin auch Möglichkeiten der \textit{Deterritorialisierung} und neuer subversiver Subjektivitäten nach Deleuze und Guattari finden lassen.

\section{Relevanz der Kursinhalte}

Laut Syllabus des Kurses werden zentrale Themen behandelt, die für meine Forschung von direkter Bedeutung sind:

\begin{itemize}
	\item \textbf{Wie KI demokratische Prinzipien untergräbt:} Diese Diskussion ergänzt meine theoretische Auseinandersetzung mit der Modulation von Subjektivität durch generative Modelle.
	\item \textbf{Totalitarismus und der Verlust von Wissen und Vertrauen:} Diese Thematik ist für meine Forschung zentral, da sie sich mit der Rolle von KI als Mittel politischer Kontrolle beschäftigt.
	\item \textbf{Demokratische Stärkung durch KI:} In meiner Masterarbeit untersuche ich, ob generative KI-Modelle durch ihre probabilistische Struktur einen Raum für neue Formen von Subjektivierung und Widerstand schaffen.
\end{itemize}

\section{Fazit}

Der Kurs bietet mir eine vertiefte theoretische Auseinandersetzung mit den politischen Implikationen von Künstlicher Intelligenz, die für meine Masterarbeit essenziell sind. Ich werde die in diesem Kurs behandelten Debatten aktiv in meine Arbeit einfließen lassen und methodisch zur Analyse von Kontrollmechanismen und Subjektivitätsproduktion nutzen. Ich bitte daher um die Anerkennung dieses Kurses als Wahlfach in meinem Masterstudium.

\vspace{1cm}
\end{document}
