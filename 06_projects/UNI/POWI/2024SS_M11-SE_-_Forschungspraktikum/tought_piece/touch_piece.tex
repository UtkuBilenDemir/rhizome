%%%%%%%%%%%%%%%%%%%%%%%%%%%% Define Article %%%%%%%%%%%%%%%%%%%%%%%%%%%%%%%%%%
\documentclass[nobib, openany, justified, a4paper, 14pt]{tufte-book}
%%%%%%%%%%%%%%%%%%%%%%%%%%%%%%%%%%%%%%%%%%%%%%%%%%%%%%%%%%%%%%%%%%%%%%%%%%%%%%%

%%%%%%%%%%%%%%%%%%%%%%%%%%%%% Citations %%%%%%%%%%%%%%%%%%%%%%%%%%%%%%%%%%%%%%%
%\usepackage[utf8]{inputenc}
\usepackage[style=authoryear-icomp]{biblatex}
%\usepackage[style=apa]{biblatex}
\addbibresource{/Users/ubd/Bibliotheca/bib.bib}
%%%%%%%%%%%%%%%%%%%%%%%%%%%%%%%%%%%%%%%%%%%%%%%%%%%%%%%%%%%%%%%%%%%%%%%%%%%%%%%

%%%%%%%%%%%%%%%%%%%%%%%%%%%%% Using Packages %%%%%%%%%%%%%%%%%%%%%%%%%%%%%%%%%%
\usepackage{newunicodechar}
\newunicodechar{🦜}{[parrot]}
\PassOptionsToPackage{prologue,dvipsnames}{xcolor}
\sloppy  % globally
\usepackage{geometry}
\usepackage{graphicx}
\usepackage{amssymb}
\usepackage{amsmath}
\usepackage{amsthm}
\usepackage{empheq}
\usepackage{mdframed}
\usepackage{booktabs}
\usepackage{lipsum}
\usepackage{graphicx}
\usepackage{color}
\usepackage{psfrag}
\usepackage{pgfplots}
\usepackage{bm}
\usepackage{epigraph}
\usepackage{titlesec}
\usepackage{tcolorbox}
\usepackage{csquotes}
\usepackage{pifont}
\usepackage{enumitem,amssymb}
% \usepackage{spoton} % adds \todo functionality I hope
%%%%%%%%%%%%%%%%%%%%%%%%%%%%%%%%%%%%%%%%%%%%%%%%%%%%%%%%%%%%%%%%%%%%%%%%%%%%%%%

% Other Settings

%%%%%%%%%%%%%%%%%%%%%%%%%% Page Setting %%%%%%%%%%%%%%%%%%%%%%%%%%%%%%%%%%%%%%%

%%%%%%%%%%%%%%%%%%%%%%%%%% Define some useful colors %%%%%%%%%%%%%%%%%%%%%%%%%%
\definecolor{maroon}{RGB}{128,0,0} %hlred
\definecolor{MAROON}{RGB}{128,0,0} %hlred
\definecolor{deepBlue}{RGB}{61,124,222} %url-links
\definecolor{deepGreen}{RGB}{26,111,0} %citations
\definecolor{ocre}{RGB}{243,102,25}
\definecolor{mygray}{RGB}{243,243,244}
\definecolor{shallowGreen}{RGB}{235,255,255}
\definecolor{shallowBlue}{RGB}{235,249,255}
\definecolor{mediumpersianBlue}{rgb}{0.0, 0.4, 0.65}
\definecolor{persianBlue}{rgb}{0.11, 0.22, 0.73}
\definecolor{persianGreen}{rgb}{0.0, 0.65, 0.58}
\definecolor{persianRed}{rgb}{0.8, 0.2, 0.2}
\definecolor{debianRed}{rgb}{0.84, 0.04, 0.33}
%%%%%%%%%%%%%%%%%%%%%%%%%%%%%%%%%%%%%%%%%%%%%%%%%%%%%%%%%%%%%%%%%%%%%%%%%%%%%%%

%%%%%%%%%%%%%%%%%%%%%%%%%% Indentation Settings %%%%%%%%%%%%%%%%%%%%%%%%%%%%%%%
\makeatletter
% Paragraph indentation and separation for normal text
\renewcommand{\@tufte@reset@par}{%
	\setlength{\RaggedRightParindent}{0pc}%1.0
	\setlength{\JustifyingParindent}{0pc}%1.0
	\setlength{\parindent}{1pc}%1pc
	\setlength{\parskip}{5pt}%0pt
}
\@tufte@reset@par

% Paragraph indentation and separation for marginal text
\renewcommand{\@tufte@margin@par}{%
	\setlength{\RaggedRightParindent}{0pc}%0.5pc
	\setlength{\JustifyingParindent}{0pc}%0.5pc
	\setlength{\parindent}{0.5pc}%
	\setlength{\parskip}{5pt}%0pt
}
\makeatother



%%%%%%%%%%%%%%%%%%%%%%%%%% Define an orangebox command %%%%%%%%%%%%%%%%%%%%%%%%
%o\usepackage[most]{tcolorbox}

\newtcolorbox{orangebox}{
	colframe=ocre,
	colback=mygray,
	boxrule=0.8pt,
	arc=0pt,
	left=2pt,
	right=2pt,
	width=\linewidth,
	boxsep=4pt
}


\newtcolorbox{redbox}{
	colframe=red,
	boxrule=0.8pt,
	arc=0pt,
	left=2pt,
	right=2pt,
	width=\linewidth,
	boxsep=4pt
}
%%%%%%%%%%%%%%%%%%%%%%%%%%%%%%%%%%%%%%%%%%%%%%%%%%%%%%%%%%%%%%%%%%%%%%%%%%%%%%%

%%%%%%%%%%%%%%%%%%%%%%%%%%%% English Environments %%%%%%%%%%%%%%%%%%%%%%%%%%%%%
\newtheoremstyle{mytheoremstyle}{3pt}{3pt}{\normalfont}{0cm}{\rmfamily\bfseries}{}{1em}{{\color{black}\thmname{#1}~\thmnumber{#2}}\thmnote{\,--\,#3}}
\newtheoremstyle{myproblemstyle}{3pt}{3pt}{\normalfont}{0cm}{\rmfamily\bfseries}{}{1em}{{\color{black}\thmname{#1}~\thmnumber{#2}}\thmnote{\,--\,#3}}
\theoremstyle{mytheoremstyle}
\newmdtheoremenv[linewidth=1pt,backgroundcolor=shallowGreen,linecolor=deepGreen,leftmargin=0pt,innerleftmargin=20pt,innerrightmargin=20pt,]{theorem}{Theorem}[section]
\theoremstyle{mytheoremstyle}
\newmdtheoremenv[linewidth=1pt,backgroundcolor=shallowBlue,linecolor=deepBlue,leftmargin=0pt,innerleftmargin=20pt,innerrightmargin=20pt,]{definition}{Definition}[section]
\theoremstyle{myproblemstyle}
\newmdtheoremenv[linecolor=black,leftmargin=0pt,innerleftmargin=10pt,innerrightmargin=10pt,]{problem}{Problem}[section]
%%%%%%%%%%%%%%%%%%%%%%%%%%%%%%%%%%%%%%%%%%%%%%%%%%%%%%%%%%%%%%%%%%%%%%%%%%%%%%%

%%%%%%%%%%%%%%%%%%%%%%%%%%%%%%% Plotting Settings %%%%%%%%%%%%%%%%%%%%%%%%%%%%%
\usepgfplotslibrary{colorbrewer}
\pgfplotsset{width=8cm,compat=1.9}
%%%%%%%%%%%%%%%%%%%%%%%%%%%%%%%%%%%%%%%%%%%%%%%%%%%%%%%%%%%%%%%%%%%%%%%%%%%%%%%

%%%%%%%%%%%%%%%%%%%%%%%%%%%%%%% MISC %%%%%%%%%%%%%%%%%%%%%%%%%%%%%%%%%%%%%%%%%%
\usepackage[acronym]{glossaries}
\usepackage{hyperref} % Setup: https://www.overleaf.com/learn/latex/Hyperlinks
\hypersetup{
	colorlinks=true,
	%citecolor=deepGreen,
	citecolor=maroon,
	linkcolor=persianBlue,
	filecolor=persianGreen,
	urlcolor=persianBlue,
	pdfpagemode=FullScreen,
}

%%%%%%%%%%%%%%%%%%%%%%%%%%%%%%%%%%%%%%%%%%%%%%%%%%%%%%%%%%%%%%%%%%%%%%%%%%%%%%%
\setcounter{tocdepth}{2}
\setcounter{secnumdepth}{2}

\newcommand{\hlred}[1]{\textcolor{Maroon}{#1}} % Print text in maroon
\newcommand{\hlgreen}[1]{\textcolor{persianGreen}{#1}} % Print text in green
\newcommand{\hlocre}[1]{\textcolor{ocre}{#1}} % Print text in green

\newenvironment{greenenv}{\color{Green}}{\ignorespacesafterend}  % Create green environment
\newenvironment{commentenv}{\color{ocre}}{\ignorespacesafterend}  % Create comment environment


\titleformat{\part}[display]
{\filleft\fontsize{40}{40}\selectfont\scshape}
{\fontsize{90}{90}\selectfont\thepart}
{20pt}
{\thispagestyle{epigraph}}

\setlength\epigraphwidth{.6\textwidth}

%\makeatletter
%\epigraphhead
%{\let\@evenfoot}
%{\let\@oddfoot\@empty\let\@evenfoot}
%{}{}
%\makeatother


%%%%%%%%%%%%%%%%%%%%%%%%%%%%%%%%%%%%%%%%%%%%%%%%%%%%%%%%%%%%%%%%%%%%%%%%%%%%%%%
%TODO LIST
\newlist{todolist}{itemize}{2}
\setlist[todolist]{label=$\square$}
\newcommand{\cmark}{\ding{51}}%
\newcommand{\xmark}{\ding{55}}%
\newcommand{\done}{\rlap{$\square$}{\raisebox{2pt}{\large\hspace{1pt}\cmark}}%
	\hspace{-2.5pt}}
\newcommand{\wontfix}{\rlap{$\square$}{\large\hspace{1pt}\xmark}}

%%%%%%%%%%%%%%%%%%%%%%%%%%%%%%%%%%%%%%%%%%%%%%%%%%%%%%%%%%%%%%%%%%%%%%%%%%%%%%%
\newcommand{\greensquare}{\marginnote{\fcolorbox{green}{green}{\rule{0pt}{3mm}\rule{3mm}{0pt}}\quad}}
\newcommand{\yellowsquare}{\marginnote{\fcolorbox{yellow}{yellow}{\rule{0pt}{3mm}\rule{3mm}{0pt}}\quad}}
\newcommand{\redsquare}{\marginnote{\fcolorbox{red}{red}{\rule{0pt}{3mm}\rule{3mm}{0pt}}\quad}}



%%%%%%%%%%%%%%%%%%%%%%%%%%%%%%% Title & Author %%%%%%%%%%%%%%%%%%%%%%%%%%%%%%%%
\title{Think Piece}
\author{Utku B. Demir}
%%%%%%%%%%%%%%%%%%%%%%%%%%%%%%%%%%%%%%%%%%%%%%%%%%%%%%%%%%%%%%%%%%%%%%%%%%%%%%%
%TODO: See if you can introduce a good spell checker like language tool
%
\begin{document}
\maketitle

% Political representation is in its overly simplified form, involves a principal-agent 
% relationship where representatives stand for and act on behalf of the represented, 
% usually on a territorial basis. It is a specific operationalisation 
% of the power deployment. Risking loss of generality by oversimplification, the political 
% representation can be seen as the delegation of the authority from a
% specific group to a political entity, either a representant or an
% organisation. As any utilisation of power, the representation introduces the question of
% legitimacy in its core. In its most classical understanding in the context
% of electoral democracy, the legitimacy of the representation is \textit{established} by a 
% strictly defined authorisation process and a framework of accountability
% \parencite{pitkin1967}. Yet, even in this seemingly unidirectional transfer 
% of power from constituency to representative,  still, the authorisation,
% the legitimacy of the source of the power, 
% and the question of the responsibilities of the one to exercise the power
% to the source of the power (both in terms of constituency, and the whole
% society) is a subject to a broad discussion \parencite[see e.g.;][]{kelsen2017,
% pitkin1967}. 
%

% Inline with Robert A. Dahl's definition of power as an
% actor's influence on another one to make them do something that probably
% wouldn't have been done by the other party \parencite*{dahl2007} underlines both the inherence
% and the potential magnitude of power the relationship/delegation between a
% representatative and constituency can deploy.
%
%
%TODO: You can introduce more discussion about the parag. above.
%TODO: More power discussion?
%
% However, the elective representation does not exhaust democratic 
% representation \parencite{saward2009}, nor a unidirectional authorisation
% process explain the nature of representative-constituency relationship. \hlred{The constructivist turn} \parencite{disch2019} in this sense emphasises the dynamic and fluid nature of political representation rather than seeing it as a static and pre-determined relationship between representatives and the represented. This shift in perspective argues that representation is an active, ongoing process of creation and negotiation. Representatives do not merely act on behalf of pre-existing groups with established interests and identities; rather, they play a crucial role in shaping and defining those interests and identities through their representational activities (ibid. 21ff). Although the discussion around the constructivist turn is often not unified in definition, it is roughly agreed on that the identity, interests, or preferences of the represented are not pre-determined but shaped\sidenote{Alternatively, \textit{extracted}.} by the representative; representation is being deployed on a much broader landscape than the electoral realm, which is in need of a large demystification process; and it all has been accumulated by the large-scale transformations the democracies went through \parencite[278f]{wolkenstein2024}.

However, elective representation does not exhaust all forms of democratic representation \parencite{saward2009}, nor does a unidirectional authorisation process adequately explain the nature of the relationship between representatives and their constituencies. \hlred{The constructivist turn} \parencite{disch2019} emphasises the dynamic and fluid nature of political representation, rather than viewing it as a static and predetermined relationship between representatives and the represented. This shift in perspective argues that representation is an active, ongoing process of creation and negotiation. Representatives do not merely act on behalf of pre-existing groups with established interests and identities; rather, they play a crucial role in shaping and defining those interests and identities through their representational activities (ibid. 21ff). Although the discussion around the constructivist turn is often not unified in its definition, it is generally agreed that the identity, interests, or preferences of the represented are not predetermined but are shaped\sidenote{Alternatively, \textit{extracted}.} by the representatives. Representation is deployed in a much broader landscape than just the electoral realm, which necessitates a significant demystification process; this has been precipitated by the large-scale transformations that democracies have undergone \parencite[278f]{wolkenstein2024}.

% The constructivist turn in representation challenges traditional models
% by highlighting the complexities and the evolving nature of decision-making 
% processes that increasingly transcend territorial boundaries 
% and involve new international and global players. This shift is accompanied by a growing influence of specialized and expert bodies, which often operate with only loose connections to traditional political institutions. This evolving landscape demands a reevaluation of the principles and practices of representation to better reflect the diverse and complex needs, characteristics, and identities of modern societies.
% Constructivist turn however, not just an observation of a shift in political
% representation. It is more a discovery that the manifestation of power is itself subject to ongoing negotiation and contestation, continually reshaped by both discursive and non-discursive practices, as it is not claimed that the constructivist part in the representation ever absent. The constructivist perspective initiates a discussion about the operation of the power in representation, and builds a new theoretical frame to examine the contemporary movements. Using the terms of Foucault's Power/Knowledge nexus, in the social
% realm the power is never completely consantrated in a specific point, nor
% it is ever a unidirectional process \parencite{Foucault1980}. Power is 
% everywhere and comes from everywhere, it is diffused throughout the
% structures of society (see ibid.); and the manifestation of power is just an
% instance of the signifier chain in the society instead of a solid entity. As
% the post-structuralist theory especially focuses on the nature of power
% whereas the subject, individuality, agency, and presence especially questioned,
% the examination and critique of the constructivist perspective finds a fertile
% ground here\dots

The constructivist turn, however, is not merely an observation of a shift in political representation. It represents a discovery that the manifestation of power is itself subject to ongoing negotiation and contestation, continually reshaped by both discursive and non-discursive practices. It is claimed that the constructivist element in representation is never absent. The constructivist perspective initiates a discussion about the operation of power in representation and builds a new theoretical framework to examine contemporary movements. Using the terms of Foucault's Power/Knowledge nexus, in the social realm, power is never completely concentrated in a specific point, nor is it ever a unidirectional process \parencite{Foucault1980}. Power is everywhere and emanates from everywhere; it is diffused throughout the structures of society (see ibid.), and the manifestation of power is merely an instance of the signifier chain in society rather than a solid entity. As post-structuralist theory particularly focuses on the nature of power, questioning the subject, individuality, agency, and presence, the examination and critique of the constructivist perspective finds fertile ground here\dots


% Lasse Thomassen's \parencite*{thomassen2019} exploration of Ernesto Laclau and Chantal Mouffe's work on populism is an examplary introduction to the post-structuralist reflection on representation. Populism is according to Laclau a political strategy that capitalises on the vagueness and fluidity of the \textit{people} to mobilise power. Populism, from Laclau’s perspective, exemplifies the performative nature of political representation -- it constructs the identity it seeks to represent. Populism's especially prominent approach to redefine the boundaries of public discourse makes it a good example how the production of a specicif representation directly translates to the amplification of certain views and others the marginalisation of others (ibid. 5ff). Similarly, a representation is a mapping \sidenote{Slightly different than the Thomassen's analogy to argue about mapping post-structuralism via map\&projection \parencite[540]{thomassen2017}, mapping is a mathematical concept which could be generally referred as a \textit{function} being used to discuss the aspects of representation here. A mapping assigns elements from a domain to codomain. Including, excluding, being defined through the nature of its association is inherent to a mapping. } which does not necessarily covers the domain it is operating on, and as the mapping happens, the production itself immediately reveals what it has excluded in the co-domain as much as what it included. Laclau's argumentation operates on the empity signifiers \sidenote{Or in other context also \textit{floating signifiers}.} that can filled in by the eexcluded in the other representations. The residue can be mapped in other mappings, leading to a common identity , a common will, a social and cultural hegemony which will eventually become a state \parencite[546f]{thomassen2017}. In this line of argumentation then, the populist articulation not necessarily an obstacle to the democratic to radical democratic struggles. As well as the horizontal, the vertical, hierarchical constructions of multitudes is in itself potentially make struggles possible. 

Lasse Thomassen's \parencite*{thomassen2019} exploration of Ernesto Laclau and Chantal Mouffe's work on populism is an exemplary introduction to the post-structuralist reflection on representation. According to Laclau, populism is a political strategy that capitalises on the vagueness and fluidity of the \textit{people} to mobilise power. Populism, from Laclau’s perspective, actively constructs the identity it seeks to represent. Populism's especially prominent approach to redefine the boundaries of public discourse makes it a good example of how the production of a specific representation directly translates to the amplification of certain views and the marginalisation of others (ibid. 5ff). Similarly, a representation is a mapping \sidenote{Slightly different than Thomassen's analogy to argue about mapping post-structuralism via map\&projection \parencite[540]{thomassen2017}, mapping is a mathematical concept which could be generally referred to as a \textit{function} used to discuss the aspects of representation here. A mapping assigns elements from a domain to codomain. Including, excluding, being defined through the nature of its association is inherent to a mapping.} which does not necessarily cover the domain it is operating on, and as the mapping happens, the production itself immediately reveals what it has excluded in the co-domain as much as what it included. Laclau's argumentation operates on the empty signifiers \sidenote{Or in another context also \textit{floating signifiers}.} that can be filled in by those excluded in other representations. The residue can be mapped in other mappings, leading to a common identity, a common will, a social and cultural hegemony which will eventually become a state \parencite[546f]{thomassen2017}. In this line of argumentation, the populist articulation is not necessarily an obstacle to democratic or radical democratic struggles. As well as the horizontal, the vertical, hierarchical constructions of multitudes themselves potentially make struggles possible.

% Thomassen's reading of Deleuze \& Guattari (D\&G) shows a clear difference to
% that of Laclau \& Mouffe. Deleuze and Guattari propose a more radical challenge to the very notions of representation and identity. They suggest that identity is not merely something to be represented but something that is constantly becoming, an ongoing process of change and differentiation that resists any final fixation in a representative structure. Thus, the constructivist approach is not merely about constructing constituencies; it is fundamentally about constructing subjects. It argues that the identity, interests, or preferences of the represented are not predetermined but are shaped by the representative in a broad landscape that extends beyond the electoral realm. This approach demands a reevaluation of representation to accommodate the complex, diverse, and continuously evolving needs and identities of modern societies(ibid. see 543-547). Furthermore, D\&G are binding their poltical imaginary also on their horizontally constructed rhizomatic understanding of the social surface. In this sense the \textit{subject} itself is a collection in itself, it is not a an individual but a collection of dividuals \sidenote{Or a machine constructed by other machines that is seeking connection to other machines.} \parencite[7ff]{deleuze1983}. The subject is an instance on a network, and as the subject, the representation is a network of networks, and the subject is not necessarily connected to these networks as a whole. Therefore, subject itself as well as, any representation of it is in a continuous de-- and reterritorialisation process. In relevance to the constructivistic argument D\&G offers a methodology and language to describe how the subject itself is also produced on a specific \textit{surface} of power, so the emancipation concept in their theory encourages nomadic behaviour and to cultivate/accumlate the desiring-production activity of subjects to build connections that would go out of the representational structures established in a specific power structure.
%

Thomassen's reading of Deleuze \& Guattari (D\&G) presents a distinct contrast to that of Laclau \& Mouffe. Deleuze and Guattari propose a radical challenge to the very concepts of representation and identity. They argue that identity is not merely something to be represented, but rather something that is continuously evolving, a dynamic process of change and differentiation that resists any final fixation within a representational structure. Thus, the constructivist approach is not just about forming constituencies; it fundamentally revolves around the formation of subjects. This perspective suggests that the identity, interests, or preferences of the represented are not pre-determined but are actively shaped by representatives across a landscape that extends beyond the electoral realm. This calls for a reevaluation of representation to address the complex, diverse, and continually evolving needs and identities of modern societies \parencite[543-547]{thomassen2017}. Moreover, D\&G anchor their political theory in a horizontally constructed rhizomatic understanding of the social fabric. Here, the \textit{subject} is seen not as an individual but as a collection of dividuals \sidenote{Or a machine constructed by other machines that seeks connections with other machines.} \parencite[7ff]{deleuze1983}. The subject is part of a network, and as such, representation itself becomes a network of networks, with the subject not necessarily connected to these networks in totality. Therefore, the subject, as well as any representation of it, is constantly undergoing processes of de- and reterritorialisation. Relevant to the constructivist argument, D\&G provide a methodology and language to discuss how the subject is also produced on a specific \textit{surface} of power, encouraging a nomadic approach and the cultivation of the desiring-production activities of subjects. This fosters connections that extend beyond the established representational structures within a given power framework.



%TODO: Come back to introduce constructivist turn

% Data and dividual tendencies
\begin{quote}

  But it is not just about the how you are represented but also "as whom". 
  You are represented, but you are represented as the given role in a specific 
  discourse realm.

  -- Lasse \cite{thomassen2017}.
\end{quote}


% Gayatri Spivak's critique of Deleuze and Foucault through "Can the Subaltern
% Speak" is especially relevant, as she challenges their assumption of the ability
% of subaltern to speak with their own voice \parencite[543]{thomassen2017}.
% However subaltern is only ever defined in a net of representation(s) which
% produced their subjectivity in the margins of already existing discourses, thus
% the subaltern can never speak for themselves because they are produced as
% agents within a particular discourse (see ibid.). Therefore, raising the
% question of how to pursue a strategy of flight from the \textit{hegemony}\sidenote{Hegemony might not be the best concept to describe it but Deleuze's lines of flight are ways to leave the gravity of the hegemonial discourses.} itself. 


Gayatri Spivak's critique of Deleuze and Foucault in "Can the Subaltern Speak?" is particularly relevant as it questions their presumption that the subaltern can speak independently \parencite[543]{thomassen2017}. However, the subaltern is always defined within a network of representations that have shaped their subjectivity from the margins of existing discourses. Thus, the subaltern cannot speak autonomously because they are constructed as agents within specific discourses (see ibid.). This provokes further inquiry into the possibilities of escaping the \textit{hegemony}\sidenote{While `hegemony' might not perfectly capture this concept, Deleuze's notion of `lines of flight' refers to escaping the constraints of hegemonic discourses.}, exploring paths of resistance against established power structures.




% This study aims to analyse the representation through its discourse under
% post-structuralism by focusing on how an emancipatory political imaginary.
% Relevant questions would be; If and how does representation emancipate? If and
% how a bottom-up constructed representation works? Is it possible to deploy such
% a representation outside of the hegemonial discourses? How is a dividual parts
% of the individual are mobilised in a constructed representation? Is it
% possible for a subject to without represantation, without being in a constituency? 
%


This study aims to analyse representation through its discourse under post-structuralism, particularly how it fosters an emancipatory political imaginary. Pertinent questions include: How does representation emancipate, if at all? How does a bottom-up constructed representation function? Is it feasible to deploy such representation outside hegemonic discourses? How are the dividual parts of an individual mobilised within a constructed representation? Is it possible for a subject to exist without representation, without belonging to a constituency?


% In order to examine some of these questions or questions to be discovered on
% the way, the study launches a plot device, \hlred{an AI algorithm}. The
% advancements in AI has made it especially relevant in the constituency
% construction processes, big data processing capacities made these algorithms
% especially relevant, since they could extract and exploit our latent dividuals,
% expressed in our data. Furthermore, through their control on relevance on the
% online realm by controlling the relevance of content, they were also impactful
% in creating specific echo chambers, filter bubbles which became influential in
% the constituency bonding processes. The advancements of generative AI
% algorithms have opened an even bigger topic which is and will be influencing
% algorithm—a theoretical entity equipped with the knowledge and capability to discern potential latent connections within the social sphere, potentially forging new communities. This tool enables the author to delve into and debate the nuances of political representation within post-structuralist political theory.
% the political sphere in myriad ways. But, this study is not trying to do a case
% based analysis, it is using a hypothetical AI algorithm, an entity that has
% knowledge and power on the potential latent connections in the
% social sphere that could also potentially construct communities. This device
% will allow the author to examine and discuss the reflections on political
% representation in the post-structural political theory. 
%


In order to examine some of these questions or discover new ones along the way, this study introduces a plot device: \hlred{an AI algorithm}. The advancements in AI have made it particularly relevant in the constituency construction processes; big data processing capabilities make these algorithms particularly useful, as they can extract and exploit our latent dividuals expressed in our data. Furthermore, by managing content relevance online, they significantly influence the creation of specific echo chambers and filter bubbles, which have become influential in the constituency bonding processes. The advancements in generative AI algorithms have opened up an even broader topic, which is and will continue to influence the political sphere in myriad ways. However, this study does not attempt a case-based analysis; instead, it uses a hypothetical AI algorithm, an entity that possesses knowledge and power over the potential latent connections in the social sphere, which could also potentially construct communities. This device will allow the author to examine and discuss the reflections on political representation in post-structural political theory.

\printbibliography
\end{document}
